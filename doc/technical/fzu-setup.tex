% vim: spelllang=en spell textwidth=120
\documentclass[deska]{subfiles}
\begin{document}

\chapter{Deska Setup at FZÚ}
\label{sec:fzu-setup}

\begin{abstract}
This chapter describes the process and deliverables which were implemented as a part of integrating the Deska system
with the pre-existing infrastructure of the Institute of Physics.  In particular, we discuss the features of the
database scheme, the process of the data conversion and the deployed output configuration generators.
\end{abstract}

\section{Database Scheme}
\label{sec:fzu-scheme}

The Deska system ships with a sample database scheme which demonstrates all of the Deska's features, but is too simple
to be directly usable for real-world needs of a typical data center.  We have therefore developed a second scheme, the
{\tt fzu} one, which serves as an example of a database layout that could be used in production environment.

\subsection{Supported Features}

The very first requirement which is placed on the scheme is to contain all of the information which is already managed
by the in-house application (see section \ref{sec:fzu-farmdb} further in this chapter) and make it available in an
efficient and easy-to-use manner.  We have also decided to add a few experimental features which might provide further
assistance with the everyday operation, like being able to keep a maintenance log of the hardware issues directly in the
database instead of using a separate application.  This feature was added late in the system design on request of the
Institute's administrators, but proved to be easy enough to implement thanks to the flexibility of the user-defined
scheme.

At the same time, we were forced to limit the scope creep of the whole system to maintain a reasonable level of
usability.  One example of a feature which we have postponed is keeping track of the individual disk drives' serial
numbers, a suggestion which appeared in November 2011.  The Deska system is ready to implement this
feature,\footnote{It's just a matter of adding new kinds {\tt diskarray} and {\tt drive} embedded into each other and
having the {\tt diskarray} be contain the existing {\tt box}, a setup which is already verified by the {\tt hardware}
and {\tt switch} pairs and the embedded {\tt interface}.} but given that the {\tt fzu} database scheme already contains
seventeen distinct kinds and that the source data for the individual serial numbers was incomplete at best,\footnote{The
data came from a variety of Excel-based sources from multiple vendors, requiring a labour-intensive manual process which
is clearly out-of-scope of the Deska project.} we have deemed this feature not important enough to warrant a late change
of the scheme layout.

\subsection{Deska Kinds}

Most of the database focuses on keeping track of the individual servers.  Each server is represented by an instance of
the {\tt hardware} kind which contains basic information like its serial numbers and a reference (see the {\tt
REFERS\_TO} relation in \secref{sec:relation-refers-to}) to the corresponding hardware model, the {\tt modelhardware}
object.  Rack placement of the server is implemented through the {\tt box} kind which is {\tt CONTAINABLE}
(\secref{sec:relation-contains}) to the {\tt hardware}, and --- to reuse the existing code --- which also provides the
same information through the {\tt CONTAINABLE} relation to the other kind which represent physical objects, the {\tt
switch}.  At the same time, {\tt box} instances, when not contained into any other objects, model standalone
``enclosures'' or physical racks.  Properties which are valid for multiple boxes at once are defined at the level of a
{\tt modelbox}.

Each server can be either kept out of operation, or be running a particular operating system and providing some
services.  At the same time, the Institute of Physics also uses virtualization to consolidate their computing needs and
increase resilience to various hardware issues.  The scheme therefore maintains a distinction between a logical {\tt
host} and the underlying {\tt hardware}.  The {\tt host} object represent an instance of an operating system running on
``some machine'', be it a virtual one or a real, physical server.  If the server is physical, the {\tt host} will
contain a {\tt hardware} object, while in case the host is a virtual one, the {\tt host} contains a {\tt
virtual\_hardware} instance.  In both cases, the {\tt host} will typically utilize the {\tt EMBED\_INTO} relation (see
\secref{sec:relation-embed-into}) to embed network interfaces, objects of the {\tt interface} kind, to represent network
connections.  An interface can be optionally plugged into a port of a {\tt switch}.  A switch is similar to the {\tt
hardware} in that it contains an instance of the {\tt box} object to specify its physical location and uses the {\tt
REFERS\_TO} to point to the corresponding {\tt modelswitch}.

In order to streamline mass installation of new machines, the {\tt hardware} kind supports templates (see
\secref{sec:relation-templatized}) with the {\tt hardware\_template} kind.  The provided conversion scripts do not use
these templates directly, for the templates are intended to reduce the {\em human} effort required to process a delivery
of {\em new} machines.

Assignment of the logical services to the individual {\tt host}s is done through the {\tt service} kind.  Its usage
accurately matches the description in \secref{sec:relation-multi-value-references} about object tagging.

The scheme also contains kinds {\tt warranty\_contract} which store information about warranty and service agreements
for each delivery of machines, a {\tt vendor} that keeps track about the manufacturer of individual items, a {\tt
network} that provides information like the VLAN number and allowed IP address range for the network interfaces, and a
pair of the {\tt extrahw} and {\tt modelextrahw} kinds that represent other pieces of equipment than servers and
Ethernet switches and therefore serve as a ``catch-all'' storage for all other devices, from disk arrays to patch
panels, rack blinds, cable holders and Fibre Channel switches.

Finally, the {\tt formfactor} is used as a helper compatibility object which specifies which objects ``fit'' into
physical boxes, so that one cannot ``insert'' a rack-mount server into a blade chassis, or a small form factor twin node
directly into a rack, while the {\tt event} can be used as a log of all problems which concern a physical object of any
kind.

The following table provides a compact reference about the defined kinds:

\begin{longtable}{ l | p{10cm}}
    \caption{Kinds defined in the FZÚ database scheme} \\
    Kind & Description \\
    \hline
    \endhead
    \label{tbl:fzu-kinds}
    {\tt service} & A list of defined roles of the hosts, which is referenced from the {\tt host}'s {\tt service}
    multi-value attribute \\
    {\tt vendor} & A list of equipment manufacturers referenced from various {\tt model*} kinds \\
    {\tt warranty\_contract} & Representation of a service contract and support agreement.  Instances of this kind are
    typically referenced from multiple {\tt hardware}, {\tt switch} or {\tt extrahw} objects at once. \\
    {\tt formfactor} & An auxiliary class serving as a target of the {\tt modelbox}'s attributes {\tt accepts\_inside}
    and {\tt formfactor} that keeps physical compatibility of the box nesting \\
    {\tt modelbox} & Shared properties of a box.  These objects shall define information like the outer dimensions of a
    rack, number of internal bays which can accommodate further boxes or the outer and accepted form factors.  Instances
    of this kind are associated to each {\tt box} either indirectly, through the corresponding {\tt modelhardware},
    {\tt modelswitch} or {\tt modelextrahw} attributes, or --- in case of boxes without any piece of equipment to match
    --- directly via the {\tt direct\_modelbox} attribute. \\
    {\tt box} & Using one instance for each physical object located in a machine room, these objects specify their
    physical placement \\
    {\tt modelhardware} & Common hardware-related properties of a server like the memory, disk or CPU configuration \\
    {\tt hardware} & A physical server \\
    {\tt virtual\_hardware} & One object for each instance of a virtual machine which runs a given {\t host} \\
    {\tt host} & An operating system instance running either on a physical hardware or inside a virtualized environment
    \\
    {\tt modelswitch} & Common properties of each model of an Ethernet switch \\
    {\tt switch } & A physical Ethernet network switch \\
    {\tt network} & Auxiliary object defining allowed IP addresses and VLANs for each logical network in use \\
    {\tt interface} & A network interconnect from a {\tt host} (into which this instance is embedded) into an (optional)
    {\tt switch} \\
    {\tt event} & Represents one item in a log of a {\tt box}' issue history \\
    {\tt modelextrahw} & A catch-all hardware for representing physical objects which are more complicated than a
    ``dumb rack'', but are unusual enough to not warrant their own top-level kind \\
    {\tt extrahw} & An instance of a special-purpose hardware of the {\tt modelextrahw} type
\end{longtable}

\section{Data conversion}
\label{sec:fzu-farmdb}

The Institute of Physics has had a web-based application for inventory management for a few years.  This
ASP.NET~\cite{asp.net} based application has served well, but has focused mostly on keeping track of rack placement of
the hardware, leaving much of more complicated tasks like assigning roles to machines unsolved.  The database also mixed
different sorts of items in one table (i.e. it made no distinctions between a typical server and an Ethernet switch or a
Fibre Channel disk array).  Converting this information to a highly structured data required by the Deska scheme has
proven challenging.  The process was not helped by the fact that the existing database, the FarmDB, used dimensions in
``rack fractions'' where a twin-sized SGI machine has width of 50\%, while the Deska scheme uses a hierarchy of nested
boxes and internal offsetting.

The conversion started with processing an SQL dump, constructing the objects and maintaining an in-memory representation
of the database.  The dictionary of objects was then walked and a dump suitable for processing by the Deska CLI
(\secref{sec:cli-usage}) got produced.

The full sources of the conversion script are available in the Deska source tree (\path{doc/farmdb/convert.py}), along
with the source data (\path{doc/farmdb/dbo.*.Table.sql}) and a helper script (\path{doc/farmdb/deunicodeify.sh}).

\section{Configuration Generators}
\label{sec:fzu-cfggen}

\section{Generator Scripts}

The rest of this chapter provides a quick overview of the configuration generators deployed at the FZU for generating
production configuration.

The configuration generators are available from the directory \path{scripts/config-generators/} in the Deska source tree
(and also on the attached CD).  The following scripts are delivered:

\begin{description}
    \item[01-demo] A demo script which walks the Deska database, printing all data it finds.  Unlike other programs,
        this script is not used in production, but just used as a simple example.
    \item[02-dhcp-dns] As the name implies, this script produces configuration files for the DNS and DHCPD servers.  As
        the required information is very similar for both of these use cases, they are grouped together, and in addition
        outputs the configuration of network switches as well.
    \item[03-text-draw-rack] This generator produces a textual representation of the racks and physical placement of all
        equipment in the QML format.
    \item[04-qml-rack] The fourth program converts the output of the \path{03-text-draw-rack} into a graphical
        representation.  The QML format is described later in this chapter.
    \item[05-services] This program produces output for various services whose input is driven by the services assigned
        to the individual machines, and therefore makes heavy use of the {\tt service} kind.  At this level, the
        configuration for the Nagios, Ganglia and Munin servers is manufactured.
    \item[06-hw-management] The final script generates a configuration file for the add-on scripts that were developed
        as a further demonstration of the Deska's power.  The produced file is merely a Python's {\tt pickle}
        representation of the models of all servers.
\end{description}

A careful reader will immediately recognize that we have indeed followed our own suggestions about implementing the
configuration generators efficiently.  The presented scripts take care not to query the database too often and try hard
to group configuration generators into similar pieces where possible.

\section{Rack View}
One of the outputs provided by the Deska system is producing a graphical representation of all the racks in the data
center.  We have chosen to utilize the excellent Qt's QML library.  The input file required by the QML application is a
text file which is easy to understand.  Furthermore, the produced data can be not only used for rendering a simple,
static image, but could be also used to define the graphical elements in a possible future GUI application.  Using QML
as the intermediate layer allows reusing the existing code.

However, using QML also mandates an extra conversion step before an actual image in a widely supported format emerges.
We have originally intedned to use Qt's own {\tt qmlviewer} application, but it proved to be hard to get static images
in a deterministic manner (the main use case is rendering complex QML-based animations into movie files for demos).  We
have therefore built a standalone Qt-based application for converting a QML input to a static image.  The utility which
supports converting a QML file into the PNG or SVG formats is available in source form in the Deska sources under the
\path{\src/qml2image/}.  Please pass the {\tt -DBUILD\_QML2IMAGE=1} flag to {\tt cmake} to enable its building.

\section{Working with State}

The configuration generators are intended to work in a stateless manner (see \secref{sec:config-generators}).  However,
certain applications, like the Bind zone file for DNS, contain a serial number which is supposed to be changed upon each
modification.  The \path{02-dhcp-dns} script is a nice, albeit a bit complicated example of how to work with this state.

The desired mode of operation is implemented by a special wrapper class instead of a plain {\tt file} object.  The class
intercepts the data written to a file, comparing the data with the previous copy, and if it detects a difference from
the previously saved state, it uses the original serial number and the current date to create a new identification in
conformance with the relevant RFC documents.

\end{document}
