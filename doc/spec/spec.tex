\documentclass{article}
\usepackage[utf8]{inputenc}
\usepackage[czech,english]{babel}
\usepackage{a4wide}

\begin{document}

\title{Deska: A Tool for Central Administration and Monitoring of a Grid Site}

\author{Jan Kundrát \and Martina Krejčová \and Tomáš Hubík \and Lukáš Kerpl}

\maketitle

\section{Background}

One of the activities taking place at the Institute of Physics of the~AS~CR (FZU)~\cite{fzu} is running a local Tier-2 computing
center connected to an international WLCG computing grid~\cite{wlcg}, supporting user programs from various scientific communities
from the whole world including the D-0 experiment in Fermilab~\cite{d0} and the famous LHC accelerator from CERN~\cite{lhc}.

\section{Motivation}

In December 2010, the total installed computing capacity available to users was about 3000~CPU cores with more than 500~TB of disk
storage, which translates to several hundreds of physical machines, with more hardware to arrive in early 2011.  All the resources
are managed by just a few system administrators (currently slightly more than three full time equivalents).  Due to the impressive
machines-per-administrator ratio, the staffers are exploring ways about how to make the system administrators' life easier and
less error prone.

Over the years, certain monitoring and management systems were deployed at the Institute, including Cfengine~\cite{cfengine},
Nagios~\cite{nagios}, Ganglia~\cite{ganglia}, Munin~\cite{munin} and various in-house monitoring appliances.  While these tools
certainly aim to make the sysadmin's life easier, introduction of each new tool also adds one more place for the administrator to
check if she wants to make sure the infrastructure works as expected, and each new tool has to be configured separately.
Therefore, it has been decided that the Institute shall fund a development of a tool which will {\em automate the configuration
process} and focus on bringing the control back to one central place.

\section{System Requirements}

The first goal of the project is to develop a database storing generic description of all the resources that the Grid
site is using; such a database should be capable of describing hardware infrastructure of a data center\footnote{That is all
physical machines, their hardware models, part numbers, rack locations, network interconnects, history of hardware failures etc},
coupling of operating system instances (aka ``hosts'') to physical machines as well as logical relations between various
services and their dependencies.  This database shall then act as the single authoritative source
of information for generating configuration of all components of the system.

FIXME: data scattered among several places... 

\subsection{Generic Database}

Due to the Institute's administrators' strong Unix background and on their explicit request, it is expected that the interface to
the database shall be provided via a CLI application instead of using a traditional web-based approach.  The main reason for this
design choice is their familiarity with text-based configuration tools and bad experience with web applications in general.  The
application interface shall be similar to a CLI-oriented interface used on enterprise-grade Ethernet switches, notably to the
Cisco IOS shell.  The format shall enable storing of plaintext dumps of the database in a version control system for auditing
purposes.

History has proven that it is extremely hard to give accurate predictions about future developments and trends in computing.  Many
contemporary inventory management tools are severely limited by constraints from the past.  Even though the whole concept of
virtualization predates the moment when personal computers became commodity, it is still common to stumble upon tools which cannot
keep track of virtual machines or do not support the ``blade'' form factor of physical servers.  Therefore, the Deska DB shall not
impose similar restrictions to its data model, and the core DB code should only assume that it is working with {\em collections of
objects} and some {\em relations between them} -- actually defining a usable {\em scheme} of the database is a separate problem
from the general database design.

As has been demonstrated numerous times during the Institute's course of operation, keeping track of all modifications is crucial
for finding out what caused a particular outage.  Therefore, the database shall support {\em versioning} of the records contained
therein.  We do not expect a need to support non-linear history (especially branching and merging), though.

User authentication and authorization shall be handled by already existing operating system facilities (PAM, SSH) and provisions
for future ACL-checking hooks shall be considered.  The whole system shall be able to scale up to thousands of managed objects on
a commodity x86\_64 server hardware.

\subsection{Database Scheme}

In addition to the general-purpose database described above,
the team shall design a scheme suitable for describing the whole IT infrastructure of the WLCG part of the Institute.
This scheme should be extensible to allow installing future pieces of equipment without large-scale code changes, and shall try to
anticipate future trends in hardware development, as well as account for more exotic hardware which could be found at Institute's
partners' sites.

Nonetheless, the design of the general database shall be influenced by the real-world DB scheme, and shall offer reasonable
assistance for administrators willing to extend the structure.  The goal here is not to achieve purity by all means, but implement
a solution which is easy to deploy, use and maintain. An example of such assistance could be a script for converting the
user-supplied database scheme into one directly usable in the Deska database. 

\subsection{Database API and Add-on Modules}

The Database shall provide an API for access to the data contained therein. The first, read-only part of the API shall be complete
enough to offer all functionality required by various modules and utilities described below.  The second part of the API shall
be used by the CLI interface for performing actual modifications to the data.  It is up for the implementors to decide on the
technical means of the API's "on-the-wire" protocol.

Bindings to scripting languages like Python shall be introduced, as the staffers prefer writing their own configuration backends
in a language which allows faster prototyping than C++.  Therefore, the API shall offer a complete bindings for Python, along with
well documented sample modules.

Additional components for generating configuration on basis of the actual system state, as retrieved through the read-only
database API, shall be developed for all systems which are currently deployed at FZU's site, namely for Nagios, BIND name service
demon, DHCP server and Ganglia and Munin master servers.  The design of the API should reflect generic needs of any data center to
allow eventual deployment of new components, should the need arise.  All modules should use only the public and documented API for
retrieving all of the data they need.

In addition, tools shall be written to perform consistency checks between the state recorded in the database and the actual system
shape where the automatic configuration generation is not feasible for safety reasons\footnote{A classic example of such a system
are network switches -- system administrators are rather leery when it comes to automated configuration as a single error can
cause large network segments to separate.}.

Furthermore, in order to streamline the day-to-day operation of the Institute, the CLI application shall include ``wizards'' for
common tasks like deploying a new machine.

\subsection{Visualization of the Datacenter Layout}

The database API shall be used to employ the contained data for providing a graphical visualization of all computer rooms and all
other equipment.  These views shall be integrated with the pre-existing monitoring tools to provide more intuitive overview of
possible issues encountered during operation.  It has not been decided yet whether these tools are going to be web-based or
implemented as a desktop application.

\subsection{Management Utilities}

The information contained in the database could be used for variety of other tasks besides generating configuration files.  One
practical use case is creating a unified interface for console access to the physical machines.  This interface shall build on the
existing tools for handling the actual access (likely a SSH tunnel, a standard KVM access or perhaps a Java-based browser method),
but encapsulate all the details like physical location from the user.

A similar application is controlling the host power state; again, the goal here is not to reimplement the OpenIPMI implementation,
but hide various differences in the protocol spoken by various machines, automatically using correct credentials for access etc.

These management utilities are likely going to be very site-specific, but effort shall be taken to attempt to properly separate
local assumptions about concrete access method from the rest of the utility.

\section{Problems to Solve}

While the specification of an ``inventory management tool'' could be perceived as an easy task; that is, however, not
the case.  The biggest complication of the whole design is finding a right balance between usability and completeness.  The
Institute has had a web-based inventory management application capable of tracking various pieces of data for years, but the
staffers were hesitant to use it and, more importantly, to {\em depend} on it for day-to-day operation.  As per an explicit request given
by the staffers, this project shall develop a console-based tool with strong focus on scripting, ease of function and the general
feel of {\em helping} them with getting things done instead of building artificial obstructions to their work for the sake of "evidence".

Given the always-changing nature of the data center operations, it is unlikely that a static database structure would accommodate
future developments.  Therefore, the structure of the database shall not hard-code any current assumptions about how a datacenter looks
like, but shall rather provide a way to describe generic {\em collections of objects and their hierarchies}.  Needless to say,
such an approach puts even bigger pressure on proper design of the system (and the database structure implemented at the Institute), as
finding a balance between being ``generic enough'' and ``usable enough'' is extremely complicated.  This neccessary complexity, which
is required in the DB level, shall not be obvious to the users when working with the UI -- indeed, the interface should be
intuitive enough so that the administrator will actually notice the improvements and start to like the system in favor of
manual hacks. 

The requirement of a dynamic database structure also makes the issue of versioning a rather complex problem, as it is now required
to implement history management of almost arbitrary data. 

The database shall offer demonstratable strength when faced with mallicious input.  It shall employ industry-standard means of
defying security threats, and be friendly to firewall.  The Institute is going to depend on the availability of the Database for
its core operation, so it means that it should be more reliable than typical gLite middleware service.  The database should also
scale to tens of thousands of managed nodes on contemporary server hardware.  

\section{Future Improvements}

The system shall be extensible enough to be used outside of the Institute's computing infrastructure.  A nice feature would be
also a ``one-click'' tool for fully automated deployment of new machines, handling everything from setting up PXE environment for
installation to actual system configuration.

Portability should be a concern, too, but it is expected that the target platform will be Linux on x86\_64 systems.  The Institute
is a heavy user of a custom Linux distribution based on Red Hat Enterprise Linux 5, which is going to be the platform of choice
for running the server.  The clients shall be reasonably easy to install on modern desktop Linux distributions.

\begin{thebibliography}{9}
    \bibitem{fzu}Institute of Physics of the AS CR home page, {\tt http://www.fzu.cz/}
    \bibitem{wlcg}The W-LCG LHC Computing Grid, {\tt http://lcg.web.cern.ch/LCG/}
    \bibitem{d0}The DZero Experiment, {\tt http://www-d0.fnal.gov/}
    \bibitem{lhc}The Large Hadron Collider, {\tt http://lhc.web.cern.ch/lhc/}
    \bibitem{cfengine}Cfengine, {\tt http://www.cfengine.org/}
    \bibitem{nagios}Nagios, {\tt http://www.nagios.org/}
    \bibitem{ganglia}Ganglia, {\tt http://ganglia.sourceforge.net/}
    \bibitem{munin}Munin, {\tt http://munin.projects.linpro.no/}
\end{thebibliography}

\end{document}
