\documentclass[12pt]{article}
\usepackage[utf8]{inputenc}
\usepackage[english,czech]{babel}
\usepackage{a4wide}
\author{Tomas Hubik}
\title{Deska}



\begin{document}

{\Huge \textbf{Deska}}

\vspace{0.2in}

{\large Tool for Central Administration of a Grid Site}

\vspace{0.5in}

{\large Scope Document}

\vspace{0.2in}

{\large Prepared for FZU AV CR}

\vspace{0.2in}

{\large Version 1.0}

\vspace{0.2in}

{\large Date 2009-Oct-22}

\vspace{0.5in}

\subsection*{Revision History}

\begin{table}[!h]
	\begin{tabular}{l l l l}
		\textbf{Date} & \textbf{Version} & \textbf{Author} & \textbf{Description} \\
		2009-Oct-23 & V1.0 & Tomas Hubik & Master plan \\
	\end{tabular}
	\label{tab:RevisionHistory}
\end{table}


\subsection*{Document Review}

\begin{table}[!h]
	\begin{tabular}{l l l l}
		\textbf{Date} & \textbf{Version} & \textbf{Reviewer Name} & \textbf{Contact Info.} \\
		2009-Oct-22 & V1.0 & Peter Parker & p.parker@bayview.com \\
	\end{tabular}
	\label{tab:DocumentReview}
\end{table}


\subsection*{Document Approval}

\begin{table}[!h]
	\begin{tabular}{l l l l}
		\textbf{Date} & \textbf{Version} & \textbf{Reviewer Name} & \textbf{Contact Info.} \\
		2009-Oct-25 & V1.0 & Peter Parker & p.parker@bayview.com \\
	\end{tabular}
	\label{tab:DocumentApproval}
\end{table}


\newpage

\tableofcontents

\newpage

\section{Introduction}

\subsection{Purpose}
This document defines the project scope for the Tool for Central Administration of a Grid Site. It establishes the business need for the Deska, and it outlines the high level requirements needed to satisfy the specified need. This document is not an exhaustive requirements description, but will instead provide overall direction for a separate -- more detailed -- set of system requirements. This document's primary purpose is to establish priority among the most important issues, considerations, features, and overall goals.

\subsection{Scope}
The Deska will be responsible for managing and monitoring of a grid site. Especially it automates configuration of several monitoring and deployment tools. All functions and features will be displayed and managed through a web interface.


\section{Business Opportunity}

\subsection{Background}
To be written by Honza. TODO
 
\subsection{Positioning}
The system will not replace any of the existing monitoring tools. It will replace only their web interfaces and display all the information by itself in some well-arranged way. It will substitute the unpractical way of configuration of currently used tools, that must be made manually by editing of several files now.

\subsection{Impact of Missed Opportunity}
The fundamental benefit of this system is that it will save time of administrators, that can focus on other things. It will also decrease the possibility of making a mistake while adding new machine and resulting time consuming finding of this mistake.


\section{Proposed Solution}

\subsection{Primary Functional Requirements}
In this section, we classify the primary features of the Tool for Central Administration of a Grid Site across three categories. Essential features cannot be done without. High-value features can be done without, although it may be very undesirable to do so. Follow-on Features are those for which it is not clear they should be included in the first release. In all cases, the lists are not exhaustive but include the most important features from a business perspective.

\subsubsection{Essential Features}
\begin{itemize}
\item Feature TODO
\end{itemize}

\subsubsection{High-Value Features}
\begin{itemize}
\item Feature TODO
\end{itemize}

\subsubsection{Follow-on Features}
\begin{itemize}
\item Feature TODO
\end{itemize}

\subsection{Primary Non-Functional Requirements (NFRs)}
In this section, we identify the primary non-functional requirements for the Tool for Central Administration of a Grid Site. This list is not exhaustive but is intended to capture the most important requirements from a business perspective. A full list of NFRs will be included in the SRS document.

\subsubsection{Performance, Throughput, and Scalability}
Performance TODO\\
The system will provide a hi level of scalability, so we will be able to run it over various monitoring and deployment tools. Configuring of single applications will be made by modules. These modules will use our system's API, so creating a new module for another application will be an easy task.

\subsubsection{Reliability and Availability}
Because of the fact, that our system will not replace any of the functional parts of current system, it is not necessary to be available 7 by 24 by 365.\\
It is necessary to be well protected against failure while generating configuration files, because this failure can lead to failure of whole grid site. This will be made by some versioning of these files, so there will be some way back if this failure occurs.

\subsubsection{Security}
Several features of the system require authorization. The system will be divided into two security levels. First can only view grid status and second can beside the viewing part make also changes. For example adding a new machine, changing machine roles, removing them and so on.

\subsubsection{Usability}
The application will be used by administrators who speak english, so the system will initially be translated into english.\\
It is also expected, that administrators have some experience with grid site maintenance.


\section{Risks}
Don't know. TODO


\section{Constraints}

\subsection{Development Process and Team Constraints}
At this time, there are no known constraints.

\subsection{Environmental and Technology Constraints}
At this time, there are no known constraints. TODO

\subsection{Environmental and Technology Constraints}
The application will be hosted on some machine with UNIX type operating system. The web part of the system will be browser independent.\\
TODO Maybe database type, is it necessary to mention language/framework used?

\end{document}