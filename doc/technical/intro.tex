% vim: spelllang=en spell textwidth=120
\documentclass[deska]{subfiles}

\begin{document}

\chapter{A Gentle Introduction}

\begin{abstract}
The first chapter explains what Deska is, and what it tries to achieve.
\end{abstract}

\section{Motivation}

The Deska project started as an internal project at the Institute of Physics of the AS CR, v.v.i\@.  Its goal is to
provide a central place for storing all information about the infrastructure of a typical grid computing center, and
offer a suite of add-on scripts which use that knowledge to make everyday operation run smoother.  A central focus of
the design was to come up with a system that propagates changes to proper places, never leaving stale content behind,
but also provides a detailed accounting log of all changes performed.

Due to the never-stopping evolution in the data center design where one cannot reliably anticipate what innovations are
coming in a few years, the system we have built sports an extensible framework that allows us to manage entities which
were unknown at the design phase.  We propose a data model that, to our best knowledge, provides enough features to
sustain the real-world growth of grid center operation, yet is simple enough to work with even for staffers without deep
knowledge of relation algebra.  For more details on how the model operates and what features are supported, please see
\secref{sec:objects-and-relations}.

\section{Implementation}

The Deska system, as delivered, can be divided into four parts.  The first part, which is the most visible to the actual
user, is the CLI console, a text-mode application that connects to the Deska server and is intended to be used for
day-to-day work, including modifications and reporting.  The CLI console is implemented in C++.  Second part, the
database server, translates the domain-specific JSON-based wire protocol called {\em Deska DBAPI} (see
\secref{sec:dbapi-protocol}) into calls to stored procedures in the PostgreSQL server, which perform any required
modifications or retrieve the data.  The database server is implemented in a mixture of Python and PL/PgSQL languages
and provides object-level versioning for all data contained therein, with features similar to Subversion.

Unlike the traditional systems, in Deska, the structure of the database itself is not set in stone, but can be tweaked
and changed by the Deska administrator, provided she follows certain rules.  Nevertheless, the standard distribution
ships a database scheme which closely models the real-world needs of the computing center in Prague.  This scheme is the
third part of the whole suite, and comprises tens of individual table definitions in standard SQL.  Any required
information about their mutual relations is communicated in the terms of Deska relations (see
\secref{sec:objects-and-relations-relations}) using conventional SQL features.

Finally, for the Deska database to actually implement useful functions, a set of scripts is shipped that uses the
information contained in the database and makes it available to various services.  This group is implemented in pure
Python and contains a layer which presents the database contents as a native Python object hierarchy.  This is further
used to create, or {\em generate}, configuration for various services which are in use in Prague, from DHCP servers and
DNS to network switches.  Included is a complete setup which makes sure that each revision, as stored in the Deska
database, is available in an external VCS repository.

\end{document}
