% vim: spelllang=en spell textwidth=120
\documentclass[deska]{subfiles}
\begin{document}

\chapter{Deska Setup at FZÚ}
\label{sec:fzu-setup}

\begin{abstract}
This chapter describes the process and deliverables which were implemented as a part of integrating the Deska system
with the pre-existing infrastructure of the Institute of Physics.  In particular, we discuss the features of the
database scheme, the process of the data conversion and the deployed output configuration generators.
\end{abstract}

\section{Database Scheme}
\label{sec:fzu-scheme}
\todo{Lukas: fzu DB scheme}

\section{Data conversion}
\label{sec:fzu-farmdb}

The Institute of Physics has had a web-based application for inventory management for a few years.  This
ASP.NET~\cite{asp.net} based application has served well, but has focused mostly on keeping track of rack placement of
the hardware, leaving much of more complicated tasks like assigning roles to machines unsolved.  The database also mixed
different sorts of items in one table (i.e. it made no distinctions between a typical server and an Ethernet switch or a
Fibre Channel disk array).  Converting this information to a highly structured data required by the Deska scheme has
proven challenging.  The process was not helped by the fact that the existing database, the FarmDB, used dimensions in
``rack fractions'' where a twin-sized SGI machine has width of 50\%, while the Deska scheme uses a hierarchy of nested
boxes and internal offsetting.

The conversion started with processing an SQL dump, constructing the objects and maintaining an in-memory representation
of the database.  The dictionary of objects was then walked and a dump suitable for processing by the Deska CLI
(\secref{sec:cli-usage}) got produced.

The full sources of the conversion script are available in the Deska source tree (\path{doc/farmdb/convert.py}), along
with the source data (\path{doc/farmdb/dbo.*.Table.sql}) and a helper script (\path{doc/farmdb/deunicodeify.sh}).

\section{Configuration Generators}
\label{sec:fzu-cfggen}

\section{Generator Scripts}

The rest of this chapter provides a quick overview of the configuration generators deployed at the FZU for generating
production configuration.

The configuration generators are available from the directory \path{scripts/config-generators/} in the Deska source tree
(and also on the attached CD).  The following scripts are delivered:

\begin{description}
    \item[01-demo] A demo script which walks the Deska database, printing all data it finds.  Unlike other programs,
        this script is not used in production, but just used as a simple example.
    \item[02-dhcp-dns] As the name implies, this script produces configuration files for the DNS and DHCPD servers.  As
        the required information is very similar for both of these use cases, they are grouped together, and in addition
        outputs the configuration of network switches as well.
    \item[03-text-draw-rack] This generator produces a textual representation of the racks and physical placement of all
        equipment in the QML format.
    \item[04-qml-rack] The fourth program converts the output of the \path{03-text-draw-rack} into a graphical
        representation.  The QML format is described later in this chapter.
    \item[05-services] This program produces output for various services whose input is driven by the services assigned
        to the individual machines, and therefore makes heavy use of the {\tt service} kind.  At this level, the
        configuration for the Nagios, Ganglia and Munin servers is manufactured.
    \item[06-hw-management] The final script generates a configuration file for the add-on scripts that were developed
        as a further demonstration of the Deska's power.  The produced file is merely a Python's {\tt pickle}
        representation of the models of all servers.
\end{description}

A careful reader will immediately recognize that we have indeed followed our own suggestions about implementing the
configuration generators efficiently.  The presented scripts take care not to query the database too often and try hard
to group configuration generators into similar pieces where possible.

\section{Rack View}
One of the outputs provided by the Deska system is producing a graphical representation of all the racks in the data
center.  We have chosen to utilize the excellent Qt's QML library.  The input file required by the QML application is a
text file which is easy to understand.  Furthermore, the produced data can be not only used for rendering a simple,
static image, but could be also used to define the graphical elements in a possible future GUI application.  Using QML
as the intermediate layer allows reusing the existing code.

However, using QML also mandates an extra conversion step before an actual image in a widely supported format emerges.
We have originally intedned to use Qt's own {\tt qmlviewer} application, but it proved to be hard to get static images
in a deterministic manner (the main use case is rendering complex QML-based animations into movie files for demos).  We
have therefore built a standalone Qt-based application for converting a QML input to a static image.  The utility which
supports converting a QML file into the PNG or SVG formats is available in source form in the Deska sources under the
\path{\src/qml2image/}.  Please pass the {\tt -DBUILD\_QML2IMAGE=1} flag to {\tt cmake} to enable its building.

\section{Working with State}

The configuration generators are intended to work in a stateless manner (see \secref{sec:config-generators}).  However,
certain applications, like the Bind zone file for DNS, contain a serial number which is supposed to be changed upon each
modification.  The \path{02-dhcp-dns} script is a nice, albeit a bit complicated example of how to work with this state.

The desired mode of operation is implemented by a special wrapper class instead of a plain {\tt file} object.  The class
intercepts the data written to a file, comparing the data with the previous copy, and if it detects a difference from
the previously saved state, it uses the original serial number and the current date to create a new identification in
conformance with the relevant RFC documents.

\end{document}
