% vim: spelllang=en spell textwidth=120
\documentclass[deska]{subfiles}
\begin{document}

\chapter{Installing Deska}
\label{sec:building}

\begin{abstract}
This chapter guides the reader through the installation process of the Deska application suite.
\end{abstract}

\section{Quick Evaluation}

The installation procedure of the Deska server is a rather complicated process, as it involves separate configuration of
the server and clients, and optionally also a proper configuration for the output configuration generators.  In order to
quickly evaluate the whole Deska setup, there is also a sample script which installs all required packages, deploys the
database and fills it with sample data.

Do {\bf NOT} use this script on real, production machines.  The configuration which is produced is deliberately
insecure, and is therefore intended just for quick evaluation in a virtual machine.  The script has been tested on a
``Basic Server'' installation of Scientific Linux 6.1, and might require adjustments on other systems.

At first, transfer all the RPM packages and the {\tt deska-demo.sh} shell script from the enclosed CD disk to the target
system.  Enter the directory with the packages and execute the shell script as root:

\begin{minted}{console}
# ./deska-demo.sh
\end{minted}

After the script finishes, the system is ready to be used.  Just login as user {\tt franta} with password {\tt franta}
and execute the \path{deska-cli} as usual to play around with a fully configured Deska system with a sample
configuration generator that prints all {\tt host}s in a database.

The rest of this chapter contains all steps required for a proper deployment of Deska into a production environment.

\section{Packages}

\todo{Honza: verify these steps. Should be done by somebody else than me, obviously.}

The easiest way of installing the Deska application suite is from the RPM packages delivered as a part of this project.
We have provided the following packages suitable for immediate installation on Red Hat Enterprise Linux~6 \cite{rhel}
(version 6.1 or newer, including derivatives like the Scientific Linux), which are available on the enclosed CD (see
\secref{sec:cd-structure}) and built for the x86\_64 architecture:

\begin{itemize}
    \item {\tt deska-client}, for the command-line application (CLI),
    \item {\tt deska-server}, for the Deska server and all supporting tools,
    \item {\tt deska-libs}, the shared libraries for both client and server,
    \item {\tt deska-python-libs}, containing a Python library for access to the Deska server,
    \item {\tt deska-devel}, for development files like the C++ headers,
    \item {\tt deska-debuginfo}, which contains proper object symbols and a copy of the source for reporting bugs
\end{itemize}

The Deska system also depends on a few external packages which are not commonly available in the default RHEL6
repositories.  These packages are available on the enclosed CD as well.

To install the selected packages, either set up a YUM repository, or simply use the {\tt yum localinstall} command.
Please keep in mind that installation has to be performed as user {\tt root}:

\begin{minted}{console}
# yum localinstall path/to/deska/packages/*.rpm
\end{minted}

The packages provided by the Deska team properly list their dependency information.  Unfortunately, some of the
dependencies are on software which is not packaged on RHEL6 by default.  All packages are available on the CD, though.

A typical user's workstation or a laptop does not have to contain the whole Deska system.  To install just the Deska
CLI, proceed in the following way, installing packages {\tt deska-client} and {\tt deska-libs}:

\begin{minted}{console}
# yum localinstall path/to/deska-{client,libs}-*.rpm
\end{minted}

After the installation is complete, please proceed with configuring the Deska server by following instructions
in~\ref{sec:build-server-setup}.

\section{Manual Building}

While the recommended installation method is via the RPM packages, as discussed in the previous chapter, some platforms
might require compilation directly from the source code.  The Deska system can be built on any reasonably modern Linux
system.  The following platforms are equally tested and supported:

\begin{itemize}
    \item Red Hat Enterprise Linux \cite{rhel} 5 (release 5.6 or newer, including derivatives like Scientific
        Linux~\cite{scientific-linux})
    \item Red Hat Enterprise Linux 6 (release 6.1 or newer, including derivatives like the Scientific Linux)
    \item Gentoo Linux~\cite{gentoo}
\end{itemize}

\subsection{Prerequisites}

All platforms require a C++ compiler with decent support for templates.  Please be advised that compilation of the CLI
involves instantiation of rather complicated template structures which might require unusual amounts of RAM.  We have
observed a real memory consumption of up to 1.5~GB of RAM for each compile job.

The build process requires the following packages as an absolute minimum on a RHEL5 machine:

\begin{itemize}
    \item CMake (2.6 or later)~\cite{cmake}
    \item Boost Date Time (1.41 or newer)~\cite{boost}
    \item Boost Program Options (1.41 or newer)
    \item Boost Python (1.41 or newer)
    \item Boost System (1.41 or newer)
    \item Boost Unit Test Framework (1.41 or newer)
    \item GNU Readline (5.1 or newer; in case of RHEL5, we also require the {\tt termcap} library version 2.0.8 or
        newer)~\cite{gnu-readline}
    \item The {\tt libebt} library (optional, shipped in the sources)~\cite{libebt}
    \item The {\tt json\_spirit} library (optional, shipped in the sources)~\cite{json-spirit}
    \item The non-standard Boost Process library (patched version which is required for the tests is shipped in the
        sources)~\cite{boost-process}
\end{itemize}

The set of libraries listed above is enough to build the CLI of the Deska system.  However, in order to be able to
deploy the Deska server, the following packages are required to be present on the target server system, in addition to
the former list:

\begin{itemize}
    \item PostgreSQL (9.0 or newer)~\cite{postgresql}
    \item Python (2.6, or Python 2.4 with the {\tt simplejson}~\cite{simplejson} module version 2.1.0)~\cite{python}
    \item PsycoPg2 (2.0.14 or newer)~\cite{psycopg}
    \item The {\tt pg-python} procedural language (1.0.0 or later; not to be confused with the PlPython
        package)~\cite{pg-python}
    \item GitPython (the 0.1 branch; tested with 0.1.7, later branches have changed the API in an incompatible
        way)~\cite{git-python}
\end{itemize}

Although not strictly compulsory for regular builds, the following packages are required for the unit tests.  Running
unit tests after each manual build is highly suggested:

\begin{itemize}
    \item PgTAP unit test framework for Postgres (0.25 or later)~\cite{pgtap}
\end{itemize}

Fully automated test runs might require the following packages to be present and configured:

\begin{itemize}
    \item CMake (2.8 or later provides proper Git integration for nightly builds~\cite{deska-dashboard})
    \item Git (1.6.6 or later)
\end{itemize}

Finally, in order to build the documentation from source, the following packages will have to be present:

\begin{itemize}
    \item The {\tt pdflatex} \LaTeX~PDF compiler
    \item The {\tt minted} \LaTeX~package~\cite{latex-minted}
    \item The {\tt pygments} Python package~\cite{pygments}
    \item The {\tt subfiles} \LaTeX~package (optional, shipped with Deska)~\cite{latex-subfiles}
\end{itemize}

\subsection{Getting the Sources}

Apart from the copy available on the enclosed CD (see \secref{sec:cd-structure}) and the stable releases available from
the project homepage~\cite{deska-project}, the sources are also available from an anonymous Git
repository~\cite{deska-git}:

\begin{minted}{console}
$ git clone git://repo.or.cz/deska.git
Cloning into deska...
remote: Counting objects: 23348, done.
remote: Compressing objects: 100% (9151/9151), done.
remote: Total 23348 (delta 14099), reused 23208 (delta 14001)
Receiving objects: 100% (23348/23348), 5.12 MiB | 183 KiB/s, done.
Resolving deltas: 100% (14099/14099), done.
\end{minted}

\subsection{Compilation}

After the sources are obtained and the source tree is ready, use the following command to perform the
build.\footnote{Some ssytems, most notably the RHEL5, suffer from a suboptimally built {\tt libreadline} which fails to
specify all of the required library dependencies.\cite{rhel5-readline-bug}.  The Deska build system includes special
tweaks for building on such systems through the {\tt -DEXTRA\_READLINE\_LIBRARY} option which shall point to a .so
library providing missing symbols.  Typical names are {\tt libtinfo.so.5} or {\tt libtermcap.so.2}.}  The build
process is controlled by standard CMake options (like the {\tt -DCMAKE\_INSTALL\_PREFIX} for target path) and a few
custom ones.  Be sure to pass on the {\tt -DRUN\_SQL\_TESTS=1} statement to include complex test cases which involve
interaction with a remote PostgreSQL database.  If you would like to build the documentation as well, use the {\tt
-DBUILD\_DOCS=1} option.

\begin{minted}{console}
$ cd deska/
$ mkdir _build
$ cd _build/
$ cmake -DCMAKE_INSTALL_PREFIX=/opt -DRUN_SQL_TESTS=1 -DBUILD_DOCS=1 ..
-- The C compiler identification is GNU
-- The CXX compiler identification is GNU
-- Check for working C compiler: /usr/bin/gcc
-- Check for working C compiler: /usr/bin/gcc -- works
-- Detecting C compiler ABI info
-- Detecting C compiler ABI info - done
-- Check for working CXX compiler: /usr/bin/c++
-- Check for working CXX compiler: /usr/bin/c++ -- works
-- Detecting CXX compiler ABI info
-- Detecting CXX compiler ABI info - done
-- Boost version: 1.46.1
-- Found the following Boost libraries:
--   system
--   date_time
--   unit_test_framework
--   python
--   program_options
-- Using libebt from /home/jkt/deska/src/3rd-party/libebt-1.3.0
-- Using json_spirit from /home/jkt/deska/src/3rd-party/json_spirit_4.04
-- Using boost::process from /home/jkt/deska/src/3rd-party/process
-- Found GNU readline: /usr/lib64/libreadline.so
-- Found PythonLibs: /usr/lib64/libpython2.7.so
-- Will run SQL database tests upon each `make test` run.
-- Will re-initialize the database before each test
-- Copying directory /home/jkt/deska/_build/tests/sql
-- Writing out test configuration
-- Configuring done
-- Generating done
-- Build files have been written to: /home/jkt/deska/_build
$ time make -j4
# ...
\end{minted}

You might want to change the {\tt -j4} option to a lower number if your build machine is memory-constrained, or increase
it to a higher number, should you use a host with higher number of CPUs and plenty of RAM.

\subsection{Test Invocation}

After the sources have been built, it is highly recommended to run the automated test suite.  No extra configuration is
required, apart from an available PostgreSQL server binary.  The test case will configure and start its own PostgreSQL
server instance, using the {\tt /dev/shm} {\tt shmfs} in-memory filesystem as the backing store, and will use just the
Unix socket for communication.  This test assumes that the PostgreSQL administration tools, like the {\tt initdb} and
the {\tt postgres} binary, are available to the current user.

\begin{minted}{console}
$ time ../run-standalone-tests.sh
# This step might take several minutes to complete.
100% tests passed, 0 tests failed out of 48
\end{minted}

\subsection{Installation}

When the tests have finished correctly, the client tools and various Python modules can be installed by running {\tt
make install}.  The server, however, requires more complex procedure to set up.

\section{Setup and Configuration}
\label{sec:build-server-setup}

After having installed the server utilities, either from packages or by running the {\tt make install} command, it is
time to deploy the database and configure the Deska server.

\subsection{Server Setup}

The very first step is to configure a PostgreSQL server.  It is recommended that it runs on the same host as the {\tt
deska-server} script for security reasons (see the \secref{sec:server-security-model} for details).  The server shall
accept connections over Unix sockets and use the {\tt ident} scheme for user authentication over these sockets.
Furthermore, each (physical) user who will have access to the Deska server will have to be present in the PostgreSQL
database, be the member of the {\tt deska\_user} role, and has a Unix account with properly configured SSH keys for
automated authentication to the server machine.

Here is an example configuration for the PostgreSQL's \path{pg_hba.conf} showing the required options for the {\tt
d\_fzu} database:

\begin{minted}{text}
local   d_fzu     all     ident
\end{minted}

The {\tt deska-server} itself does not run autonomously, but is instead started on-demand when a user launches the {\tt
deska-cli}.  The console will open an SSH connection to the server, start a wrapper scripts which sets the required
parameters, and then pass control to the actual {\tt deska-server} program.

A sample script is available from the CD as the {\tt run-deska.sh}.  Here is how a typical configuration looks like:

\label{sec:deska-server-wrapper}
\begin{minted}{bash}
#!/bin/bash

# The database name to use
export DESKA_DB=d_fzu

# Enable configuration generators. The tests/prepare-git-repo.sh script can be
# used to set up this directory tree, and sample scripts in the
# scripts/config-generators directory can be symlinked into the
# $DESKA_CFGGEN_SCRIPTS directory to activate those currently in use at the FZU.
export DESKA_CFGGEN_BACKEND=git
export DESKA_CFGGEN_SCRIPTS=/path/to/cfggen/scripts
export DESKA_CFGGEN_GIT_REPO=/path/to/cfggen/repository
export DESKA_CFGGEN_GIT_WC=/path/to/cfggen/changesets
export DESKA_CFGGEN_GIT_PRIMARY_CLONE=/path/to/cfggen/primary

# Now that we have set up our environment, launch the server.
deska-server
\end{minted}

From now one, we will assume that the script above is available on the Deska server as \path{deska-server-launcher}
somewhere in the user's {\tt \$PATH}, and that the name of the machine hosting the Deska server is
\path{deska.farm.particle.cz}.

Now we shall create the appropriate \path{deska.ini} file to be distributed to the regular Deska users.  Here is how
this file shall look like:

\begin{minted}{bash}
[DBConnection]
Server=ssh
Server=deska.farm.particle.cz
Server=/usr/local/bin/deska-server-launcher
\end{minted}

These files shall be put into each user's home directory.

\subsection{Database Deployment}
\label{sec:build-database-deployment}

After the PostgreSQL server is configured properly, it is time to deploy the database scheme.  A database scheme
describes the layout of the database and defines names of the Deska kinds andrelations between them (see
\secref{sec:chap-objects-and-relations} for an introduction and \secref{sec:admin-dbscheme} for a list of constraints
which this scheem has to fulfill in order to be usable in Deska).

The utilities shipped with Deska support installation to an {\em empty} PostgreSQL database.  In case of an upgrade or
an attempt to repair a broken database installation, please make sure to take a proper backup first (see
\secref{sec:cli-db-backup} for details), and completely {\em remove} the original database before proceeding further.

The installation process therefore starts with creating a database and the required user roles:

\begin{minted}{console}
$ psql -U postgres
\end{minted}
\begin{minted}{sql}
CREATE ROLE deska_admin;
CREATE ROLE deska_user;
CREATE USER yourUser;
GRANT deska_user TO yourUser;
CREATE DATABASE yourDb OWNER deska_admin;
\end{minted}

When the database has been created, install the PgPython procedural language in there:

\begin{minted}{console}
$ psql -U postgres -d yourDb -v ON_ERROR_STOP=1 -f path/to/deska/install/pgpython.sql
\end{minted}

Now you can proceed to the actual installation of the Deska database which is performed by the
\path{deska-deploy-database.sh} script.  There are three required arguments, the {\tt -U} ({\tt --user}) for specifying
a PostgreSQL user name, the {\tt -d} ({\tt --database}) for the database name, and finally the {\tt -t} (or {\tt
--target}) for the name of the directory in which to install additional server-side data (as the Deska server requires
some generated code tailored to the used database scheme, and these files have to live outside of a PostgreSQL database
directory hierarchy).  Due to the security model used by Postgres, where a Python-based procedural language module has
essentially full access to the database and the whole system, using the effective Postgres server UID, it is required to
perform the installation as the PostgreSQL superuser, which is typically the {\tt postgres}:

\begin{minted}{console}
$ DESKA_SCHEME=demo ./deska-deploy-database.sh -U postgres -d yourDb \
    -t /path/to/install/server/side/tools
\end{minted}

Please keep in mind that the directory used for the {\tt -t} option to the {\tt deska-deploy-database.sh} has to be
exclusively used by exactly {\em one} instance of the Deska database.  It is perfectly acceptable to run several Deska
databases on one machine and sharing the same PostgreSQL server instance, as long as they use a different path for the
auxiliary files.

\subsection{Configuration Generators}
\label{sec:build-cfg-generators}

In a real deployment, the configuration generators will have to be heavily tailored towards the site-specific scenarios.
As an example, the initial set of scripts which are in use in Prague is available from the
\path{scripts/config-generators/} directory in the Deska sources.  In the rest of this chapter, we will describe how to
configure the Git integration and deploy these scripts.  The following steps will deploy an isolated Git repository
which will be suitable for use by Deska, but the changes will {\em not} be pushed to any remote place (we do not know
the host names or repository setup of each site's configuration management systems, after all).

Start with running a sample script which creates the git repository and fills it with the initial content.  The
\path{tests/prepare-git-repo.sh} is available from the Deska source tree:

\begin{minted}{console}
$ mkdir /path/to/cfggen/
$ ./tests/prepare-git-repo.sh /path/to/cfggen/

...some output goes here...

Please use the following as the Deska server configuration:

export DESKA_CFGGEN_BACKEND=git
export DESKA_CFGGEN_SCRIPTS=/path/to/cfggen/scripts
export DESKA_CFGGEN_GIT_REPO=/path/to/cfggen/cfggen-repo
export DESKA_CFGGEN_GIT_PRIMARY_CLONE=/path/to/cfggen/cfggen-primary
export DESKA_CFGGEN_GIT_SECOND=/path/to/cfggen/second-wd
export DESKA_CFGGEN_GIT_WC=/path/to/cfggen/cfggen-wc

A sample script has been put into /path/to/cfggen/scripts/01-demo for you.
\end{minted}

The variable definitions which the scripts asks you to use shall be put into the shell wrapper for the Deska server
which we have discussed in section \ref{sec:deska-server-wrapper} earlier in this chapter.

\end{document}
