% vim: spelllang=en spell textwidth=120
\documentclass[deska]{subfiles}
\begin{document}

\chapter{The Console Application}
\label{sec:cli-usage}

\begin{abstract}
The following chapter serves as a users' guide to the CLI application.
\end{abstract}

\section{Introduction}

The Deska CLI is a console application serving as an exclusive tool for manipulating data in the DB. For running the
CLI, some parameters and configuration file must be set. For details refer section \ref{sec:cli-connection-setup}.

You can exit the CLI using command {\tt exit} or {\tt quit} or using shortcut "Ctrl + D".

\section{Prompt and context}

When the CLI is started, you will se the command prompt, where the commands could be entered. The prompt has its context.
The context is something like a path to some object where each object in the path is nested to the previous. 
It means, that all commands manipulating, or working with objects will now be related to this context. This behavior
is very similar to an ordinary unix prompt where under context we can imagine a current directory where we are. Actually the
context could represent a set of objects, because it can contain also filters, not only single object definitions.
We will talk about filters later. Context means, that all commands manipulating, or working with objects will now be related to
object or objects represented by this context. This behavior is very similar to an ordinary unix prompt where under context
we can imagine a current directory where we are. In unix prompt all commands like {\tt ls}, {\tt cd}, {\tt mv} and so on
are also related to the current directory.

You can go into the context of some object simply by typing its kind and name. For example:

\begin{minted}{text}
> host hpv2
host hpv2> interface eth0
host hpv2->interface eth0>
\end{minted}

Now we are in context of object interface "interface eth0" embedded in "host hpv2". Stepping from the context is done by
keyword {\tt end}.

There is also a possibility of shortened types of jumping into context of a nested object using {\tt ->} in the name of the
object.

\begin{minted}{text}
> interface hpv2->eth0
host hpv2->interface eth0> end
host hpv2> end
>
\end{minted}

The {\tt interface hpv2->eth0} is equivalent to {\tt host hpv2 interface eth0}. So also keyword {\tt end} will jump into context of
"host hpv2" and not to a "top-level" context. This shortened version could be used also in commands manipulating with
objects.

\subsection{Filters}

Now we can describe, how to use filters in the context. Filter could be entered by typing kind name followed by keyword
{\tt where} and then writing filter definition in braces. The filter definition syntax could be described as this recursive
structure:

\begin{minted}{text}
<filter>     = (<andFilter>) or (<orFilter>) or (<expression>)
<andFilter>  = list of <filter> separated by '&'
<orFilter>   = list of <filter> separated by '|'
<expression> = comparison of attribute name and some value
\end{minted}

Note that each part of the filter has to be in braces. In a filter for a specific kind you can use expressions containing
attributes directly from this kind, or kinds, that are contained in this one. There is also possibility to use expressions
containing attributes of any kind, that is in some relation with this one. With relations we mean standard Deska relations
like "embed into", "contains" and "containable" and "refers to". These attributes should be prefixed by name of their
kind with dot. For example {\tt interface.ipv4}. There are two sets of operators, that can be used in expressions. First
set is for standard attributes: {\tt <}, {\tt >}, {\tt <=}, {\tt >=}, {\tt !=} or {\tt <>} and {\tt =} or {\tt ==}. The second set is for identifiers
sets. There are only two operators: {\tt contains} and {\tt not\_contains}.

Filters are an exception in a concept of context. Normally we can not be in context of a specific interface without being
in context of its host. Filters are a little bit different as a set of interfaces should not have one shared parent. So
when being in top-level context, we can write filter on any kind. When being in context of some object, this filter
relates to this context and we can filter only on objects and kinds, that are embedded directly or recursively in this object.

Now we will show some examples of filters:

Filter on hardware from a specific vendor:
\begin{minted}{text}
> hardware where (vendor == HP)
filter on hardware> end
>
\end{minted}

This was filter using attribute directly from hardware. We can also ask for the same thing using attribute from vendor.
\begin{minted}{text}
> hardware where (vendor.name == HP)
filter on hardware> end
>
\end{minted}

Now we can write filter for hardware not purchased in 2010 serving as www server.

\begin{minted}{text}
> hardware where (((purchase < 2010/01/01) | (purchase > 2010/12/31)) & (host.service contains www))
filter on hardware> end
>
\end{minted}

There is also a special filter {\tt all} which is used for selecting all instances of a specific kind:

\begin{minted}{text}
> all host
filter on host> end
>
\end{minted}

The filter honors pre-existing context, so it can be for example used to list all interface which belongs to a specific
host:

\begin{minted}{text}
> host hpv2
host hpv2> all interface
host hpv2->filter on interface> end
host hpv2>
>
\end{minted}

Here is one useful command for operations with filters. Is is command {\tt context}. It will show you current context
with the filters. It is useful as in the prompt you can see only that you are in context of a filter, but you can not
see, which filter exactly. This command will show you exactly the filter. You can also use {\tt context objects}. This
will show you list of all objects matching current context. It can show you all objects, that can be affected by commands
operating with context.

\section{Showing content of the DB}

There are two ways to look at what is currently stored in the DB. First way is showing content of the whole DB. This
is done by command {\tt dump}.

Command {\tt dump} recursively shows all objects in the DB including their attributes and nested objects. No other
relations are displayed in this format. This command also accepts an optional parameter to redirect the command's output
to the specified file.

The sample dump can look like this:
\begin{minted}{text}
> dump
host hpv2
    no hardware
    note_host "Sample host"
    service [ftp, www]
    no template_host
    no virtual_hardware
    interface eth0
        ip4 192.168.2.34
        no ip6
        mac 2E:4A:AB:89:6C:7B
        no network
        no note
        no template_interface
    end
end
host golias2
    no hardware
    note_host "Another sample host"
    no service
    no template_host
    no virtual_hardware
end
service www
    no note
end
service ftp
    no note
end
service dns
    no note
end
\end{minted}

Another method is using command {\tt show}. When the user has not entered a context yet, this command shows only list of top-level objects without any
additional information. When in a context, it will display information about object or objects from the current context. The
information contains all attributes with resolved values from templates. When some attribute value was taken from
a template, information about source of this value is also printed. If some object is contained in one being currently
printed, its attributes and nested kinds will be printed under attribute referring to that object. Nested objects are
printed in a way similar to the output from the command {\tt dump}. Example of output from {\tt show} on a host containing some hardware
through attribute "hardware" with attribute "ram" inherited from template for hardware with name "hp357":

\begin{minted}{text}
> host hpv2
host hpv2> show
hardware hpv2
    cpu_ht 1
    cpu_num 4
    no hepcpec
    host hpv2
    no note_hardware
    purchase 2011-12-23
    ram 4096 -> hp357
    template_hardware hp357
    no vendor
    warranty 2013-12-23
note_host "Sample host"
no service [ftp, www]
no template_host
no virtual_hardware
interface eth0
    ip4 192.168.2.34
    no ip6
    mac 2E:4A:AB:89:6C:7B
    no network
    no note
    no template_interface
end
\end{minted}

Command {\tt show} could be used also for printing information about objects outside of the current context. Like {\tt show host hpv2} or
{\tt show host hpv2 interface eth0}. This form will print the requested information without changing the current context.
The parameter can refer both to the top-level objects and to those in the current context. It means that in context of host "hpv2" you can type {\tt show interface eth0}
for showing information about interface "eth0" nested in host "hpv2".

\section{Work with changesets}

Till now, we could perform only read-only commands. For modifying content of the DB, you have to be connected to some changeset.
Changeset is some kind of session, where you can modify objects int he DB where the DB does not change. When you are done,
you can commit this changeset and your changes will be written to the DB and new revision will be created.
Start of new changeset is done using keyword {\tt start}. Then if you are not done with your work and you would like to
go home from the work, you can detach from the changeset. You should use command {\tt detach} for this purpose. Using
this, you will not loose your work in progress. CLI will ask you for a detach message commenting what is going on in
your changeset. You can optionally type the message as a command parameter. For example {\tt detach Adding new machine}.
When you would like to connect to your changeset again, you should use command {\tt resume}. This command will show
you a list of changesets with some additional information and you can choose one to connect to. You can also find your
changeset in this list if your CLI dies, or if you exit it without detaching. Sou your work will not be lost. Each changeset has
its ID. You can see it in the list and you can also obtain it using command {\tt status}. This command will tel you
if you are in some changeset and if you are, it will tell you in which. You can then use this ID directly as parameter
of the command {\tt resume}. This ID does not change, so if you know it, you can skip the choosing using list of
all changesets. If you want to abort current changeset and discard all work you have don in it, type simply {\tt abort}.
When you are done, you can commit your work using command {\tt commit}. The CLI will ask you for a commit message of as
in case of detach, you can type it directly as a command parameter.

\section{Modifying objects and their attributes}

In this section we describe how to create, rename and delete object and also how to set or remove a value of their
attributes.

\subsection{Objects creating, deleting and renaming}

There are three ways on how to create an object. First one is an explicit one using keyword {\tt create}. For example {\tt create
host hpv2} will create object "host hpv2". You can also directly create nested object (including its parent, if it
doesn't exist yet) using a single {\tt create host hpv2 interface eth0} call. Note that the keyword {\tt create} does not change
the current context, but acts as a standalone action -- an object is created, but the user's context remains the same.
Second way how to create an object is through an attempt to step into the context on a non-existing object.
When trying to do that, CLI will ask you if you want to create this object. Obviously, this second will change the
context, for example:

\begin{minted}{text}
> host hpv2
Object(s) host hpv2 do(es) not exist. Create? y
host hpv2> end
> host hpv2
host hpv2> end
>
\end{minted}

When going into the context of a nested object where the parent does not exist either, the CLI will ask only for creating the
parent --- as the parent did not exist previously, the user surely intends to create the second object as well.

Finally, the third way of creating an object is through the {\tt new} keyword. This command works only on the nested
objects, and lets the server assign a name to the newly created object automatically. This feature is useful when creating
new object where the custom name does not make much sense. You specify only the kind name after this keyword new:

\begin{minted}{text}
> hardware ibm458
hardware ibm458> new failure
hardware ibm458->failure failure_1> end
hardware ibm458> new failure
hardware ibm458->failure failure_2> end
hardware ibm458> end
>
\end{minted}

Please keep in mind that the user cannot make any assumptions about the newly assigned name.  The current server
implementation will use a monotonous sequence as the bases for the name, but extensions are free to change this behavior
to e.g. include timestamps in the created names, etc.

In this example you can see that the keyword {\tt new} changes the context to enter the just-created object.  This might
come handy when working with ``append-only'' objects like in an event log.

Deleting objects is similar to creating. It is done using keyword {\tt delete}. You can use the same parameters as in
create, like {\tt delete host hpv2}. When you are trying to delete a nested object directly, use {\tt delete host hpv2 interface
eth0}. This will delete only the interface, not the whole host. When deleting parent of some nested objects, CLI will
also delete all these nested objects (after confirmation).

Renaming of objects is also very simple, through the {\tt rename} keyword. The format of
parameter for {\tt rename} is similar as in create, but you have to specify the new name of the object. For example
{\tt rename host hpv2 hpv3} will rename host "hpv2" to "hpv3". When you are renaming an object which is connected using relation
"contains" or "containable" to another objects, the CLI will ask you if you want to rename these connected objects
as well as they have the same name. If you do not rename them, the connection between them will be lost.

For renaming, creating and deleting objects you can also use filters. Here is the example of deleting all interfaces from
a host with some specific characteristic:

\begin{minted}{text}
> host hpv2
host hpv2> delete interface where (note == "to delete")
Are you sure you want to delete objects(s) interface from filter? y
host hpv2> end
>
\end{minted}

Shortened notations like {\tt interface hpv2->eth0} could be also used for deleting or renaming. Note, that when renaming
object using this shortened notation, parameter identifying a new name should not contain parents name. For example:

\begin{minted}{text}
> rename interface hpv2->eth0 eth1
>
\end{minted}

\subsection{Modifying attributes}

The simplest way of modifying an attribute of a specific object is by typing the attribute name and then the desired value
when being in context of the object:

\begin{minted}{text}
> host hpv2 interface eth0
host hpv2->interface eth0> ip4 192.168.25.65
host hpv2>
\end{minted}

You can also remove the value of the attribute using keyword {\tt no}. For example {\tt no ip4}. There are also two special
keywords for operations with identifier sets. It is keyword {\tt add} for adding an identifier to a set and keyword
{\tt remove} for removing an identifier from a set --- e.g. {\tt add service www} for adding identifier "www" to a set
with name service. You also set the set directly as a value of standard attribute using the notion {\tt service [ftp, www,
dns]}, which simply overwrites the current value with the specified data.

Here is the list of all supported types and syntax for the values:

\begin{itemize}
    \item{String} - Characters in quotes or apostrophes when containing spaces, or characters without any spaces
    \item{Identifier} - List of alphanumerical letters and {\tt -} or {\tt \_}, not ending with {\tt -}, separated by
        {\tt ->} (when containing more than two items)
    \item{Identifier set} - List of Identifiers in square brackets ({\tt []} separated by comma ({\tt ,})
    \item{IP Address v4} - The dot-decimal notation~\cite{ipv4-dot-decimal}
    \item{IP Address v6} - Various supported formats~\cite{rfc5952}
    \item{MAC Address} - IEEE~802 format with colon as separators and lowercase letters
    \item{Integer} - Integer
    \item{Double} - Double
    \item{Boolean} - strings "true" or "false"
    \item{Date} - Date in the {\tt YYYY-MM-DD} format
    \item{Timestamp} - Timestamp in the {\tt YYYY-MM-DD HH:MM:SS} format.  Seconds optionally accept decimal fraction.
\end{itemize}

You can also set more attributes using one line notation. For example:
\begin{minted}{text}
host hpv2->interface eth0> ip4 192.168.25.65 ip6 123f:1234::1 
host hpv2->interface eth0>
\end{minted}

Moreover it is possible to step into context of some object and set its attributes using one line notation. This will
change the context only temporarily for the attributes setting. The context remains the same as before after processing
the line. Here is an example:
\begin{minted}{text}
host hpv2> interface eth0 ip4 192.168.25.65 ip6 123f:1234::1 
host hpv2>
\end{minted}

Also this notation does the work:
\begin{minted}{text}
>host hpv2 interface eth0 ip4 192.168.25.65 ip6 123f:1234::1 
>
\end{minted}

It is also possible to set attributes of more objects using one line. The objects must be nested.
\begin{minted}{text}
>host hpv2 note "Some note" interface eth0 ip4 192.168.25.65 ip6 123f:1234::1 
>
\end{minted}

You can of course combine attribute settings with their removals of operations with sets as in this example:
\begin{minted}{text}
>host hpv2 no note add service www interface eth0 ip4 192.168.25.65 
>
\end{minted}
This example removes note from host "hpv2", adds identifier "www" to set named "service" and sets IP address of nested
interface "eth0". When any of the objects does not exist, CLI will ask you, if you want to create it.

Moreover it is possible to access attributes of objects, that are contained in one in the context. So if some host
contains hardware, you can set or remove attributes of the hardware from context of the host. And is the hardware
does not exist and you are trying to set its attributes, CLI will ask you, if you want to create this hardware and
connect it to the host. When confirmed, CLI will create hardware of the same name as out host and set its attribute.

\section{Batched operations}

It is also possible to load the commands modifying objects from a file. You can prepare a file with commands you would
like to perform. One command per line. Then you can load them using command {\tt batch}, where you give the filename
of the file to this command as a parameter. CLI will load and execute all these commands. Note, that you have to be
connected to some changeset to perform modifications. You can use also comments in the file. The comments are lines
beginning with {\tt \#}.

\section{History log}

Deska provides similar log of changes to source versioning systems. You can print a list of all revisions using
command {\tt log}. In the list you can see also author af a revision, timestamp, when it was committed and a commit
message. If you do not want to see all changes, but you are interested in for example changes from a specific period
or in revisions from a specific author, you can use filter. Syntax of this filter is similar to syntax of filters for
objects. The filter definition syntax could be described as this recursive structure:

\begin{minted}{text}
<filter>     = (<andFilter>) or (<orFilter>) or (<expression>)
<andFilter>  = list of <filter> separated by '&'
<orFilter>   = list of <filter> separated by '|'
<expression> = comparison of some revision metadata and some value
\end{minted}

Note that each part of the filter has to be in braces. In an expression you can use the following operators: {\tt <}, {\tt >},
{\tt <=}, {\tt >=}, {\tt !=} or {\tt <>} and {\tt =} or {\tt ==}.

Here is the list of attributes, that can be used in expressions:

\begin{itemize}
    \item{revision} - Revision ID (r1, r45, r21, ...).
    \item{author} - Author name (alphanumerical letters, {\tt -} or {\tt \_}).
    \item{message} - Commit message (characters in quotes or apostrophes when containing spaces, or characters without any spaces).
    \item{timestamp} - Timestamp in the {\tt YYYY-MM-DD HH:MM:SS} format.  Seconds optionally accept decimal fraction.
    \item{<kind name>} - The attribute name is some kind name and value could be some object name.
\end{itemize}

Here is an example of filter showing revisions, where object "host hpv2" was modified by user "tom":
\begin{minted}{text}
> log ((host == hpv2) & (author == tom))
Revisions:

Revision | Author   | Time stamp                 
         | Commit message
===================================
r1       | postgres | 2011-Jul-26 14:34:42
         | Initial revision
--------------------------------------------------------------------------------
r2       | tom      | 2011-Sep-12 12:17:26
         | Host hpv2 created
--------------------------------------------------------------------------------
r3       | tom      | 2011-Dec-27 23:48:26
         | Host hpv2 modified
> 
\end{minted}

\section{Diff}

When looking on history log, or when working in a changeset, you could be interested in changes made. There is a command
{\tt diff} for these purposes. When used without any parameter in a changeset, you will get diff between the changeset and
its parent revisions. It means changes made in this changeset. You can use two revisions as parameters to get diff between
these two revisions. The format of the diff is following:

Object "host hpv2" created:

\begin{minted}{diff}
+ host hpv2
+ end
\end{minted}

Object "host hpv2" renamed to "hpv3":
\begin{minted}{diff}
- host hpv2
+ host hpv3
end
\end{minted}

Object "host hpv2" deleted:
\begin{minted}{diff}
- host hpv2
- end
\end{minted}

Attribute of object "host hpv2" set:
\begin{minted}{diff}
host hpv2
+     note_host "Some note"
end
\end{minted}

Attribute of object "host hpv2" removed:
\begin{minted}{diff}
host hpv2
-     note_host "Some note"
end
\end{minted}

Attribute of object "host hpv2" changed:
\begin{minted}{diff}
host hpv2
-     note_host "Some old note"
+     note_host "Some new note"
end
\end{minted}

All other changes are only combinations of possibilities above. For example creating and setting attributes of an
object could look like this:
\begin{minted}{diff}
+ host hpv2
+     note_host "Some new note"
+ end
\end{minted}

Or renaming and changing attributes of some object could look like this:
\begin{minted}{diff}
- host hpv2
+ host hpv3
-     note_host "Some old note"
+     note_host "Some new note"
end
\end{minted}

There will never be shown any attributes when some object was deleted.

The {\tt diff} has also feature that generates file for command {\tt batch}. Using this feature you can for example backup
your work in a changeset to a file. To use this feature, only use a filename where to generate the commands as a first
parameter of the command {\tt diff}.

\subsection{Configuration diff}

There is also one more type of diff. It is command {\tt configdiff}, that shows differences in generated configuration files
caused by changes in out changeset. You can also use parameter {\tt regenerate} to force the generators to regenerate the
configuration files to see fresh diff.

\section{Commit conflicts}

When working in some changeset, another user could be faster and start and finish his work before we committed our work.
It means, that parent of our changeset is obsolete. As a history in the Deska is linear, it is obvious, that the commit
will not work in this situation. If you try to commit, CLI will inform you, that rebase is needed. There is command
{\tt rebase} for this purposes. It generates a file summarizing changes in both our changeset and changes made while
we were working. Than it opens an editor where we perform some kind of merge of our changes and changes made by other users.
When the rebase is done. We will get the changeset with new parent and changes determined by our merge. Now we can try to
commit our work again and now, we hope, successfully.

The format of the file is very intuitive and is in format of commands we are familiar from the CLI or command {\tt batch}.
We can not revert changes made by someone else, so information about changes made by someone else are in form of comments
(lines beginning with {\tt \#}).

\section{Non interactive mode}

When you are working in the CLI, it will quite often bother you with questions if you really want to delete some object,
if you want to create it and so on. These questions could be useful to prevent typos, but they are slowing the work down.
You can turn these questions off by command {\tt noninteractive off} or turn again by {\tt noninteractive on}. Using
only {\tt noninteractive} you will get information if this mode is turned on or off. In non-interactive mode all
questions will be confirmed with exception of renaming of contained and containable objects. This is the only case in
which these objects will not be renamed as it will be impossible to rename only one object in this mode. If you want to
rename them in the non-interactive mode all, you have to do it manually.

Also commands asking for something could not be run in non-interactive mode. It means you can not run {\tt rebase}, or use
{\tt commit}, {\tt detach} or {\tt resume} without parameter specifying commit/detach message of changeset ID.

Also all commands loaded from file using command {\tt batch} are running in non-interactive mode.

\section{Database backup and restore}
\label{sec:cli-db-backup}

It may be useful to perform the backup of the whole DB. You can backup only persistent revisions. For backup of
changeset, it is possible to use {\tt diff <filename>} for each changeset you would like to backup. For backup of the
revisions including their changesets and information about author and commit message, you can use command {\tt backup}
with parameter specifying file, where the backup will be created. This file is not intended to be modified manually,
but if you find some reason to do so, you can find the format of the file in the programmers documentation.

For restoring the DB from a backup, there is a command {\tt restore}. As a parameter it takes a filename of a file with
the backup. There is a special role called {\tt deska\_admin} for the restore. You can not perform restore without this
role. See DB part of the documentation for more information. The whole DB has to be empty before performing the restore.

\section{CLI Scripting}

You can prepare your own files with commands, that could be executed as batches from the CLI. In opposite to command
{\tt batch}, where the set of commands was restricted only to commands modifying objects, when preparing this file, you
can use also commands like {\tt start}, {\tt commit} and so on. The file can be loaded to the CLI using command
{\tt execute}. Execute will take each line from the file and executes it in the same way as you are writing it in the
console manually. The only difference is, that these commands are run in the non-interactive mode. There is also a
possibility of using comment, that are lines beginning with {\tt \#}.

\section{Integrated help}

The last command in the CLI is command {\tt help}. It will show you a short description of usage for each CLI command.
When used without parameter, you will get information about usage for each command in the CLI. You can use command name
to obtain information about only one command. You can also use some kind name as a parameter to obtain information about
attributes and nested kinds of the kind. The last case is parameter {\tt kinds}, which will show list of defined kinds.

\section{Tab completion}

There is a tab completion feature in the CLI as you know it from the Linux Bash. It could be used for completing of
keywords as well as object names. Press <TAB> key to perform completion. The double press will show the list of all
possibilities.

\todo {Is there needed some more detailed description what and where is completed?}

\section{Commands history}

The CLI stores history of all commands typed in the console. You can access this history using arrows up and down. Fro
searching in this history use shortcut "CTRL + R". The file where the history is saved is a part of a configuration
file.

\section{CLI Configuration}

The CLI has several values, that can or must be defined in order to get the CLI working. In the table \ref{tab:configfile}
you can see list of parameters from the configuration file. The file name id {\tt deska.ini} and should be placed in
the directory from where we are executing the CLI.

The file has standard configuration file format. It means, that section names are in {\tt []} on separate lines and
value assignments are in format {\tt <parameter name> = <parameter value>}. For example:
\begin{minted}{text}
[DBConnection]
Server=./src/deska/server/app/deska-server

[CLI]
HistoryFilename=.deska_cli_history
HistoryLimit=64
LineWidth=80
\end{minted}

\label{sec:cli-connection-setup}
There is one crucial mandatory parameter in the configuration file. It is parameter {\tt Server} in section
{\tt DBConnection}. This parameters specifies path to an executable to Deska server for connection to the DB including
arguments. If the executable takes some arguments, they should be one per line as in this example:
\begin{minted}{text}
[DBConnection]
Server=./deska-server
Server=arg1
Server=arg2
\end{minted}

\begin{longtable}{ p{2.2cm} | l | p{1.5cm} | c | p{3.5cm} | p{2.5cm} }
    \caption{Parameters in {\tt deska.ini}} \\
    Parameter section & Parameter name & Parameter type & Mandatory & Description & Default value \\
    \hline
    \endhead
\label{tab:configfile}
    DBConnection & Server & vector of strings & yes & path to executable for connection to Deska server including arguments & no \\
    CLI & HistoryFilename & string & yes & path to file where the history will be stored & {\tt \~/.local/ share/deska/ .deska\_cli\_history} \\
    CLI & HistoryLimit & unsigned integer & no & number of lines stored in history & 64 \\
    CLI & LineWidth & unsigned integer & no & maximum width of line in the console & no \\
    CLI & NonInteractive & boolean & no & Force non-interactive mode & false \\
    \hline
\end{longtable}

\section{Command line parameters}

In the table \ref{tab:cmdlineparams} you can see overview of parameters, that can be used in the command line.

\begin{longtable}{ l | l | l | p{7cm} }
    \caption{Command line parameters} \\
    Parameter name & Short name & Parameter type & Description \\
    \hline
    \endhead
\label{tab:cmdlineparams}
    non-interactive & n & no parameter & Force non-interactive mode \\
    help & h & no parameter & Shows help message with list of all possible parameters \\
    version & v & no parameter & Shows information about application version \\
    dump & d & optionally filename & Performs dump of the DB (as command {\tt dump}) \\
    backup & b & filename & Performs backup of the DB to a file (as command {\tt backup}) \\
    restore & r & filename & Restores the DB from a file (as command {\tt restore}) \\
    execute & e & filename & Executes commands from a file (as command {\tt execute}) \\
    \hline
\end{longtable}


\end{document}
