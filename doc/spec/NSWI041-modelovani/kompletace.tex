\documentclass[12pt]{article}
\usepackage[utf8]{inputenc}
\usepackage[english,czech]{babel}
\usepackage{a4wide}
\usepackage{graphicx}
\author{Jan Kundrát}
\title{Deska: Documentation Overview}



\begin{document}

{\Huge \textbf{Deska: Documentation Overview}}

\vspace{0.2in}

{\large Prepared for the Institute of Physics of the AS CR}

\vspace{0.2in}

{\large Version 1.0}

\vspace{0.2in}

{\large Date 2009-Dec-18}

\section{Overview}

The purpose of this document is to provide an easy-to-use overview of the pieces of documentation the team has produced during the
Fall 2009 for the NSWI049 lectures.  It consists of various documents which were already reviewed during the lectures.  The main
added value of such a compilation is an overview about what we achieved and what is the purpose of various documents we produced.

{\bf For an executive summary of the Deska project, please kindly proceed to the ``Specification'' document at the end of this
compilation.}

\section{Vision}

In the {\em Vision} document, the team has worked with the Institute in order to understand what the problem to solve is and to
see and comprehend its background.  This phase involved extensive talks and negotiations with the Institute, including on-site
visits and many presentations and talks, as well as travelling to other HEP institutions.  Representatives met staffers from
CC-IN2P3 from Lyon, France and INFN, Italy and talked about their approach to asset management and monitoring.

\section{Software Requirements Specification}

For the {\em SRS}, the team's attention has shifted to more concrete understanding of FZU's needs.  Part of the work was to have a
look at other software products in the problem domain in order to evaluate their scope and features.

\section{Use Cases}

The {\em Use Cases} document collects formalized observations gathered in previous phases of the project.  The intended use cases
were divided into several groups and an initial skeleton about the information flow was developed.

\section{Business Model}

The {\em Business Model}, or {\em Business Process} document was created to expand the use cases defined in the SRS document.  The
individual use cases were expanded and more detailed information flow was specified.

\section{Analytical Class Model}

The {\em Analytical Class Model} document consists of a diagram of the database structure the team expects to implement.  Its
purpose is to accurately model relations of the individual objects managed by the system.  Additionally, the basic ideas about the
architecture of additional components were depicted in a similar way.

\section{Function Model}

The {\em Function Model} tries to demonstrate the state flow of the system.  The team has included diagrams demonstrating both the
formalized flows and an informal drawing about the ``shape'' of the system as well.

\section{Glossary}

The {\em Glossary} provides a list of acronyms and abbreviations common in the HEP/HPC/Grid environment that might not be known
for the intended audience of this document.

\section{Specification}

While outside the scope of the NSWI041 and quite similar to the Vision, the {\em Specification} document is intended as an
executive overview about what the team is about to implement.  The main purpose of this document is to explain what the project
intends to solve, and do it at a rather limited length.


\end{document}
