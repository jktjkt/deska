\documentclass[12pt]{article}
\usepackage[utf8]{inputenc}
\usepackage[english]{babel}
\usepackage{a4wide}
\usepackage{moreverb}
\author{Jan Kundrát}
\title{Minutes from the GUI demonstration at FZU}

\begin{document}
\maketitle

\section{Introduction}

A meetings was held on May 4th 2010 at FZU.  The purpose was to present the
current ideas about how to design the CLI portion of the deska-db and to gather
ideas of real users-to-be.  The meeting was organized by Jan Kundrát of the
Deska team with three participants from the FZU: Jan Švec, Tomáš Kouba and Čeněk
Zach.

\section{The Concept of the CLI}

All attendees stated that they indeed welcome the overal design concepts of the
CLI interface.  It was stressed out that the CLI is generally able to provide
cleaner and more user-friendly experience to the end user.  A comparison to the
already existing web-based tool being used at the Institute was quickly
undergone with the result that the CLI is expected to be much more usable.

\section{Nesting and Scoping}

The general way to use the deska CLI, as demonstrated by the presentation
performed by Kerpl and Hubík the other day, was found to be acceptable and
probably the best one.  Most of the time was spent evaluating the two concepts,
the {\em nesting} one and the {\em referring} one.

These concepts refer to different ways in which one can express the concept of
several depths of nesting objects into each other.  The main motivation and a
perfect example at the same time is a physical location of a particular machine.
A HW box is usually located in a rack, which is itself placed in a room, which
usually belongs to a certain building.  The problem is that this approach gets
into trouble (consult the RackMonkey tool) when dealing with different HW than
rack-mount pizza boxes.  Almost all of the new and energy efficient HW are
nowadays being built in a smaller form factor, be it blade servers, the twin
cases as pioneered by SGI or at least sharing of power supplies as seen in IBM's
xServer iDataPlex solution.  The Deska DB should anticipate future extensions
and evolution in the industrial design, and therefore it has to support
arbitrary nesting of components.

\subsection{Nesting}

One way to express such a nesting is to emulate it closely in the database dump.
Such an approach is widely used in scripting languages like Python, where
indentation determines the scope of the code and variables.  Similar approach
could be then used with Deska, as indicated in the following code snippet:

\begin{listing}{1}
rack serverovna
    rack L01
        hw golias110
            bay 9
            model dl140
        end

        rack salix01_02
            inherit sgi-twin
            bay 10

            hw salix01
                bay 0
                model xe310
                interface eth0 switch ... port ...
                disk "2x 300GB SAS"
            end

            hw salix02
                bay 1
                model xe310
            end
        end

        hw blabla
            ...
        end
    end
end
\end{listing}

The main motivation for indication of nesting in such a way was to avoid the
need for explicit naming of ``anonymous containers'' like the chassis for a twin
server (see line 8 above).  However, it turned out that one cannot really use
unnamed objects in an effective way, and that trying to mask that fact would
only need to confusion.  It was also observed that the names of such objects
should be kept unique in the global scope.

\subsection{Referring}

On the other hand, {\em referring} simply creates a set of named objects in a
flat, global namespace and adds relations among them in an explicit way, as
demonstrated in the following example:

\begin{listing}{1}
rack serverovna
    model room
end

rack L01
    model APC-rack
    in serverovna
end

rack saltix05_06
    model sgi-twin
    in L05 bay 3
end

...

host saltix06
    serial X0011212
    rack saltix05_06 bay 1

    interface eth0 mac ... net wn-nat ip ... switch swL051 port 46

    role wn
end
\end{listing}

Please note the difference in using the explicit {\tt in} statements on lines 7
and 12 and via the {\tt rack \ldots bay \ldots} statement on line 19.

\subsection{Decision about the Format}

All attendees came to an agreement that the decision cannot be made on a purely
objective bases, as they recognize benefits associated with each of the formats.
Therefore, a decision had to be made mostly on personal subjective choice while
keeping a professional background and familiarity with similar tools (like the
Cisco IOS) in mind.

The outcome of the meeting was to go with the {\em referring} approach.  All
attendees unanimously voted for such a way, despite the fact that $\frac{3}{4}$
of the Deska developers voted for the {\em nesting} way.  An interesting
observation is that all people who had certain familiarity with managing a data
center voted for {\em referring}, while those of the purely academical
background voted for {\em nesting}.

\section{Network Interfaces}

A similar decision had to be made with regard to network interfaces and their
{\em bonding} together.  Again, an interface could have multiple {\em child}
interfaces associated with, and one has to agree about which way to express such
a situation to choose.  The following three concepts were mentioned:

\subsection{Nesting}
\begin{listing}{1}
host xxx
    interface bond0
        physical eth0 mac ...
        physical eth1 mac ...
    end
end
\end{listing}

\subsection{Enumeration}
\begin{listing}{1}
host xxx
    interface bond0 ip ... physical eth0,eth1
    interface eth0 ...
    interface eth1 ...
end
\end{listing}

\subsection{Referring}
\begin{listing}{1}
host xxx
    interface bond0 ip ...
    interface eth0 in bond0
    interface eth1 in bond0
end
\end{listing}

\subsection{Decision}

For the sake of consistency with the rest of the system, the attendees ruled out
the second option, the {\em enumeration} altogether.  Preferences about {\em
nesting} and {\em referring} were similar to the physical nesting of HW
components with the eventual result to go with {\em referring} again, especially
due to the way the Cisco IOS works when using Etherchannel.

\section{Closing Remarks}

The meeting was well-received by all parties overal.  It was suggested to
consider adding a ``tree-view'' dump format to the DB which would provide a
structured view of its structure (essentially similar to the {\em nesting}
format).  However, such a dump format should be {\bf complementary} to the
regular dumps, and the database won't support restoring from such formats.  The
purpose of such a dump, or rather an {\em extra view}, is to assist the users in
visualizing the data center, not to allow two distinct ways to use the
application.  It was also noted that such a view format could be implemented
using for example the Python bindings for the DB and not via the main CLI tool.

The employees of the Institute who were present at the meeting expressed their
content with the direction the Deska project is heading to.  They very much
welcomed that the Deska team was willing to revisit the decision to go with
nesting when the users expressed their concern with such an interface.

\end{document}
