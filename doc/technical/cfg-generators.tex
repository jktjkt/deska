% vim: spelllang=en spell textwidth=120
\documentclass[deska]{subfiles}
\begin{document}

\chapter{The Configuration Generators}
\label{sec:config-generators}

\begin{abstract}
FIXME: describe how the config generators work
\end{abstract}

FIXME\ldots

\subsection{Technical notes for the showConfigDiff DBAPI method}

The DBAPI provides just a single method for running the configuration generators on the server side.  The aim of that
function is to handle everything ``behind the scenes'', from setting up the configuration output directory and
triggering the transformations to diffing the output.  The crucial idea here is that the CLI should not contain any
domain-specific knowledge of how the whole process works, but instead shall present the user with just a single ``show
me the changes'' functionality.

There are many methods for correctly transferring the diff information and conveying that to the user.  Current version
of the DBAPI spec deals with this complexity by essentially ignoring the complications and supporting only the most
rudimentary form of the changes, the {\tt git diff} output.

Using this approach, the actual data being transferred are simply dumped to the user's screen.  Everything else is
handled by the SCM integration layer of the Deska server.

\subsubsection{Pending Changesets}

The basic unit to operate on when dealing with the configuration generating scripts is a pending changeset.  Only a
changeset can possibly trigger output modifications, and only when transformed into a persistent revision.  Similarly,
the actual configuration data are needed either at the time of a commit (in order to be pushed into a SCM repository of
the output configuration files), or when the user wants to review them.  These two moments are identical for us -- they
identify a point in time where the output has to be ready and reflect the present state of the DB.

The output of the configuration generators is a function of the database state.  When the state of the database does not
change, neither should the configuration generators\footnote{There are also the usual corner cases, though -- suppose a
DNS zone file.  The zone file typically contains a textual information about the ``age'' of a zone, which is typically
done by embedding a transformed timestamp inside.  Clearly, this information depends not only on the database state, but
also on some external information, like the current time or the ``generation'' of a particular configuration file.  For
these cases, the Deska DBAPI adds a ``force generators re-run'' option which will make sure that a stale copy of the
configuration output will not get used.  In addition to that, it's possible to ship a plugin which will use a policy to
decide whether to re-use a previous output or not.}.  The Deska DB server can therefore cache the outputs on a changeset
level -- ie. maintain a configuration output separately for each changeset.

Deciding whether to regenerate or reuse a previous version is then just a matter of maintaining a simple boolean flag
for the freshness of the output.  Creating a new changeset or rebasing to a new revision set it to false, while a
successful generator run sets it to true.

\subsubsection{SCM Branching}

In most situations, the configuration output will likely be stored in a SCM system.  Therefore, it makes sense to
integrate the output management logic to the SCM-specific operations.  The reference implementation of the Deska server
uses {\tt git}'s {\tt git-new-workdir} and branches quite heavily.

\subsubsection{How to do that in Git}

\begin{itemize}
    \item {\tt git-new-workdir} for setting up a WC the first time
    \item {\tt git.cmd.Git(path).add('-A')} for creating an index
    \item {\tt git.cmd.Git(path).diff('-{-}color', '-{-}staged')} for obtaining a diff
    \item {\tt git reset -{-}hard HEAD \&\& git clean -d -f -f} for throwing out all the changes
\end{itemize}

\end{document}
