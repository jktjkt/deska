\documentclass[12pt]{article}
\usepackage[utf8]{inputenc}
\usepackage[english]{babel}
\usepackage{a4wide}
\author{Tomáš Hubík}
\title{Deska SRS}



\begin{document}

{\Huge \textbf{Deska}}

\vspace{0.2in}

{\large Tool for Central Administration of a Grid Site}

\vspace{0.5in}

{\large System Requirements Specification}

\vspace{0.2in}

{\large Prepared for the Institute of Physics of the AS CR}

\vspace{0.2in}

{\large Version 1.1}

\vspace{0.2in}

{\large Date 2009-Nov-11}

\vspace{0.5in}

\subsection*{Revision History}

\begin{table}[!h]
	\begin{tabular}{l l l l}
		\textbf{Date} & \textbf{Version} & \textbf{Author} & \textbf{Description} \\
		2009-Nov-02 & V1.0 & Tomáš Hubík & Document structure \\
		2009-Nov-11 & V1.1 & Jan Kundrát & Use cases and requirements specified \\
	\end{tabular}
	\label{tab:RevisionHistory}
\end{table}


\subsection*{Document Review}

\begin{table}[!h]
	\begin{tabular}{l l l l}
		\textbf{Date} & \textbf{Version} & \textbf{Reviewer Name} & \textbf{Contact Info.} \\
	\end{tabular}
	\label{tab:DocumentReview}
\end{table}


\subsection*{Document Approval}

\begin{table}[!h]
	\begin{tabular}{l l l l}
		\textbf{Date} & \textbf{Version} & \textbf{Reviewer Name} & \textbf{Contact Info.} \\
	\end{tabular}
	\label{tab:DocumentApproval}
\end{table}


\newpage

\tableofcontents

\newpage


\section{Introduction}

\subsection{Purpose}
The purpose of this document is to define the specific requirements for the Tool for Central Administration of a Grid Site 
(henceforth, referred to as ``the System'') and to detail the specifications for the features, capabilities, critical attributes 
and major characteristics of the proposed system. It is intended to be read by management and administrators of Institute of 
Physics of the AS CR for the purpose of evaluating the benefits and feasibility of the proposed application as well as to provide 
a basis for the estimation of the time and effort necessary to construct, test, deploy and maintain it. This document does not 
describe how, when or where any of these activities will be performed or who will do them.

\subsection{Scope}
Main purpose of the Deska project is to provide an effective aid for system
administrators in their daily tasks related to running a production grid
cluster.  The main aim is to reduce data duplication and aggregate certain
metrics to a central place for monitoring.  For more information and a detailed
overview of main tasks please see the Scope document.

\subsection{System Context}
The whole system can be easily divided into two major component.  The first of
them is the Database providing an authoritative answer about the site's
topology, the other is the Monitoring Dashboard.

\subsection{Primary Stakeholders}
The following is a list client stakeholders for different areas within the project. Each area can have many reference stakeholders 
who should be consulted for requirements information gathering. Each area also has one primary stakeholder who, among all reference 
stakeholders, resolves disagreement and has final approval for requirements in that area.

\begin{table}[!h]
	\begin{tabular}{| l | l | l |}
		\hline
		\textbf{Area}		& \textbf{Primary Stakeholder} & \textbf{Reference Stakeholders}\\
        \hline
        Administrators	& Tomáš Kouba           & Tomáš Kouba\\
                          & & Jiří Horký\\
                          & & Jan Švec\\
                          & & Jan Kundrát\\
        \hline
        Management	& Jiří Chudoba           & Jiří Chudoba\\
                          & & Miloš Lokajíček\\
        \hline
	\end{tabular}
	\label{tab:PrimaryStakeholders}
\end{table}

\subsection{Acronyms and Abbreviations}

\begin{table}[!h]
	\begin{tabular}{| l | l |}
		\hline
		\textbf{Acronym / Abbreviation}		& \textbf{Expanded Term}\\
		\hline
		ACID	& Atomicity, Consistency, Isolation, Durability\\
		\hline
		API		& Application Programming Interface\\
		\hline
		CLI		& Command-line Interface\\
		\hline
		DB		& Database\\
		\hline
		DSA		& Digital Signature Algorithm\\
		\hline
		DBMS	& Database Management System\\
		\hline
		ECO		&	Engineering Change Order\\
		\hline
		FR		&	Functional Requirement\\
		\hline
		GUI		& Graphical User Interface\\
		\hline
		HW		& Hardware\\
		\hline
		NFR		&	Non-functional Requirement\\
		\hline
		OS		& Operating System\\
		\hline
		PKI		& Public Key Infrastructure\\
		\hline
		RSA		& Rivest, Shamir, Adleman\\
		\hline
		SCM		& Software Configuration Management\\
		\hline
		SRS		& System Requirements Specification\\
		\hline
		SSH		& Secure Shell\\
		\hline
		UC		&	Use Case\\
		\hline
	\end{tabular}
	\label{tab:AcronymsAndAbbreviations}
\end{table}

\subsection{How This Document is Organized}

The following sections provide all known system requirements, including both functional and nonfunctional requirements. 
This document is complete except where noted with reference to an external source. It is helpful but not necessary to read 
the sections in a sequential order. Section 2 describes the project constraints and assumptions. Section 3 describes the 
project risks and how these will be mitigated. Section 4 describes the functional requirements (FRs) of the System. Most 
functional requirements exist to directly support business process; some exist to support the correct operation of the system 
itself. All functional requirements are described in terms of use cases. Section 5 describes the non-functional requirements 
(NFRs) of the System. Section 6 provides a Project Glossary that includes project-related terms as well as software 
development related terms.

\subsection{Engineering Change Orders}
\label{steering-committee}

In order to guarantee the immutability of the specification to both sides of the
contract, it was decided that any change to the specification that was already
agreed upon has to be approved by the Steering Committee.  This body consists of
representatives of both the team responsible for the implementation as well as
from the Institute.  Exact personal constitution is outside the scope of this
document.

Only the Steering Committee is authorized to issue Engineering Change Orders for
this project.

\subsection{References}

\begin{enumerate}
	\item Tool for Central Administration of a Grid Site Scope document
	\item	Tool for Central Administration of a Grid Site Business Process Diagrams
	\item Tool for Central Administration of a Grid Site Use Case Diagrams
	\item Tool for Central Administration of a Grid Site Project Glossary
\end{enumerate}


\section{Constraints and Assumptions}
The following sections provide additional detail on the matching sections in the Tool for Central Administration of a Grid Site 
Scope document.

\subsection{Development Process and Team Constraints}
At this time, there are no known constraints.

\subsection{Environmental and Technology Constraints}

\subsubsection{Software Constraints}
At this time, there are no known constraints.

\subsubsection{Hardware Constraints}
The server part of the system shall run on a moderately-sized x86-compatible
server-class machine.

\subsubsection{Project Language}
We assume we will present our project at international conferences. And another computing centres
may want to use our program. In addition, there is an assumption, that administrators has good
English knowledge. For these reasons, we choose English for the entire project, including web page
and all documents.


\subsection{Delivery and Deployment Constraints}
The target platform for the core parts of the application is a UNIX-like
operating system, preferably a Red Hat Enterprise Linux-compatible distribution.
All the web interfaces of the application are to be browser-independent.

The Customer expressed a strong desire to have influence in the design of
essential parts of the system.   Therefore, we have to discuss major decisions
with employees of the Institute of Physics and their administrators.

The exact time frame in which the system will be put into production is unknown
at this time.

\section{Risk Mitigation}

\subsection{Technological Risks}
There are no known technological risks at this time.

\subsection{Skills and Resources Risks}
In order to establish a well-performing team, we have examined the programming
skills of all programmers who will work on the code base of the Project.  All
design decisions will be additionally verified by the requester's
representative.

\subsection{Requirements Risks}
In order to reduce the risks and delays related with sudden incompatible changes
to the requirements, a Steering Committee will be established as specified in
section \ref{steering-committee}.  Any changes to the requirements of the
project are to be explicitly approved by the Committee.

\subsection{Political Risks}
Certain groups in the EGEE/WLCG communities have expressed their preference of
usage of various tools which are already deployed in the Grid.  Therefore, the
team shall take care to comply with these decisions unless a major technical
obstacle emerges.  Failure to do so might jeopardize the adoption of the System
throughout the Grid and shed a wrong light on the Institute.


\section{Functional Requirements}
This section defines the actors who use this system while supporting the targeted business processes, as well as the use cases this 
system provides to those actors.

\subsection{Primary Functional Requirements}
In this section, we classify the primary features of the Hotel Reservation System across three categories. This list is an updated 
version of the same list from the Scope document. Some FRs have changed priority and a few new ones have been added.\\
Essential features cannot be done without. High-value features can be done without, although it may be very undesirable to 
do so. Follow-on features are those for which it is not clear they should be included in the first release. In all cases, 
the lists are not exhaustive but include the most important features from a business perspective.

\subsubsection{Essential Features}
\begin{itemize}
	\item Central storage of authoritative information about the whole grid site
        at various levels of detail (hardware, operating system, services and
        their relations).  A SCM-managed storage of these data in human-readable
        text files.
	\item Automated generating of configuration files in response to new data
        committed to the central database.  A flexible and unified API for
        retrieving information from the DB and for invocation of the generating
        components.
  \item Implementation of sample components for core systems currently
        deployed at the Institute (Nagios, Cfengine's {\tt groups.conf}).
\end{itemize}

\subsubsection{High-Value Features}
\begin{itemize}
	  \item Data aggregation from monitoring tools to one well-arranged place
    \item Dashboard providing a view of overall system health
    \item Visualization of performance/availability metrics
\end{itemize}

\subsubsection{Follow-on Features}
\begin{itemize}
    \item Correlating data form various sources in order to discover anomalies
        in system performance with the aim to identify possible bottlenecks and
        troubles early
	  \item Implementation of modules for other tools than those that are currently used on the Institute
    \item A GUI or CLI interface for database access
\end{itemize}

\subsection{Actors}
These are the roles of persons and systems that interact with the System.

\begin{center}
	\begin{tabular}{| p{5.5cm} | p{9.5cm} |}
		\hline
		\textbf{Actor name}		& \textbf{Description}\\
		\hline
		System Administrator	& General site maintenance\\
		\hline
		Manager of the Institute	& Verifying general system reliability and availability\\
		\hline
		Students on an internship	& Limited access to the system\\
		\hline
	\end{tabular}
	\label{tab:Actors}
\end{center}

\subsubsection{Actor: System Administrator}
A system administrator is a fully qualified employee of an institution.  As
such, she has already received a training in how to operate the services and is
supposed to know what changes have the potential of breaking the whole site, and
therefore is assumed to know what she is doing.  However, if the system could
reduce the possibility of making errors, this opportunity should be explored
nonetheless.

\subsubsection{Actor: Manager of the Institute}
A management stuff is usually responsible for working with accounting data
including system availability and reliability.  She is usually also curious
about what the individual administrators do in their paid time.

\subsubsection{Actor: Students on an internship}
In several use cases, it might be beneficial for the system to allow students on
an internship a temporary and severely limited access to the central management
system.  However, as the amount of time and effort which could be invested in
their training is rather limited and because there are usually no straight legal
contracts about possible damage, it is critical to ensure there actors cannot
cause any significant damage to the system.

\subsection{Use Cases}

\begin{center}
	\begin{tabular}{| p{4cm} | p{1.5cm} | p{1.5cm} | p{7cm} |}
		\hline
		\textbf{Use Case Name} & \textbf{Priority} & \textbf{Number} & \textbf{Description}\\
        \hline
        Adding a New Machine & E & 1 & A new instance of HW is to be registered\\
        \hline
        Changing Machine Role & E & 2 & A purpose of an already existing machine is being changed\\
        \hline
        Changing Essential Properties & E & 3 & A critical property of a machine (e.g. hostname) changes\\
        \hline
        Modification to Regular Property & H & 4 & A non-critical data related to a machine changes\\
        \hline
        Deleting a Machine & E & 5 & A machine is retired\\
        \hline
        Review of Changes & E & 6 & Auditing record is created for each change performed through the system\\
        \hline
        Reinstallation of a Host & H & 7 & A complete reinstallation of a machine in needed\\
        \hline
        Verifying System Status & H & 8 & Status of each managed host is included in a monitoring dashboard\\
        \hline
        Metric Visualization & H & 9 & Graphical visualization of various performance and availability metrics\\
		\hline
        Data Correlation & F & 10 & Data from various sources are correlated to each other\\
        \hline
	\end{tabular}
	\label{tab:UseCases}
\end{center}

For more detailed information see Use Case Diagrams document and Business Process Diagrams document.


\subsection{Applications}

\begin{center}
	\begin{tabular}{| p{5cm} | p{10cm} |}
		\hline
		\textbf{Application Name} & \textbf{Description}\\ \cline{2-1}
															& \textbf{UseCases}\\
		\hline
        The Database & An authoritative storage for the database of machines\\ \cline{2-1}
                        & Supports UCs: E1, E2, E3, E4, E5 \\
        \hline
        Versioning Backend & A versioning storage for data contained in the Database \\ \cline{2-1}
                        & Supports UCs: E6 \\
        \hline
        Configuration Components & Components generating configuration for already deployed system tools \\ \cline{2-1}
                        & Supports UCs: E1, E2, E3, E4, E5 \\
        \hline
        Dashboard & A monitoring interface providing an overview of system health status and performance \\ \cline{2-1}
                        & Supports UCs: H8, H9, F10 \\
		\hline
	\end{tabular}
	\label{tab:Applications}
\end{center}

\subsection{Use Case Detailed Requirements}

\subsubsection{The Database Requirements}
This section lists all of the detailed requirements for the Database part of the system.

\begin{center}
	\begin{tabular}{| p{2.5cm} | p{12.5cm} |}
	 	\hline
		\textbf{Req. Code} & \textbf{Requirement Description}\\
		\hline
        E1-1    & Storing all required properties for each object being managed\\
        \hline
        E1-2    & Providing ACID semantics\\
		\hline
	\end{tabular}
	\label{tab:DatabaseRequirements}
\end{center}

\subsubsection{The Versioning Backend Requirements}
This section lists all of the detailed requirements for the versioning storage subsystem.

\begin{center}
	\begin{tabular}{| p{2.5cm} | p{12.5cm} |}
	 	\hline
		\textbf{Req. Code} & \textbf{Requirement Description}\\
		\hline
        E6-3    & Providing ACID semantics\\
		\hline
        E6-4    & Keeping an audit log of all performed changes including the credentials
                  of the author and the real change performed\\
        \hline
	\end{tabular}
	\label{tab:VersioningBackendRequirements}
\end{center}

\subsubsection{The Configuration Components Requirements}
This section lists all of the detailed requirements for the components which propagate changes in the DB to the managed machines.

\begin{center}
	\begin{tabular}{| p{2.5cm} | p{12.5cm} |}
	 	\hline
		\textbf{Req. Code} & \textbf{Requirement Description}\\
		\hline
        E1-5    & Integration with systems already deployed at the Institute, most notably Nagios and Cfengine\\
        \hline
	\end{tabular}
	\label{tab:ConfigurationComponentsRequirements}
\end{center}


\subsubsection{The Dashboard Requirements}
This section lists all of the detailed requirements for the monitoring dashboard.

\begin{center}
	\begin{tabular}{| p{2.5cm} | p{12.5cm} |}
	 	\hline
		\textbf{Req. Code} & \textbf{Requirement Description}\\
		\hline
        H8-6    &   Providing a well arranged view on the managed infrastructure\\
		\hline
        H9-7    &   Generating graphs depicting interesting performance and availability
                    metrics obtained from third-party sources\\
        \hline
	\end{tabular}
	\label{tab:DashboardRequirements}
\end{center}



\section{Non-Functional Requirements}
Note: non-functional requirements are distinguished from FRs by starting each requirement number above 100. Therefore, the first 
NFR for Use Case \#E1 would be given the code E1-101.

\subsection{Performance}
This System will not require significant demands on the hardware.

\subsubsection{Current Release}

\begin{center}
	\begin{tabular}{| p{2.5cm} | p{12.5cm} |}
		\hline
		\textbf{Req. Code} & \textbf{Requirement Description}\\
		\hline
		E1-101	& Each change of the data in the DB should complete in 30 seconds\\
		\hline
	\end{tabular}
	\label{tab:PerformanceRequirements}
\end{center}

\subsubsection{Future Releases}
No additional performance requirements for future releases.


\subsection{Scalability}

\subsubsection{Current Release}

\begin{center}
	\begin{tabular}{| p{2.5cm} | p{12.5cm} |}
		\hline
		\textbf{Req. Code} & \textbf{Requirement Description}\\
		\hline
		E1-102	& The system should scale up to ten thousands of individually managed entities\\
		\hline
	\end{tabular}
	\label{tab:ScalabilityRequirements}
\end{center}

\subsubsection{Future Releases}
No additional scalability requirements for future releases.


\subsection{Availability}

\subsubsection{Current Release}

\begin{center}
	\begin{tabular}{| p{2.5cm} | p{12.5cm} |}
		\hline
		\textbf{Req. Code} & \textbf{Requirement Description}\\
		\hline
		E1-103	& The system should be resilient to various network and system outages\\
		\hline
	\end{tabular}
	\label{tab:AvailabilityRequirements}
\end{center}

\subsubsection{Future Releases}
No additional availability requirements for future releases.


\subsection{Reliability}

\subsubsection{Current Release}

\begin{center}
	\begin{tabular}{| p{2.5cm} | p{12.5cm} |}
		\hline
		\textbf{Req. Code} & \textbf{Requirement Description}\\
		\hline
		E1-104	& The administrators shall have the ability to review each change for correctness
        and to block disruptive changes from propagating to production systems\\
		\hline
	\end{tabular}
	\label{tab:ReliabilityRequirements}
\end{center}

\subsubsection{Future Releases}
No additional reliability requirements for future releases.


\subsection{Security}

\subsubsection{Current Release}

\begin{center}
	\begin{tabular}{| p{2.5cm} | p{12.5cm} |}
		\hline
		\textbf{Req. Code} & \textbf{Requirement Description}\\
		\hline
		E1-105	& All components of the system shall make use of an already existing user
        management systems, namely with SSH RSA and DSA keys and an X.509 PKI infrastructure\\
		\hline
	\end{tabular}
	\label{tab:SecurityRequirements}
\end{center}

\subsubsection{Future Releases}
No additional security requirements for future releases.


\subsection{Manageability}

\subsubsection{Current Release}

\begin{center}
	\begin{tabular}{| p{2.5cm} | p{12.5cm} |}
		\hline
		\textbf{Req. Code} & \textbf{Requirement Description}\\
		\hline
		E1-106	& All user-tunable system settings will be integrated into the Configuration Components\\
        \hline
        E1-107  & A strict API shall be defined for the Configuration Components and the DB to adhere to\\
		\hline
	\end{tabular}
	\label{tab:ManageabilityRequirements}
\end{center}

\subsubsection{Future Releases}
No additional manageability requirements for future releases.


\subsection{Usability}

\subsubsection{Current Release}

\begin{center}
	\begin{tabular}{| p{2.5cm} | p{12.5cm} |}
		\hline
		\textbf{Req. Code} & \textbf{Requirement Description}\\
		\hline
		E1-108	& All information in the database shall be managed from a CLI-only environment\\
		\hline
        F1-109  & An optional GUI/web-GUI interface for editing the Database shall be developed\\
        \hline
	\end{tabular}
	\label{tab:UsabilityRequirements}
\end{center}

\subsubsection{Future Releases}
No additional manageability requirements for future releases.


\subsection{Maintainability}

The Institute expects to use the system in production for the next ten years.
It is therefore expected to stick with programming languages that will be
available and supported in this time frame when developing the application.

In order to make such a long-term maintenance feasible, the team will focus on
covering the system with test cases when reasonable.

The whole code base of the project will be stored in a SCM repository and a
reliable bug tracker will be deployed in order to ensure that user-reported
problems will be handled in a timely manner.  Based on previous experience, the
Git source code management tool shall be employed for storing the source codes,
while interaction with system users (the employees of the Institute of Physics)
shall be conducted via the issue tracker integrated in the Redmine project management tool.

In order to further increase efficiency when dealing with the customer's issues,
optional direct communication channels are expected to be established, too.

When dealing with bugs, a high priority will be given to issues directly
blocking production use of the system.  The last call in specifying a bug's
priority will be given to the real administrators.

A list of bug priorities are the following:
\begin{itemize}
	\item Severity 1, Fatal - Entire system is affected and cannot be used at
        all, or produces results which put a critical part of the infrastructure
        in danger.
	\item Severity 2, Major - Part of the system or critical functionality is affected and an acceptable workaround is not possible.
	\item Severity 3, Medium - A functional problem requires increased effort to avoid via a temporary workaround. The problem cannot be indefinitely deferred.
	\item Severity 4, Minor - Functioning of system is not significantly impaired and users can live with the problem for now.
	\item Severity 5, Enhancement - A change or addition to the requirements is desired to address an unanticipated deficiency.
\end{itemize}


\subsection{Extensibility}

\begin{center}
	\begin{tabular}{| p{2.5cm} | p{12.5cm} |}
		\hline
		\textbf{Req. Code} & \textbf{Requirement Description}\\
		\hline
        E1-110	& The API between the Database and the Configuration Components
        will export enough information to enable future development of
        additional Configuration Components\\
		\hline
	\end{tabular}
	\label{tab:ExtensibilityRequirements}
\end{center}


\section{Project Glossary}
Refer Tool for Central Administration of a Grid Site Project Glossary document.


\end{document}
