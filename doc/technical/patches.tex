% vim: spelllang=en spell textwidth=120
\documentclass[deska]{subfiles}
\begin{document}

\chapter{Patches to the Existing Tools}
\label{sec:patches}

\begin{abstract}
This appendix provides a quick list of flaws or opportunities for improvements in some of the third-party tools and
libraries which we are using in Deska.
\end{abstract}

All of the patches which are discussed below are either present in the Deska source tree on the enclosed CD, or
available from the corresponding upstream bug/patch trackers.

\section{PostgreSQL}

The PostgreSQL~\cite{postgresql} database server has a feature which identifies an offending tuple that caused an
operation to fail.  The code already supports reporting conflict of {\tt UNIQUE} constraints, but no details of the row
were passed when the failure was detected by a custom {\tt CHECK} or elsewhere.  Certain operations in Deska perform
multi-row operations in one go, and it is helpful if the server provides enough information to identify which row caused
the problem.

We have developed a patch which adds a message about the offending row for the {\tt CHECK} constraints.  The patch has
been modified by one of the PostgreSQL core hackers, Tim Lane, to better fit the existing coding style, and is now
merged~\cite{postgresql-check-patch} into the {\tt master} branch.\footnote{Incidentally, the patch also led to a job
offer from a PostgreSQL-related company from a foreign, European country to the patch author.}

\section{Psycopg2}

The PsycoPg2~\cite{psycopg} library for accessing the PostgreSQL database from Python.  We have hit two issues with this
library.

One of them is an incompatible change between versions {\tt 2.4.1} and {\tt 2.4.2}.  Version {\tt 2.4.2} added a new
method of setting parameters for session isolation and transaction management via the {\tt set\_session} method, but
unfortunately changed behavior of the existing function {\tt set\_isolation\_level}.  We have changed our code to check
the version of the library in use, and the Deska server will choose the appropriate behavior to work with all supported
PsycoPg2 versions.

Another drawback of the PsycoPg2 library is its failure to pass on additional diagnostic information, as included in
PostgreSQL's error messages.  Contrary to the provided documentation, the exception object does {\em not} contain the
extra ``detail'' information, as reported by PostgreSQL.  We have not pursued this behavior further.

\section{PgPython}

The PgPython~\cite{pg-python} is an implementation of the Python~3 procedural language for PostgreSQL's stored
procedures.  Using this library, we have had issues with their routines for converting data between the PostgreSQL and
Python native data types.  It proved very hard to distinguish between the {\tt ARRAY} and {\tt TEXT} data types, a
crucial distinction on which our stored procedures for manipulating the {\tt identifier\_set} attributes heavily rely.
We have not determined the root cause of this problem and chosen instead to add special custom code for explicit
conversion of the {\tt None} values and empty strings to the expected array format.

\section{Boost.Process}

The Boost.Process~\cite{boost-process} is an out-of-tree extensions for the Boost programming libraries.  We have
extended its operation through patching the C++ iostreams {\tt streambuf} implementation to keep track of the data
flowing through the stream.  This feature proved to be very convenient for producing detailed error messages when the
client detected a mishap in the structure of the JSON data provided by the server.

During the course of implementing the patch for the above mentioned feature, we have identified a bug in the library's
handling of interrupted syscalls, where the code failed to deal with the {\tt write(2)} syscall writing less data than
asked for, and similarly {\tt read(2)} returning less bytes than expected.  We have fixed the code to handle these
situations properly, fixing a few compiler warnings in the process, but were unable to push the fix upstream as the
change depends on the previously introduced debugging infrastructure.

\section{CMake/CDash}

We use KitWare's CDash~\cite{cmake-cdash} web application, part of CMake~\cite{cmake}, as a web dashboard listing state
of the automated builds and unit tests~\cite{deska-dashboard}.  We have contributed a patch which allows it to work with
the Git~\cite{git} repository managed by Redmine~\cite{redmine}.  The patch has been accepted
upstream~\cite{cdash-redmine-patch}.

\section{JSON Spirit}

The JSON Spirit~\cite{json-spirit} library has an option to transmit the floating point numbers in a ``compact'' format,
without the trailing zeros.  This option, the {\tt remove\_trailing\_zeros}, unfortunately works a little bit too
aggressively, transmitting {\tt 10.0} as {\tt 10.}.

We have fixed the bug in the embedded copy of the JSON Spirit sources.

\end{document}
