% vim: spelllang=en spell textwidth=120
\documentclass[deska]{subfiles}
\begin{document}

\chapter{Conclusion}
\label{sec:conclusion}

The goal of the Deska suite~\cite{deska-project} has always been making the work of the grid system administrators
easier and more efficient.  Instead of duplicating the ongoing work and replacing well-working tools, we have attempted
to fill up the void between various systems deployed today and provide a central place to store all important pieces of
information about a grid computing site (\secref{sec:intro}).

We have achieved this goal through building a {\em generic}, versatile database supporting advanced object storage based
on the user-defined database scheme.  The database supports server-side querying, boasts full support for data
versioning and audit logging, and provides many advanced features which allow us to properly represent real-world
entities and relations between them using flat RDBMS~\footnote{Relation Database Management System} tables
(\secref{sec:objects-and-relations}).  To our best knowledge, no existing ORM~\footnote{Object Relation Mapping} system
supports these features, making the Deska a unique achievement among today's systems.

A generic command-line console have been built which talks to the database over a custom, well-documented protocol, and
supports working with the user-defined data in an effective and intuitive manner.  The console and the associated
scripts sport a friendly interface, support external scripting and thereby encourage deep integration with add-on tools.

The advanced features of the generic database are put into use through a database scheme reflecting varying needs of
sites throughout the WLCG computing grid~\cite{wlcg}.  We have demonstrated that the developed database layout contains
all features required to handle real-world needs of a fairly typical Tier-2 data center in Prague~\cite{farm}.  We have
converted the existing data from a proprietary in-house system into the Deska database, and based on the comments from
the site administrators, the Deska is very usable and could significantly reduce the maintenance burden imposed by the
day-to-day operation of the computing center.

The configuration generating programs and add-on scripts delivered as a part of the Deska project support each and every
feature required by the datacenter in Prague.  As of January 2012, the system is undergoing thorough testing in a pilot
mode at the {\tt prague\_lcg2} WLCG computing site, with full production deployment expected later this year.

The Deska system has been presented at three international conferences: GRID'2010 (Dubna, Russia~\cite{dubna-kundrat}),
CHEP 2010 (Taipei, Taiwan~\cite{chep-2010-deska}), and HEPiX Fall 2011 (Vancouver, Canada~\cite{hepix-2011-deska}).
Articles about Deska produced by the team have been published in the Journal of Physics~\cite{jop-deska} and in
conference proceedings (GRID'2010 at JINR, ~\cite{dubna-deska-proceedings}).

Altogether, the Deska system, as presented, solves a real-world problem, fulfills the objectives promised in its
specification, and is ready for future extensions and further maintenance.

\end{document}
