\documentclass[a4paper]{jpconf}
\usepackage[czech,english]{babel}
\usepackage[utf8]{inputenc}
\usepackage{url}
\usepackage[pdfborder="0 0 0"]{hyperref}
\usepackage{graphicx}
\usepackage{booktabs}
\usepackage{listings}

\begin{document}

\title{Deska: Tool for Central Administration of a Grid Site}
\author{Jan Kundrát, Martina Krejčová, Tomáš Hubík, Lukáš Kerpl}
\ead{kundratj@fzu.cz}
\address{Institute of Physics, AS CR, Na Slovance 22, 182 21, Prague, Czech
Republic}

\begin{abstract}
Running a typical Tier-2 site requires mastering quite a few tools for fabric management. Keeping an inventory of installed HW
machines, their roles and detailed information, from IP addresses to rack locations, is typically done using various in-house
applications ranging from simple spreadsheets to web applications. Such solutions, whose documentation usually leaves much to be
desired, typically do not prevent a significant duplication of information, and therefore the data therein quickly become obsolete.

After having deployed Cfengine as one of a few sites in the WLCG environment, the Prague Tier-2 site set forth to further automate the
fabric management, developing the Deska project. The aim of the system is to provide a central place to perform changes, from adding
new machines or moving them between racks to changing their assigned service roles and additional metadata. The database provides an
authoritative source of information from which all other systems and services (like DHCP servers, Ethernet switches or the Cfengine
system) pull their data, using newly developed configuration adaptors. An easy-to-use command line interface modelled after the Cisco
IOS-based switches was developed, enabling the data center administrators to easily change any information in an intuitive way.

We provide an overview of the current status of the implementation and describe our design choices aimed at further reducing the system
engineers' workload.
\end{abstract}

\section{Background}

One of the activities taking place at the Institute of Physics of the~AS~CR (FZU)~\cite{fzu} is running a local Tier-2 computing center
connected to an international WLCG computing grid, supporting user programs from various scientific communities from the whole world
including the D-0 experiment in Fermilab and the famous LHC accelerator from CERN.  In December 2010, the data center consisted of
roughly 350 machines, with more hardware to arrive in early 2011.  All these resources were managed by just a handful of system
administrators.  Additional hardware procurements were in the process and the overal installed capacity had been steadily growing over
the years, but the capacity of the administration team stagnated at roughly three full-time equivalents, with no matching increase in
the number of staffers planned in the intermediate future~\cite{dubna-kundrat}.

Given the impressive machines-per-administrator ratio, the staffers were exploring how to make the system administrators' life easier
and less error prone.  Duplication of information was identified as a problem significantly contributing to the stress levels and
occasional operational issues.

\section{Duplication of Information}

Consider the typical scenario when new machines arrive to the data center and have to be installed and put into
production.  The information about each node has to be scattered into smaller chunks and stored in the configuration of core networking
services; the machines have to be configured and added into various monitoring appliances.  In the end, a record for each machine has
to be kept in at least the following places:

\begin{itemize}
    \item HW inventory DB
    \item Warranty \& issue tracking
    \item Switch port configuration
    \item DHCP server
    \item DNS
    \item Cfengine roles
    \item Torque's CPU multipliers
    \item MRTG \& RRD network graphs
    \item Nagios
    \item Ganglia
    \item Munin
    \item \ldots
\end{itemize}

Some of these places are notoriously prone to omission from the update process, leading to eventual inconsistencies.  It can also be
frustrating for the staffers to duplicate work over and over again by iterating over a (usually under-documented) checklist.  A central
place storing information about all machines in the network is clearly missing and none of the existing tools marketed as inventory
managers were deemed suitable for this task.  Therefore, it has been decided that the Institute shall fund a development of a tool
which will {\em automate the configuration process} and focus on bringing the control back to a central place.

\section{Analysis of Existing Tools}

The Institute has been using an internal web application for general hardware inventory management, but without any integration with
the rest of the management stack.  The database contained information about IP addresses, hostnames and network topology, but these
pieces of information weren't actually utilized throughout the network.  The staffers were reportedly having issues with the user
interface provided by the web application.  Being accustomed to the Unix way of doing things over a command line interface, using a web
application felt like a step backwards.

Most of the existing third-party tools, proprietary or free software, again focus on the web side of affairs; it is rare to find an
appliance which offers a scripting API.  It's also rather surprising to routinely come across tools which cannot deal with machines
with different form factors than traditional ``pizza boxes'', such as blade servers or SGI twins.  Furthermore, most of the inventory
management applications are aimed mainly at automatic discovery of very heterogeneous environment, which is not the most serious issue
the Institute is facing.

Finally, even the proprietary solutions, like the System Director from IBM, offer only a limited subset of desired functionality.
While it's clear that certain features like automatic discovery and tight integration with vendor's low-level fabric management tools
are a clear benefit, these advantages are of little value when the basic functionality is simply lacking.

Considering all points mentioned above, we maintain that a new tool is indeed needed at this point, simply because no already existing
system offers all key features we require.  Therefore, the Deska suite shall focus on general usability, shall offer a CLI access,
integrated revision control for all records stored in the database, support scripting and be generic enough to allow future extensions.

\section{Architecture of Deska}

The goal of the project is to develop a database storing generic description of all the resources that a Grid site is using;
such a database should be capable of describing hardware infrastructure of a whole data center, from physical machines with
corresponding part numbers and rack locations to the network interconnects, coupling of operating system instances (aka ``hosts'') to
physical machines as well as logical relations between various services and their dependencies.  This database then acts as the
single authoritative source of information for generating configuration of all components in the system.

\subsection{Generic Database}

As the history has proven that it is extremely hard to give accurate predictions about future developments and trends in computing,
the database does not impose arbitrary restrictions to its data model.  The core DB code only assumes that it is working with {\em
collections of objects} and some {\em relations between them}. Actually defining a usable {\em scheme} of the database is a separate
problem from the general database design.  The database supports linear {\em version control} of the records contained therein.

This component of Deska is based on a PostgreSQL server wrapped with an interface implementing a Deska-specific JSON-based API.
Additional business logic for interactively providing a feedback of the impact of performed changes to the administrator is implemented
on this level, too.

\subsection{Management Interface}

Being the first part of Deska which is actually visible to the regular Unix system administrator, the application interface is similar
to a CLI-oriented interface used on enterprise-grade Ethernet switches, notably to the Cisco IOS shell.  Such a format allows storing
of plaintext dumps of the database in a version control system for auditing purposes.

The CLI interface works on the same level of abstraction as the generic database, that is, it does not contain any specific knowledge
of the database scheme being used.  All information required for the operation of the CLI is retrieved via the Deska's database API,
and no changes to the CLI source code are needed when the database scheme changes.

In future, the CLI interface will include ``wizards'' for accelerating common tasks like deploying a new machine.  A strong emphasis is
given on usability, and the CLI will support a non-interactive mode of operation suitable for scripting and batched updates.

\subsection{Database Scheme}

The database scheme is a rather abstract part of the Deska DB, but an important one nonetheless.  This scheme must be extensible to
allow installing future pieces of equipment without large-scale code changes, shall try to anticipate future trends in hardware
development, but also attain a reasonable simplicity allowing people to use it without significant pain.  The goal here is not to
achieve purity at any cost, but implement a solution which is easy to deploy, use and maintain.

\subsection{Add-on Modules}

The information contained in the database shall be actually put into real use.  Towards that goal, a set of components generating
configuration will be provided.  These modules will consult the Deska API and will be used for templating configuration for all core
network services currently deployed at the Institute, namely for Nagios, the BIND name service demon, DHCP server and Ganglia and Munin
master servers.  The design of the API reflects generic needs of any data center to allow eventual deployment of new components,
should the need arise.  All modules will use only the public and documented API for retrieving all the data they need.

These modules will be invoked when an administrator decides to push her changes to the central database.  The Deska server will
present her with a list of changes performed by herself along with the proposed new variant of the generated files.  This way, she can
assess the scope of changes she has just performed, greatly reducing the potential of a configuration error knocking down the whole
infrastructure.

In addition, tools will be written to perform consistency checks between the state recorded in the database and the actual system
shape where the automatic configuration generation is not feasible for safety reasons.  A classic example of such a system are network
switches where network administrators are rather leery of automated configuration because a single error can cause large network
segments to separate.

\section{Database Structure}

The core Deska database deals exclusively with flat lists of objects, with each list sharing objects of a common type.  An object in this case refers simply to an
equivalent of the C language {\tt struct}, that is, a data structure containing list of attributes.  Each attribute has a name and an
associated primitive data type like a string, an integer, a floating point number or an IP address.  Defining additional ``primitive
types'' is possible, but requires modifying the Deska's source code.  This way, the database layer which is responsible for rather
low-level tasks like maintaining proper revision control does not have to deal with complicated inter-relations between objects.  The
only methods implemented at the lowest layer are targeted at manipulating individual objects in the lists and performing queries against
them.  An overview of the implemented methods is available at {\tt src/deska/db/Api.h} in the Deska distribution~\cite{deska-project}.

\subsection{Abstract and Concrete Objects, Inheritance and Templates}

Suppose a typical database scheme for a data center.  It should certainly keep track of individual machines (the physical objects in
the racks), but also abstract away certain features which are defined by the brand or model of each machine.  For example, it would be
a waste of time having to specify rack height of 1U for each server instead of simply stating that it's a HP~DL-360.  Therefore, the
database has to offer a way of {\em referencing} an object of a different kind, and it makes sense to describe hardware models in a
separate object pool than the physical machines.  In the default database scheme shipped with Deska, the hardware models are of type
{\tt hwmodel} while the physical machines are of the type {\tt host}.

Further down this road, it's obvious that some models share certain properties, like their form factor.  It could therefore be
interesting for the database to support {\em specialization} of entries, where objects of the same kind have the ability to inherit
certain properties from a {\em template} and override values which have to differ.  Again, the lower layer of the database abstraction
does not have to deal with actually interpreting these values, it just has to offer a way to report this information to the upper
layers.  The Deska DB supports this specialization by the keyword {\tt template}.

The following is an example of how the textual dump of the database could look like.  We demonstrate all of the advanced concepts, like
using the {\tt template} keyword for specializing a generic definition on line 7, referencing a foreign object kind on line 17 and
embedding an additional object on lines 20 and 21.

\lstset{numbers=left}
\noindent\begin{minipage}{\textwidth}
\begin{lstlisting}
hwmodel sgi-twin
    in generic-rack
    occupies height 1
end

hwmodel xe310
    template sgi-twin
    benchmark hepspec 68
    cpu ``E5420''
    ram 16GB
    sockets 2
    cores-per-socket 4
    disk "SAS 300GB"
end

host salix01
    hw xe310
    serial X0008274

    interface eth0 mac 00:30:48:C5:42:BE net wn-nat ip 172.16.1.1
    interface bmc shared-port eth0 net monitoring ip 192.168.10.1

    role wn
end
\end{lstlisting}
\end{minipage}

Please note that the above example is rather limited and not self-contained; it is intended merely as a demonstration of how the
configuration looks like.  A much more detailed example is available in the Deska source repository~\cite{deska-project} in file {\tt
doc/cli/demonstration}.

\subsection{Building Friendly Scheme}

While the design described so far is without any doubt beneficial for the database implementation, it leaves much to be desired when
human users come into play.  Therefore, we have to build on these foundations and offer a rich, scalable layer allowing users to work
with these records without undertaking extra pain.  One example of such an extension can be well illustrated with network interfaces.
For brevity, this article simplifies matters a bit and assumes that there is a one-to-one mapping between network addresses and network
interfaces, which is not true, especially in a modern data center.

Clearly, any machine can trivially have multiple network interfaces, maybe one connected to a public network and one to a management
one.  Because the underlying database manages only flat lists of objects, one would have to explicitly create a brand new object for
each interface of each machine, greatly inflating the effort needed when deploying new machines.  That's when the upper layers of Deska
come into play.  While the dedicated object for an interface indeed gets created under the hood, this complexity is hidden by the UI and the interface
is {\em embedded} into the host to which it logically belongs.  This is a natural concept which corresponds to the real world we are
modelling.

When querying the database scheme, the API shall therefore inform the upper layers that there is a logical link between the instances
of the {\tt host} type and an {\tt interface} type, and the relation is of a special kind which allows embedding the interface
definition to the host section.  Similar relations for one-to-one mapping, as commonly found between physical machines and the running
OS image when not using virtualization, are also supported.

\subsection{Defining and Extending the Scheme}

The Deska's motto is a meme {\em don't repeat yourself} and it proclaims to reduce the information duplicity.  It would be rather
hypocrite to propose yet another domain-specific language for describing the real database scheme, as there already is a perfectly
ubiquitous language for describing tabular layouts, the SQL.  Deska is therefore shipped as a set of stored procedures and accompanying
scripts which merely look at the pre-existing structure of an SQL database, using PostgreSQL's facilities for run-time scheme inspection,
and pre-generate the database code for dealing with individual scheme being used on-site.

It is clear that the table layout cannot be arbitrary, simply because of the sheer complexity of such a design.  We therefore have to
impose a set of restrictions on the database scheme.  We start with defining a primary key to always act as a ``name'' of an item.  All
other columns have to be of one of a pre-defined set of types, and their name will directly correspond to the name of the attribute
they are modelling.

Database integrity constraints are fully supported; the only requirement is that they have an explicitly assigned name which encodes
the type of the modelled relation.  When the constraints have these names, Deska can automatically extract the relationship information
and pass it further via its API.

\section{Current Status}

As of January 2011, the core database prototype is ready and passes the initial round of rudimentary unit tests.  The API design has
been completed, and the CLI interface is currently in the works. We expect to have an instance of Deska in production by the end of
2011.

\section{Conclusion}

We have described the general structure of the Deska database, along with an explanation about our design choices.  Our approach is
believed to be future-proof and enable further extension to accommodate new, emerging paradigms which come into usage at today's data
centers.  We recognize that there is a delicate border between designing an overly generic system which is hard to use by the real
users and proposing an application which is too much tailored towards today's needs that it risks obsolescence in just a few years,
essentially blocking a sustainable development.

Some of the Deska's features, like the powerful CLI interface along with support for version control and integration of various helper
features, are, to our best knowledge, unique in its class of tools.  We are looking forward to having a usable prototype later this
year, along with the first deployment in production at the Institute of Physics in Prague.

\ack
This work was supported by the Institute of Physics of the Academy of Sciences of the Czech Republic.

\section*{References}
\bibliography{references}
\bibliographystyle{iopart-num}

\end{document}
