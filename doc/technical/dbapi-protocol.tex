% vim: spelllang=en spell textwidth=120
\documentclass{article}
\usepackage[utf8]{inputenc}
\usepackage[czech,english]{babel}
\usepackage{a4wide}
\usepackage{listings}
\usepackage{longtable}
\usepackage{etoolbox}
\usepackage{lmodern}
\usepackage[T1]{fontenc}
\usepackage{textcomp}

\begin{document}

\title{Deska: The Database Access Protocol}

\author{Jan Kundrát}

\maketitle

\newcommand{\deskaFuncRef}[1]{{\tt {#1}}}

\begin{abstract}
This chapter discusses design of the protocol that serves as an exclusive communication method for accessing the Deska
database server.  We describe the overall design, as well as provide a detailed API reference along with example usage.
This document is intended to serve as a complete source of information for both DBAPI usage and possible independent
reimplementation.
\end{abstract}

\section{Background}

First of all, it might not be obvious why bother with designing a custom protocol for accessing a database.  The Deska
database, after all, is built on PostgreSQL, which already provides a library which can be used to access database
servers.  However, using a RDBMS-specific transport library would make the Deska application suite needlessly tied to
one particular database server implementation.  Exposing a low-level SQL-like interface, even when wrapped by stored
procedures, would make it very hard to prevent users from accessing the database without first going through a
well-controlled set of entry points.  In addition to that, we argue that it is possible to provide an API which is
easier and safer to use, compared to the raw SQL usage.

\section{Protocol Overview}

The Deska Database API is modelled after a traditional RPC interface where user invokes a procedure on the remote host
and waits for its result.  This in turn determines the mode of operation, which happens in {\em command} -- {\em
response} cycles.  That is, after a client issues a particular command, it will block and wait for the server's
response.  No form of command pipelining is available at this point, but certain provisions are in place which will
allow future extensions towards non-blocking operation.

After defining the general mode of operation for an interface, there is a need to establish a method for serialization
of data sent over the wire.  There are many widely-used ways for serializing commands, from binary-only protocols to the
XML-RPC family.  For the Deska DB API, we have chosen to use JSON, a fairly recent arrival to the already broad family
of serialization standards.  The main reason behind this choice is the fact that the protocol itself is reasonably
terse, so that one can pretty easily look at the raw form of traffic and tell what is going on without using extra tools
or frameworks.  There is also quite a number of ready-made libraries for JSON parsing, so the JSON itself is as
ubiquitous as, say, the XML.

Even though the JSON standard itself is well-established, we still had to establish a common way of converting our RPC
requests to a particular JSON form. To that end, we have chosen to use JSON {\em objects}\footnote{The reader should get
herself familiar with the notion of JSON before continuing with the rest of this document, as we will not go into much
detail explaining how low-level JSON works.}, which are merely a sequence of key-value pairs, with key being a string
identifier and value an arbitrary JSON object.  

Due to certain technical limitations, a simple stream of JSON objects is rather impractical to parse.  In fact, none of
the common libraries for JSON parsing that we have tried play well with JSON streaming.  Some of these libraries would
happily attempt to blindly ``read everything'' from a stream, convert that to a string and only then attempt the actual
parsing.  Needless to say, such an approach will not work at all when the connection expects to exchange several rounds
of JSON command-response pairs, each of them occurring in response to the previous one.  Even the more advanced JSON
libraries have issues with parsing a sequence of JSON objects following each other, without any delimiters, as most of
these tools try to ``peek'' at the next character of the stream input after parsing a full JSON object, while the ``next
character'' is typically not available at that point.  Therefore, in order to facilitate the usage of these libraries,
we have chosen to mandate that each complete JSON object at the top-level scope has to be followed by a newline (LF),
the {\tt "\textbackslash n"} character.  Given that the newline character will never occur in a serialized JSON data,
this usage is completely safe and allows the protocol implementations to use per-line parsing for any IO interaction.

\section{JSON Serialization}

This section defines how an abstract notion of a command or response is converted to a JSON object.

\subsection{Commands and Responses}

First of all, each command has to identify its purpose, ie. which function to invoke on the remote end of the
connection.  A name is a perfect identifier.  Furthermore, to allow future pipelining of commands to achieve greater
throughput, we have to add some identification to each command-response pair.  This identification has to be provided by
the client.  This identification is called a {\em tag}, and apart from being a string, there are no further
requirements on its format.

Suppose the hypothetical command {\tt foo}, which does not accept any arguments.  It will be transmitted using the
following format, provided the client decided to use {\tt "0"} as the associated tag (we will omit the trailing line
feed for brevity in the following examples):

\begin{lstlisting}
    {"command": "foo", "tag": "0"}
\end{lstlisting}

A response is always constructed by simply changing the {\tt command} key to a {\tt response} literal and repeating the
{\tt tag} key, therefore a response for the above command will look like this:

\begin{lstlisting}
    {"response": "foo", "tag": "0"}
\end{lstlisting}

The {\tt response} key is included in order to allow stateless implementations of the DBAPI parsers.  Had the parameter
been omitted, it would not be possible to enforce proper structure of the JSON data without keeping track of what
command used what tag.

Please note that these provisions for {\em future} pipelining are here mainly to design the protocol in
forward-compatible way; none of the implementations shipped in Deska support command pipelining (as of 2011).  This
specification will be extended if pipelining gets allowed in future.

\subsubsection{Passing Arguments}

Most of the commands will very likely accept a few arguments.  These commands are transmitted in a very similar way to
the one we discussed above, with the arguments simply appended to the top-level JSON object.  The parameters are,
however, not repeated in the relevant response, which retains just the (response, tag) pair:

\begin{lstlisting}
    {"command": "foo", "arg1": "arg1 value",
        "name of arg2": "some value", "tag": "0"}
    {"response": "foo", "tag": "0"}
\end{lstlisting}

The JSON standard mandates that white space outside of string literals is ignored, and therefore implementations are
free to add it to make the JSON data more readable, or not include them at all.  In addition, to improve the
interoperability, the Deska DBAPI mandates that the order of appearance of individual key-value pairs inside a JSON
object is {\em not} significant, and therefore one is explicitly allowed to use a sequence of {\tt arg2}, {\tt command},
{\tt arg1} for the example above.  The order of key-value pairs in the response object can be also different.

\subsubsection{Returning Data}

So far we have only established procedures which do not return any usable data at all.  In practice, though, most of the
commands have to return a usable value.  This is achieved by extending the top-level JSON object of the response with a
single key named after the command, which contains the JSON serialization of the return value.  No other items are
added, changed or returned.

For example, if the above command was extended with a return value of {\tt string} type, a full command-response pair
would look like this one:

\begin{lstlisting}
    {"command": "foo", "arg1": "arg1 value",
        "name of arg2": "some value", "tag": "0"}
    {"response": "foo", "tag": "0", "foo": "The returned value"}
\end{lstlisting}

Again, the order of appearance of the individual key-value pairs in a particular JSON object is not relevant.

\subsection{Serializing Data}

So far, we have mentioned that the data are transfered ``serialized as JSON values'', but did not specify how this
serialization is defined and what the mapping between native data types in a programming language (like an {\tt int}, or
a {\tt string}), the database server and the JSON transport is.  The following sections provide detailed information in
this matter.  For a quick reference, please check the overview tables in section \ref{sec:json-data-types-reference} on
page~\pageref{sec:json-data-types-reference}.

\subsubsection{Primitive types}

There are four basic data types in JSON, a {\tt string}, an {\tt integer}, a {\tt float} and a {\tt boolean}.  These
values are directly used for transport of the corresponding C++ data types.  However, this mapping is not fully
bidirectional for practical reasons -- one cannot look at the serialized JSON value and deduce what the original data
type was; a context is required for that operation.  This allows us to ``overload'' these JSON data types and serialize
multiple source types using a single JSON type.  A typical example of this is any particular C++'s {\tt enum}, whose
symbolic name is transmitted as a string literal with the matching name.

Suppose the following C++ definition:

\begin{lstlisting}[language=c++]
    enum Foo {
        BAR,
        BAZ
    };
\end{lstlisting}

A value of {\tt BAR} is transferred as a JSON string {\tt "BAR"}.  There is no technical way of distinguishing between
a value of the {\tt enum Foo(BAR)} and the string literal {\tt "BAR"}; the correct interpretation is always defined by
the context in which that literal occurs.

The JSON string is further overloaded for serialization of timestamps, ie. particular moments in time.  The C++ API uses
Boost's {\tt boost::posix\_time::ptime} as a timestamp representation, and serializes the data in the {\tt "YYYY-mm-dd
hh:mm:ss"} form, like {\tt "2011-04-07 17:22:33"}.

Similarly, a date is stored into the {\tt boost::gregorian::date} and serialized using the {\tt "YYYY-mm-dd"} form, like
{\tt "2011-04-07"}.

\subsubsection{Containers}

The Deska DBAPI supports passing two different kinds of containers, a {\em sequence} and a {\em map}.  The sequence is
usually represented by the C++'s {\tt std::vector}, while the map maps well to the {\tt std::map}.

A sequence is represented as the JSON list with the individual list items being the respective serialized values of the
underlying data type. Suppose we want to send a {\tt std::vector<int>} of (1, 2, 3) as the {\tt a1} argument of command
{\tt foo}:

\begin{lstlisting}
    {"command": "foo", "tag": "0", "a1": [1, 2, 3]}
\end{lstlisting}

A map is only allowed to be indexed by string literals. A map similar to the sequence from the previous example would
look like this:

\begin{lstlisting}
    {"command": "foo", "tag": "0", "a1": ["a": 1, "b": 2, "c": 3]}
\end{lstlisting}

The example above is equivalent to a {\tt std::map<std::string, int>} which contains $"a" \rightarrow 1$, $"b"
\rightarrow 2$ and $"c" \rightarrow 3$.  Please note that the order of appearance of these items in a map is undefined,
as the {\tt std::map} container will not respect the order of insertion.  This is consistent with the general rule of
the Deska DBAPI which mandates that the order of occurrence of the individual items in a JSON object is not significant.

\subsubsection{Custom Objects}

Custom objects, ie. instances of C++'s classes, are transmitted as JSON objects. Due to various popular JSON
implementations not respecting the order of appearance of individual values, the arguments to a class' constructor are
passed prefixed by their name.  Therefore, given class {\tt Foo} with the following C++ constructor:

\begin{lstlisting}[language=c++]
    Foo::Foo(int i, const std::string &s);
\end{lstlisting}

We can modify our example command {\tt foo} to accept an instance of class {\tt Foo} like this:

\begin{lstlisting}
    {"command": "foo", "tag": "0", "a1": {"i": 10, "s": "something"}}
\end{lstlisting}

Again, the user of the Deska DBAPI cannot infer the class name from this value without knowing an extra bit of
information about the syntax of the command {\tt foo} -- without that knowledge, we are typically not aware of the
existence of class {\tt Foo} at all.  That crucial information is, as usual, determined from the current context in a
command or response.

Alternatively, certain simpler classes (like the IP addresses or MAC addresses of Ethernet interfaces) are better
transfered in their canonical string form.  As a rule of thumb, if a particular class has a rich internal structure, or
is complexity is closer to that of an {\tt Deska::Db::ObjectRelation} than to, for example, {\tt
boost::posix\_time::ptime}, it's better to use the complex form in preference to the string serialization.

It is also important to note that this section only serves as a generic guideline about how to proceed when implementing
JSON I/O for custom data types.  Considerable amount of work is required to be able to exchange custom values over JSON,
see section \ref{sec:json-extending} for details.

\subsubsection{Optional Values}

Certain data fields are designed to allow either a value of the underlying data type, or a special NULL value.  On the
C++ side, this is achieved by wrapping the data type inside a {\tt boost::optional<>} container.  In the JSON
serialization, the NULL can be simulated in two ways, either by omitting the value altogether, or by using JSON's {\tt
null} keyword as the JSON value.  In the context of optional function arguments, the Deska DBAPI does not distinguish
between the two conventions, so the following examples are equivalent, provided that the arguments {\tt a1} is defined
an accept NULL:

\begin{lstlisting}
    {"command": "foo", "tag": "0", "a1": null}
    {"command": "foo", "tag": "0"}
\end{lstlisting}

Please note that this rule does {\em not} translate to ``one can freely omit {\tt null} values in any place''; indeed,
explicitly returning {\tt null}s is compulsory e.g. when processing object attributes.

\subsection{Exceptions}

There are certain situations where the execution of a particular command terminates in an unexpected manner.  One reason
could be that the user of the API has made an error and for example tried to retrieve an object which does not exist, or
that the execution hit an implementation-specific failure which has to be communicated to the caller, like an error with
SQL processing.  For this reason and in order to prevent cluttering the API with artificial {\tt NULL} values and
special placeholders, the Deska DB API supports {\em exceptions}.

When the Deska server has encountered an error when processing a particular response, it will return a JSON object in
the usual form (ie. with {\tt "command"} replaced by {\tt "response"} and the rest of the data left intact), and instead
of adding the returned data in a key named after the API method, it will create another top-level key called {\tt
"dbException"} pointing to a JSON object representing an exception.  The new object will have the following structure:

\begin{lstlisting}
    {"type": "TypeOfException", "message": "description"}
\end{lstlisting}

Where the key {\tt "type"} is always a string identifying the exception and the {\tt "message"} always contains a string
with a textual message usable for debugging.  The object might contain more data, which could be used to pass
computer-readable data to the API user.  The usual rules for representing values apply.

This is the list of currently defined exceptions, along with their other arguments:

\begin{longtable}{ p{44mm} p{80mm} p{24mm} }
    \caption{List of supported exceptions} \\
    Exception Name & Meaning & Additional arguments \\
    \hline
    \endhead
    {\tt InvalidKindError} & The specified kind is not valid & -- \\
    {\tt InvalidAttributeError} & Cannot access the specified attribute of a given kind & -- \\
    {\tt NotFoundError} & The database does not contain such object.  This exception is returned only in case when no
    better exception is available; it means that attempts to follow an invalid kind should result in the {\tt
    InvalidKindError}, etc. & -- \\
    {\tt NoChangesetError} & Tried to perform an operation which requires being attached into a pending changeset
        without one & -- \\
    {\tt ChangesetAlreadyOpenError} & Tried to access/create/\ldots a changeset while being already attached to one & -- \\
    {\tt FreezingError} & An error related to attempted combination of a frozen view and an active changeset
    manipulation & -- \\
    {\tt FilterError} & The specified filter is not valid & -- \\
    {\tt ReCreateObjectError} & Attempted to re-create a deleted object in the same changeset. See
    \deskaFuncRef{restoreDeletedObject}. & -- \\
    {\tt RevisionParsingError} & Failed to parse the revision ID & -- \\
    {\tt ChangesetParsingError} & Failed to parse the temporary changeset ID & -- \\
    {\tt SqlError} & Execution of SQL statements resulted in an error & -- \\
    {\tt ServerError} & The server has encountered an internal error & -- \\
\end{longtable}

FIXME: more filter exceptions with fine-grained semantics

\subsection{Adding Low-Level Data Types}
\label{sec:json-extending}

Adding a low-level data type, ie. an equivalent of a {\tt string}, {\tt int} or similarly defined entity, is also
possible.  Users are strongly encouraged to contact the Deska project before undertaking these changes, if only to
prevent possible namespace clashes in future.

In general, the following has to be done to add a low-level data type to Deska:

\begin{itemize}
    \item Pick up an identifier to use for identifying the new type
    \item Create the SQL data type, if applicable
    \item Create the corresponding C++ class, if applicable
    \item Extend the {\tt Deska::Db::Value} variant to include the new type
    \item Create a specialization of the {\tt template<> struct JsonConversionTraits<>}, its {\tt extract()} and {\tt
        toJson()} methods, which are responsible for proper serialization and de-serialization of the native C++ data
        types to JSON
    \item Instantiate the {\tt template JsonField\& JsonField::extract(T*)} method
    \item Update the documentation and unit tests
    \item Add the new type to the {\tt Deska::Db::KindAttributeDataType}
\end{itemize}

\subsection{Serialization Quick Reference}
\label{sec:json-data-types-reference}

The following table provides a quick and compact overview of the officially supported data types and their respective
JSON serialization.  The first column, the name, is also used as the type identification in the {\tt kindAttributes()}
API function.

\begin{longtable}{ p{20mm} | p{29mm} p{47mm} p{30mm} }
    \caption{JSON serialization of data types for attribute values} \\
    Name & SQL Data Type & C++ Data Type & JSON Serialization \\
    \hline
    \endhead
    {\tt string} & VARCHAR & {\tt std::string} & {\tt "example text"} \\
    {\tt int} & INT & {\tt int} & {\tt 123} \\
    {\tt identifier} & VARCHAR & {\tt Deska::Db::Identifier} & {\tt "id-no-spaces"} \\
    {\tt double} & DOUBLE & {\tt double} & {\tt 333.666} \\
    {\tt date} & date & {\tt boost::gregorian::date} & {\tt "2011-04-20"} \\
    {\tt timestamp} & timestamp & {\tt boost::posix\_time::ptime} & {\tt "2011-04-20 14:28:33"} \\
    {\tt ipv4address} & FIXME & {\tt boost::asio::ip::address\_v4} & {\tt "127.0.0.1"} \\
    {\tt ipv6address} & FIXME & {\tt boost::asio::ip::address\_v6} & {\tt "::1"} \\
    {\tt macaddress} & macaddr & {\tt Deska::Db::MacAddress} & {\tt "00:16:3e:37:53:2B"} \\
    {\tt enum\_X}\footnote{See section \ref{sec:json-extending} for instructions about how to add new enumerations to both
        the Deska DB and the C++ library.} & ENUM (``A'', ``B'') & {\tt typedef enum \{A, B\} X} & {\tt "A"} \\
\end{longtable}

In addition to the types mentioned above, which are usable and allowed as the data types for objects' attributes, there
are several ``internal'' data types which are exchanged over JSON, but which cannot be used in the context of
attributes.  These types are typically utilized in transfer of various data for internal purposes of the Deska
DBAPI JSON protocol, like type-safe serialization of revisions, changeset identification, or for passing along the object kind
relation information etc.  The following types are defined:

\begin{longtable}{ p{60mm} p{90mm} }
    \caption{JSON serialization of values for internal use} \\
    C++ Data Type & JSON Serialization \\
    \hline
    \endhead
    {\tt Deska::Db::RevisionId} & {\tt "r123"} \\
    {\tt Deska::Db::TemporaryChangesetId} & {\tt "tmp123"} \\
    {\tt std::vector<T>} & {\tt [A, B, C]}
        \newline(With the exact format of A/B/C depending on type~{\tt T})\\
    {\tt std::map<std::string, T>} & {\tt \{"key1": A, "key2": B, "key3": C\}}
        \newline(With the exact format of A/B/C depending on type~{\tt T})\\
    {\tt Deska::Db::PendingChangeset::\newline{~ ~}AttachStatus} &
        One of {\tt "DETACHED"} or {\tt "INPROGRESS"} \\
    {\tt Deska::Db::PendingChangeset} &
        {\tt \{"changeset": }TemporaryChangesetId{\tt, "author": }string{\tt, "status": }
        PendingChangeset::AttachStatus{\tt, "timestamp": }timestamp{\tt, "parentRevision": }RevisionId{\tt, "message": }
        string {\tt, "activeConnectionInfo": }string/{\tt null\}} \\
    {\tt Deska::Db::KindAttributeDataType} &
        One of {\tt "string"}, {\tt "int"}, {\tt "identifier"}, {\tt "double"}, {\tt "date"}, {\tt "timestamp"}, {\tt
        "enum\_X"} or a user-defined type \\
    {\tt Deska::Db::ObjectRelation} &
        One of \newline {\tt\{"relation": "EMBED\_INTO", "target": }Identifier{\tt\}}, \newline
        {\tt\{"relation": "MERGE\_WITH", "target": }Identifier{\tt\}}, \newline
        {\tt\{"relation": "REFERS\_TO", "target": }Identifier{\tt\}}, \newline
        {\tt\{"relation": "IS\_TEMPLATE", "target": }Identifier{\tt\}} \newline or \newline
        {\tt\{"relation": "TEMPLATIZED", "target": }Identifier{\tt\}} \\
    {\tt Deska::Db::RevisionMetadata} &
        {\tt \{"revision": }RevisionId{\tt, "author": }Identifier{\tt, "timestamp": }timestamp{\tt,
        "commitMessage": }string{\tt\}}
        \\
    {\tt Deska::Db::ObjectModification} &
        One of \newline
        {\tt \{"command": "createObject", "kindName": }Identifier{\tt, "objectName": }Identifier{\tt \}}, \newline
        or \newline
        {\tt \{"command": "deleteObject", "kindName": }Identifier{\tt, "objectName": }Identifier{\tt \}}, \newline
        or \newline
        {\tt \{"command": "renameObject", "kindName": }Identifier{\tt, "oldObjectName": }Identifier{\tt,
        "newObjectName": }{\tt Identifier\}}, \newline
        or \newline
        {\tt \{"command": "setAttribute", "kindName": }Identifier{\tt, "objectName": }Identifier{\tt, "attributeName":
        }Identifier{\tt, "attributeData": }Value{\tt, "oldAttributeData": }Value{\tt \}} \\
    {\tt Deska::Db::ComparisonOperator} &
        {\tt "columnEq"} or {\tt "columnNe"} or {\tt "columnGt"} or {\tt "columnGe"} or {\tt "columnLt"} or
        {\tt "columnLe"}
        \\
    {\tt Deska::Db::Expression} &
        One of \newline
        {\tt \{"condition": }ComparisonOperator{\tt, "kind": }Identifier{\tt, "attribute": }Identifier{\tt, "value":
        }Value{\tt \}}, \newline
        or \newline
        {\tt \{"condition": }ComparisonOperator{\tt, "metadata": }Identifier{\tt, "value": }Value{\tt \}}
        \\
    {\tt Deska::Db::Filter} &
        One of \newline
        {\tt \{"operator": "and", "operands": [}Filter, Filter,\ldots{\tt]\}}, \newline
        or \newline
        {\tt \{"operator": "or", "operands": [}Filter, Filter,\ldots{\tt]\}}, \newline
        or \newline
        Expression

\end{longtable}

\section{Database Interface}

The Deska DBAPI provides functions for accessing the information contained in the database.

\subsection{The Deska DBAPI at a Glance}

The API can be divided into six groups.  The first part are commands for querying the database scheme, which are
described in section \ref{sec:api-group-dbscheme}.  These commands inform the user about what kinds of top-level objects are
defined, what relations exist among them and which attributes belong to each object.  Second part of the API
(section \ref{sec:api-group-data-retrieval}) is concerned with returning actual object data.  Going further, the third
part (section \ref{sec:api-group-data-modification}) focuses on manipulating the data in the database, while the fourth
part (section \ref{sec:api-group-vcs}) allows the user to manage the individual changesets and eventually commit the
results, or provide a frozen view of the DB state for consistency.  The fifth part (section \ref{sec:api-group-history})
discusses browsing the revision history and the last part (section \ref{sec:api-group-conf-generators}) controls
execution of the server-side configuration generators.

Each session would typically start with a call to \deskaFuncRef{kindNames} to find out what kinds of top-level objects
are defined at all.  Applications which are interested in relations between objects, like what objects are designed to
``belong'' to other objects, and therefore are best represented as an embedded part, should call
\deskaFuncRef{kindRelations} to fetch that information.  Finally, calling \deskaFuncRef{kindAttributes} for each
top-level object will return attributes valid at the scope of each object.

After establishing the metadata structure, applications which require only read-only access can proceed with retrieving
individual objects' data.  The API provides a few functions for that, starting with the \deskaFuncRef{kindInstances} for
determining identifiers (names) of particular objects, \deskaFuncRef{objectData} for fetching values of object's
attributes, and \deskaFuncRef{resolvedObjectData} as a variant of \deskaFuncRef{objectData} which natively supports
resolving inherited values from object templates.  For batched retrieval of data for multiple objects of the same kind,
use the \deskaFuncRef{multipleObjectData} and \deskaFuncRef{multipleResolvedObjectData}.

Before potentially using any methods for object modification, each application has to open a changeset to work in.  The
application can achieve that either by calling the \deskaFuncRef{startChangeset}, or possibly by attaching into one of
the already-created changesets (listed by \deskaFuncRef{pendingChangesets}) through \deskaFuncRef{resumeChangeset}.
Should the user decide not to need the pending changeset anymore, she can call the \deskaFuncRef{abortCurrentChangeset}
to permanently throw away any changes, or use \deskaFuncRef{detachFromCurrentChangeset} to ``detach'' from the
changeset, preserving the modifications for a later commit.  When all changes have been performed and the user decides to
make them permanent, she shall use the \deskaFuncRef{commitChangeset} to commit them into a persistent revision.

Certain applications, like dumping the database contents, require to temporarily ``freeze'' the state of the database to
guarantee that the result accurately represents the real database state.  The DBAPI provides a pair of functions for
this purpose, the \deskaFuncRef{freezeView} and \deskaFuncRef{unFreezeView}, see section
\ref{sec:api-revision-semantics} for details.

It is possible that someone has already committed a new version after our original user created a changeset, but before
she attempted a commit.  In that case, the new commit won't be performed, and the user should verify that the changes are still
valid via the diffing functions (see section \ref{sec:api-group-history}).  After the verification, the user shall use the
\deskaFuncRef{rebaseChangeset} to inform the database that the changeset is now based on a new, updated permanent
revision; this will allow a future \deskaFuncRef{commitChangeset} to succeed.

Objects can be created via the \deskaFuncRef{createObject} function, deleted through the \deskaFuncRef{deleteObject} and
renamed via the \deskaFuncRef{renameObject}.  Attribute values are set via the \deskaFuncRef{setAttribute} call.
Certain attributes accept a {\tt NULL} value, customary to model an unset attribute; to use it, also use
\deskaFuncRef{setAttribute}.  In case an object gets deleted by accident in a pending changeset, it can be restored to
its previous state via the \deskaFuncRef{restoreDeletedObject}.  It is an error to attempt to delete and create an
object with the same name in one changeset.  Users can perform several actions in one DBAPI command by calling the
\deskaFuncRef{applyBatchedChanges} function.

Results of these modifying operations can be consulted via the \deskaFuncRef{listRevisions},
\deskaFuncRef{dataDifference} and \deskaFuncRef{dataDifferenceInTemporaryChangeset}.  These functions work on the
immediate attributes of an object, and do not take templates into account; for functions which operate on the resolved
state of objects, use the \deskaFuncRef{resolvedDataDifference} and
\deskaFuncRef{resolvedDataDifferenceInTemporaryChangeset}.

Use the \deskaFuncRef{showConfigDiff} function to retrieve a human-readable description of how the actions performed in
the current changeset affect the generated configuration output.

\subsubsection{Revision Semantics}
\label{sec:api-revision-semantics}

Certain methods of the Deska DBAPI accept an optional argument for specifying a revision of the database data to
operate on.  When the session is attached into a pending changeset, an unspecified revision always refers to the current
state of the changeset, ie. the parent revision and all changes applied in the active changeset so far.

When the version is not specified and there is no active changeset, though, the implicit revision value means ``whatever
the database connection happens to see at the moment'', which clearly lacks important consistency promises.  There are
few ways to achieve consistent operation.  The first option is to always use absolute revision specification in each
query,  the second option is to operate from inside a changeset, and the third is to start a lightweight multiversion
control transaction on the database side by means of \deskaFuncRef{freezeView}.  This transaction is freed upon
disconnect from the database, or by an explicit call to the \deskaFuncRef{unFreezeView}.  It is an error to try to start
or resume a changeset when in the multiversion transaction and vice versa.

\subsubsection{Filtering Results}
\label{sec:api-filters}

Some functions defined in the DBAPI support reducing the working set of the data that they operate on via {\em filters}.
These functions are \deskaFuncRef{kindInstances}, \deskaFuncRef{multipleObjectData},
\deskaFuncRef{multipleResolvedObjectData}, \deskaFuncRef{listRevisions}, \deskaFuncRef{pendingChangesets} and all
diffing functions (\deskaFuncRef{dataDifference}, \deskaFuncRef{dataDifferenceInTemporaryChangeset},
\deskaFuncRef{resolvedDataDifference} and \deskaFuncRef{resolvedDataDifferenceInTemporaryChangeset}.  Each of them
allows the user to specify a condition which, if present and not null, will be applied to the domain of the data they
operate on, and records which do not match the user-supplied filter will be excluded from further evaluation.  This is
intended to serve as an accelerator for operations which are best handled with server-side searching.

A basic unit for building filters is a simple {\em expression}.  There are two basic kinds of expressions, an expression
referring to a revision or changeset {\em metadata}, or an expression comparing the value of an object's attribute.  In
both cases, the comparison is performed using defined comparison operator against a user-supplied constant.  At this
time, the defined comparison operators are equality, non-equality, being greater/lesser than, and greater-or-equal and
less-or-equal.  The constant value against which a property is compared can be anything which is acceptable in the
context of a Deska attribute value, a revision identification, a temporary changeset identification, or any similar
revision (or changeset) metadata.

When comparing against revision or changeset metadata, the ``property name'' is specified using the {\tt metadata} key,
and can refer to any property which is normally returned by the function being extended with a filter (like revision
IDs, commit/changeset timestamps, revision authors, state of a pending changeset etc.).  When filtering by a Deska
object's attribute, the key has to be specified by a tuple of {\tt kind} and {\tt attribute} identifiers.

More complicated conditions can be created by chaining the expressions into logical conjunction and disjunction.
Implementations are expected to limit the maximal allowed depth of operator nesting to prevent overly expensive
operations and to mitigate a possible DoS attack.

Under certain circumstances, it is be useful to allow querying of objects using indirect criteria, like ``return all
hosts with an interface that is connected to a particular VLAN''.  The \deskaFuncRef{multipleObjectData} querying {\em
hosts} should be used for this purpose, and its filter shall be set to the following:

{\tt \{"condition": }ComparisonOperator{\tt, "kind": "interface", "attribute": "vlan", "value": }Value{\tt \}}

The database server will parse this filter and attempt to issue a corresponding {\tt JOIN} statement, connecting the
corresponding tables to each other via correct relation.  Using these
``foreign'' attributes is allowed only in case that both kinds are connected via a compatible relation.  If there is no
such relation, the Deska server returns the {\tt FilterError} exception.

FIXME: change the sentence above when fine-grained filter exceptions are introduced

\subsubsection{Output Ordering}

Certain functions return a list of values.  The general rule is that the order of returned items is not stable, unless
specified otherwise.  The sorting may even vary among several executions of the same command with the very same
arguments executed in a single session.  That said, there are certain situations where sorting the results on the server
side clearly improves the client's efficiency, and is therefore fixed for the DBAPI.

Any function which returns a list of pending changesets or a list of persistent revisions is guaranteed to sort the
result by the changeset or revision, lower numbers first.  Examples of these functions are \deskaFuncRef{listRevisions}
and \deskaFuncRef{pendingChangesets}.

Functions which return lists of objects or their identifiers, like \deskaFuncRef{kindNames},
\deskaFuncRef{kindInstances} etc, do not provide any guarantees about the sort order.

The diffing functions (see section \ref{sec:api-group-history}) sorts the results according to special rules, see their
description for more details.

\subsection{API Reference}

\newcommand{\deskaFunc}[5]
{\paragraph{#1}\label{sec:api-ref-#1}

{#4}

{#5}

\subparagraph{Sample JSON input} {\tt \{"command": "{#1}", "tag": "T"{#2}\}}

\subparagraph{Sample JSON output} {\tt \{"response": "{#1}", "tag": "T"\ifstrequal{#3}{}{}{, "{#1}": {#3}}\}}

}

\newcommand{\deskaUnsortedRes}{The order of the returned data is not specified.}

% usage: \deskaFunc{id}{comma and a list of arguments}{returned value}{summary}{detail}
\setcounter{secnumdepth}{4}

\subsubsection{Querying the Database Scheme}
\label{sec:api-group-dbscheme}

Functions in this group are used to query the scheme of the database, ie. a list of allowed objects, their attributes
and the defined object relations.

\deskaFunc{kindNames}{}{["z", "a", "b", "foo bar"]}
    {Return list of names of configured top-level Kinds}
    {This function returns a list of configured top-level objects.  Any of these objects can be subsequently used in the
    {\tt kindName} argument to various other functions.  \deskaUnsortedRes}

\deskaFunc{kindAttributes}{, "kindName": "something"}
    {\{"bar": "int", "baz": "identifier", "foo": "string", "price": "double"\}}
    {Return a list of attributes defined for a particular kind identified by {\tt kindName}}
    {The returned data is a list of \textless name, datatype\textgreater { }pairs.  \deskaUnsortedRes}

\deskaFunc{kindRelations}{, "kindName": "something"}
    { [ \\
            \{"relation": "EMBED\_INTO", "target": "hardware"\}, \\
            \{"relation": "MERGE\_WITH", "target": "second-kind"\}, \\
            \{"relation": "REFERS\_TO", "target": "second-kind"\}, \\
            \{"relation": "IS\_TEMPLATE", "target": "target-kind"\}, \\
            \{"relation": "TEMPLATIZED", "target": "by-which-kind"\} \\
            ]}
    {Retrieve relations between different Kinds}
    {This function returns a list of relations for the specified kinds of entities -- for more details and examples, see
    the ObjectRelation struct.  \deskaUnsortedRes}

\subsubsection{Retrieving Object Data}
\label{sec:api-group-data-retrieval}

Functions from this group are used for fetching actual data from the database.

\deskaFunc{kindInstances}{, "kindName": "something", "revision": "r123"}
    { ["a", "b", "c"] }
    {Returns identifiers of all objects of a given kind}
    {This function returns a list of identifiers describing all objects of a specified kind which are present in the
    database in the specified revision.  The {\tt revision} argument is optional, and if missing, its assumed value
    depends on the usual rules -- see section \ref{sec:api-revision-semantics} on page
    \pageref{sec:api-revision-semantics} for details.  Another optional argument, {\tt filter}, can be used to select
    a subset of objects to operate on.  The filter expressions can reference not only the immediate attributes of the
    selected object kind, but can also follow defined object relations to create more complex queries.  \deskaUnsortedRes}

\deskaFunc{objectData}{, "kindName": "kind", "objectName": "object", "revision": "r123"}
    {\{"foo": "bar", "baz": "id", "int": 10, "real": 100.666, "price": 666\}}
    {Get all attributes for a particular object instance}
    {This function returns values of all attributes which are defined for the given object.  No template resolving is
    performed, see \deskaFuncRef{resolvedObjectData} for a function which operates on the resolved object state.

    This function is guaranteed to return all attribute instances, including the {\tt null} values.  The only exception are
    embedded objects, for which the \deskaFuncRef{kindAttributes} returns a special attribute referring to the foreign
    key reference (in the demo setup, this would be the {\tt host} attribute for an {\tt interface}), but this special
    attribute is always missing from the \deskaFuncRef{objectData} output.

    For a batched equivalent for bulk retrieval of object data, see \deskaFuncRef{multipleObjectData} and
    \deskaFuncRef{multipleResolvedObjectData}.  \deskaUnsortedRes
    }

\deskaFunc{multipleObjectData}{, "kindName": "kind", "filter": Filter, "revision": "r123"}
    {\{"object1": \{"foo": "bar1", "baz": "id1", "int": 1, "real": 100.666, "price": 666\}, \\
       "object2": \{"foo": "bar2", "baz": "id2", "int": 2, "real": 200.666, "price": 333\}
      \}}
    {Get all attributes for instances matching the specified filter}
    {This is the batched equivalent of \deskaFuncRef{objectData}.  Each object which matches the criteria will be
    present in the result, even if it has no attributes.  The amount of data returned is the same as in
    \deskaFuncRef{objectData} function.  \deskaUnsortedRes}

\deskaFunc{resolvedObjectData}{, "kindName": "kind", "objectName": "object", "revision": "r123"}
    {\\ \{"foo": ["obj-defining-this", "bar"], "baz": ["this-obj", 666]\}}
    {Get effective values for all attributes for a particular object instance, taking templates into account}
    {This function returns effective values of all attributes defined for a particular object or inherited from its
    template parent along with the identifier of the object at which level the value is specified.

    The amount of returned data is always the same as the \deskaFuncRef{objectData} result.  \deskaUnsortedRes}

\deskaFunc{multipleResolvedObjectData}{, "kindName": "kind", "filter": Filter, "revision": "r123"}
    { \{ \\
        "obj1": \{"foo": ["obj-defining-this", "bar"], "baz": ["obj1", 666]\}, \\ 
        "obj2": \{"foo": ["obj-defining-this", "bar"], "baz": ["obj2", 333]\} \\ 
      \} }
    {Gets effective values for all attributes of a matching set of objects, taking templates into account}
    {This is the batched version of the \deskaFuncRef{resolvedObjectData}.  \deskaUnsortedRes}

\subsubsection{Performing Modifications}
\label{sec:api-group-data-modification}

Any function which modifies objects stored in the Deska database belongs to this group.

\deskaFunc{deleteObject}{, "kindName": "something", "objectName": "name"}
    {}
    {Delete an instance of an object}
    {This function deletes a particular instance of an object of a given type from the database.}

\deskaFunc{createObject}{, "kindName": "something", "objectName": "name"}
    {}
    {Create a new instance of an object}
    {This function creates a new instance of a given object.  All attributes are initially unset, which means that you
    might have to call \deskaFuncRef{setAttribute} before you will be allowed to commit your changes if some attributes
    have a constraint not to be NULL.

    It is an error to attempt to create an object which has been deleted in the current changeset; see
    \deskaFuncRef{restoreDeletedObject} for details.  Doing so will return the {\tt ReCreateObjectError} exception.}

\deskaFunc{restoreDeletedObject}{, "kindName": "something", "objectName": "name"}
    {}
    {Undo an object deletion that was performed in the same pending changeset}
    {This function exists in order to enable ``undoing'' an object deletion in the same changeset where the deletion was
    attempted.  Due to the way how changesets are implemented, it is not allowed to delete an object and simultaneously
    create a new one with the same name in just one changeset.  Therefore, attempting to re-create a just deleted object
    as a remedy of the inadvertent action via \deskaFuncRef{createObject} will fail with the {\tt ReCreateObjectError}
    exception.  This function will instead completely undo the deletion, without leaving any trace in the history.  The
    object is completely restored, including all attributes and references.}

\deskaFunc{renameObject}{, "kindName": "something", "oldObjectName": "name", "newObjectName": "name"}
    {}
    {Change the name of an object}
    {This function will rename an object.  All references to the old name are automatically updated to point to the new
    one.}

\deskaFunc{setAttribute}{, "kindName": "something", "objectName": "name", "attributeName": "name", "attributeData": "something"}
    {}
    {Set an object's attribute to something}
    {This function will update the value of a particular object's attribute.  To ``remove'' an object's attribute, set its
    value to {\tt null}.}

\deskaFunc{applyBatchedChanges}{, "modifications": [ \\
    \{"command": "createObject", "kindName": "k1", "objectName": "o1"\}, \\
    \{"command": "deleteObject", "kindName": "k2", "objectName": "o2"\}, \\
    \{"command": "renameObject", "kindName": "k3", "oldObjectName": "ooooold", "newObjectName": "new"\}, \\
    \{"command": "setAttribute", "kindName": "k5", "objectName": "o5", "attributeName": "a5", \\
    "attributeData": "new", "oldAttributeData": "old"\} \\
    ]}
    {}
    {Apply a list of batched modifications at once}
    {Use this function to send a list of modifications to the database in one network operation, potentially saving
    substantial amounts of time.

    The format of the list of changes to be performed is the same as what the various diffing functions (section
    \ref{sec:api-group-history} return.)
    }

\subsubsection{Version Control}
\label{sec:api-group-vcs}

This group contains functions for creating, modifying and committing changesets.  For history retrieval, see the {\em
Querying History}, section \ref{sec:api-group-history}.

\deskaFunc{startChangeset}{}
    {"tmp123"}
    {Create a temporary changeset for modifying the DB}
    {This function will create a new changeset and immediately attach current session to it.  All subsequent changes
    performed in the session will be stored in this temporary changeset.

    It is an error to attempt to create a new changeset while being already attached to another one, or to combine
    changeset access with frozen views.
    }

\deskaFunc{commitChangeset}{, "commitMessage": "message"}
    {"r123"}
    {Commit current in-progress changeset}
    {This operation will commit the temporary changeset (ie. everything since the corresponding
    \deskaFuncRef{startChangeset}/\deskaFuncRef{resumeChangeset} call) into the production DB.  A new persistent
    revision will be created.}

\deskaFunc{rebaseChangeset}{, "parentRevision": "r666"}
    {}
    {Make current in-progress changeset appear as a child of a specified revision}
    {In order to prevent a possible loss of information, Deska won't allow a commit of an in-progress changeset to the
    persistent, production revisions unless the latest persistent revision is the same as was present at the time the
    user started working on her in-progress copy. For example, if there was a revision X and user A started working on a
    changeset J, and while the J still was not comitted, another user went ahead and created revision X+1, user A won't
    be able to push her changes to the DB, as the J changeset is internally marked as "I'm based on revision X". In
    order to be able to push J and turn it into a persistent revision, it has to be explicitly marked as derived from
    X+1, which is exactly what this function performs.}

\deskaFunc{pendingChangesets}{}
    {[
     \{"changeset": "tmp123", "author": "user", "status": "DETACHED", "timestamp": "2011-04-07 17:22:33",
     "parentRevision": "r666", "message": "message", \\ "activeConnectionInfo": null\}
    ]}
    {Return a list of pending changesets}
    {This function returns a vector of metadata for pending changesets which are currently available in the
    database.

    An optional argument {\tt filter} can be used to limit the amount of data returned.  Filters can reference any
    available property of the temporary changeset.

    The results are sorted by the changeset identification using numeric sort of its value, lowest changesets first.}

\deskaFunc{resumeChangeset}{, "revision": "r123"}
    {}
    {Re-open a pre-existing changeset}
    {This function will attach current session to a pre-existing changeset which hasn't been committed yet. An example
    where doing that would be handy is upon the initial connect to the DB, where the client would typically call
    \deskaFuncRef{pendingChangesets}, and ask the real person whether she wants to resume working on her changes,
    perhaps because the original session has died.}

\deskaFunc{detachFromCurrentChangeset}{"message": "clarify why"}
    {}
    {Detach this session from its active changeset}
    {The purpose of this function is to facilitate a way to temporarily detach from a changeset which still needs some
    time before it could be committed.

    There could be many changesets in the Deska DB (all stored on the Deska DB server) which might belong to various
    users.  Each session (ie. a connection to the database) could only have at most one particular changeset associated
    with itself, and that changeset is called the {\em current} one.  That changeset that will receive all updates from
    the functions which perform database modifications.

    After the former active changeset is detached, it remains available for further processing via the
    \deskaFuncRef{resumeChangeset} function, but the current session is not associated with an active changeset anymore.
    This is intended to make sure that user has to explicitly ask for her changes to be "set aside" instead of doing
    that implicitly from inside \deskaFuncRef{startChangeset}.}

\deskaFunc{abortCurrentChangeset}{}{}
    {Abort an in-progress changeset}
    {This function will throw away any changes made so far in the current active changeset and remove that changeset
    from the list of pending changestes.}

\deskaFunc{freezeView}{}{}
    {Freeze the view of the revisions}
    {Start a server-side multiversion transaction which freezes the server's idea about what the latest revision in the
    database is. See section \ref{sec:api-revision-semantics} for details.

    It is an error to start or resume a changeset while the database state is frozen via this function; doing so will
    result in the {\tt FreezingError} exception.}

\deskaFunc{unFreezeView}{}{}
    {Unfreeze the client's view on the persistent revisions}
    {This function removes the frozen view of the database state.  See \deskaFuncRef{freezeView} and section
    \ref{sec:api-revision-semantics} for details.}

\subsubsection{Querying History}
\label{sec:api-group-history}

Functions for browsing history of individual objects, retrieving lists of differences, etc.

\deskaFunc{listRevisions}{}
    {[
    \{"revision": "r123", "author": "user", "timestamp": "2011-04-07 17:22:33", "commitMessage": "message"\}
    ]}
    {Return a list of metadata for matching revisions}
    {This function returns a list of metadata for revisions matching the search criteria.  When no search criteria are
    specified, it will return metadata for all persistent revisions.  The filter can reference all attributes of a
    revision, as well as any object data.

    The result is sorted by the revision using a numeric sort, with lowest values returned first.}

\deskaFunc{dataDifference}{, "revisionA": "r1", "revisionB": "r2"}
    {[ \\
    \{"command": "createObject", "kindName": "k1", "objectName": "o1"\}, \\
    \{"command": "deleteObject", "kindName": "k2", "objectName": "o2"\}, \\
    \{"command": "renameObject", "kindName": "k3", "oldObjectName": "ooooold", "newObjectName": "new"\}, \\
    \{"command": "setAttribute", "kindName": "k4", "objectName": "o4", "attributeName": "a4", \\
        "attributeData": "new", "oldAttributeData": "old"\} \\
    ]}
    {Return differences between the database state in the specified versions}
    {This function will produce a list of differences between the database state in the two indicated revisions.  The
    returned value is a list of ``basic modifications'', each representing creation of a new object, deletion of an
    existing one, or setting a single attribute.

    The order of the returned data is significant; for example, if an object has been created and its attribute set to a
    particular value, the creation shall be present before the record for setting an attribute.

    An optional filter is allowed to reference any object's attributes, as well as revision metadata.

    The format of the returned data is the same as the argument of \deskaFuncRef{applyBatchedChanges}.}

\deskaFunc{dataDifferenceInTemporaryChangeset}{, "changeset": "tmp123"}
    {[ \\
    \{"command": "createObject", "kindName": "k1", "objectName": "o1"\} \\
    ]}
    {Return differences of the database state in the specified changeset compared against its parent revision}
    {See \deskaFuncRef{dataDifference} for details.}

\deskaFunc{resolvedDataDifference}{, "revisionA": "r1", "revisionB": "r2"}
    {[ \\
    \{"command": "createObject", "kindName": "k1", "objectName": "o1"\}, \\
    \{"command": "deleteObject", "kindName": "k2", "objectName": "o2"\}, \\
    ]}
    {Return differences in the resolved versions of object states between the specified versions}
    {This function behaves in the exactly same way as the \deskaFuncRef{dataDifference}, except that the changes are not
    between the ``raw'' object attributes, but reflect the differences in the fully resolved versions of data, taking
    templates  into account.}

\deskaFunc{resolvedDataDifferenceInTemporaryChangeset}{, "changeset": "tmp123"}
    {[ \\
    \{"command": "createObject", "kindName": "k1", "objectName": "o1"\} \\
    ]}
    {Return differences between the resolved database state in the specified changeset and its parent revision}
    {See \deskaFuncRef{dataDifferenceInTemporaryChangeset} and \deskaFuncRef{resolvedDataDifference} for details.}

\subsubsection{Server-side Configuration Generators}
\label{sec:api-group-conf-generators}

Functions that control generating of configuration files on the server side.

\deskaFunc{showConfigDiff}{, "forceRegenerate": true}
    {"human-readable-output"}
    {Show the human readable difference in the generated configuration, as determined by changes in the current
    changeset.}
    {This function will make sure that the server has run its configuration generators at least once for the current
    state of the active changeset, and then return a human readable description of these changes.  The format of the
    resulting data is not meant for machine processing, but for a direct consumption by humans.

    If the {\tt forceRegenerate} attribute is present and {\tt true}, the server will forcefully discard its pre-cached
    copy of the configuration output, and will generate a fresh one.}

\end{document}
