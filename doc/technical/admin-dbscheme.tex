% vim: spelllang=en spell textwidth=120
\documentclass[deska]{subfiles}
\begin{document}

\chapter{Configuring the Deska Scheme}
\label{sec:admin-dbscheme}

\begin{abstract}
This chapter leads the Deska administrator through the process of customizing the database scheme to individual site's
needs and the deployment of the database side.
\end{abstract}

\section{Creating the Scheme}

Before the Deska system is usable, one has to configure it with a proper {\em Deska scheme}.  Users working with the
{\tt deska-cli} work with kinds, object instances, their attributes and relations between them (consult
\secref{sec:objects-and-relations}), and the {\em scheme} is where the Deska administrator can define how these features
are going to look like.

The primary language of defining the Deska scheme is the data definition language (DDL) of SQL.  Ordinary SQL tables are
used for specifying supported kinds, the columns in there correspond to their attributes, and additional constraints and
triggers place further limits to the data and define the Deska relations.  All of that is explained later in this
chapter.

\subsection{Kind Creation}
If you want to create a kind, you need to create a table in SQL with the same name as your kind should have and with the
same column names as your kind's attributes should have.  Supported types of columns are listed in
\secref{sec:db-att-types} and a list of requirements on the tables are in \secref{sec:db-scheme-req}.  On a coding style
front,  it is recommended to have one table in each SQL file and prefix the file names with a numerical prefix to
clearly specify the order in which they are created.

\section{Requirements on the Table Layout}
\label{sec:db-scheme-req}

Every SQL file should start with setting {\tt search\_path} to schemas {\tt production} and {\tt deska}.

Every SQL table provided by the user needs to:
\begin{itemize}
    \item Have primary key column {\tt uid} of the {\tt bigint} type
    \item Have a {\tt UNIQUE}, {\tt NOT NULL} column {\tt name} of the {\tt identifier} type; the embedded tables should instead use a
        unique pair of a {\tt uid} and a referencing column (see \secref{sec:db-scheme-refers-to} for an example)
    \item Use a sequence's {\tt nextval} as the default value of the column {\tt uid}
\end{itemize}

Following example shows a simplest table to be used in Deska:

\begin{minted}{sql}
SET search_path TO production, deska;

CREATE SEQUENCE vendor_uid START 1;

-- vendor is the name of this kind
CREATE TABLE vendor (
    -- An internal primary key
    uid bigint DEFAULT nextval('vendor_uid')
        CONSTRAINT vendor_pk PRIMARY KEY,
    -- Name as the user-visible identifier
    name identifier
        CONSTRAINT vendor_name_unique UNIQUE NOT NULL
    -- Additional kind attributes, constraints and relation definitions shall be placed here
);
\end{minted}

Do not use table names starting with a prefix {\tt "inner\_"} or ending with {\tt "\_template"} as Deska uses these
names for internal purposes.  The maximum total length of a kind name and an attribute name together is 20 characters.
Foreign keys have to retain the {\tt ON DELETE NO ACTION} stanza, a default option. In addition, all foreign keys need
to be explicitly set to {\tt DEFERRABLE}.

\subsection{Attribute types}
\label{sec:db-att-types}
To limit the system complexity, we support only a subset of the PostgreSQL data types with some additions that come
handy in a typical data center environment.  Here is a list of the supported types:

\begin{itemize}
	\item{identifier} - this is the Deska identifier, as used for object names
	\item{identifier\_set} - set of identifiers used for multi-reference attributes
	\item{int} - for integers
	\item{bigint} - for large integers and internal UIDs
	\item{real} - for floating-point numbers
	\item{text} - for strings
	\item{ipv4} - for IPv4 addresses
	\item{ipv6} - for IPv6 addresses
	\item{macaddr} - for Ethernet MAC addresses
	\item{date} - for specifying a date
	\item{timestamp} - for timestamp
\end{itemize}

\subsection{Relations Definition}
Here we will describe how to add Deska relations (see \secref{sec:objects-and-relations-relations}) to the defined kinds.

\subsubsection{REFERS\_TO}
\label{sec:db-scheme-refers-to}

The {\tt REFERS\_TO} relation (see \secref{sec:relation-refers-to}) is defined by adding a foreign key constraint. This
constraint has to comply with following rules:

\begin{itemize}
    \item The target module table has to be already present in the database
    \item The referring column forming the relation has to be of the {\tt bigint} type
    \item The name of the constraint shall start with the {\tt rrefer\_} prefix
    \item The foreign key shall always reference the {\tt uid} column of the target table
\end{itemize}

\begin{minted}{sql}
SET search_path TO production,deska;

CREATE SEQUENCE hardware_uid START 1;

CREATE TABLE hardware (
    uid bigint DEFAULT nextval('hardware_uid')
        CONSTRAINT hardware_pk PRIMARY KEY,
    name identifier
        CONSTRAINT hardware_name_unique UNIQUE NOT NULL,
    vendor bigint 
        CONSTRAINT rrefer_hardware_fk_vendor REFERENCES vendor(uid) DEFERRABLE,
);
\end{minted}

\subsubsection{EMBED\_INTO}

The {\tt EMBED\_INTO} relation (see \secref{sec:relation-embed-into}) is defined by adding a foreign key constraint
obeying the following rules:

\begin{itemize}
    \item The target module table has to be present in the database already
    \item The referring column forming the relation has to be of the {\tt bigint} type
    \item The name of the constraint shall start with the {\tt rembed\_} prefix
    \item The foreign key shall always reference the {\tt uid} column of the target table
\end{itemize}

In contrast to the ``ordinary'' Deska tables which have a constraint on a unique value of column {\tt name}, on case of
the embedded tables this constraint shall instead enforce uniqueness of the ({\tt name}, reference-to-parent) columns,
similar to the {\tt interface\_pk\_namexhost} in the following example:

\begin{minted}{sql}
SET search_path TO production,deska;

CREATE SEQUENCE interface_uid START 1;

-- interfaces of host
CREATE TABLE interface (
    -- this column is required in all plugins
    uid bigint DEFAULT nextval('interface_uid')
        CONSTRAINT interface_pk PRIMARY KEY,
    -- this column is required in all plugins
    name identifier NOT NULL,
    host bigint
        CONSTRAINT rembed_interface_fk_host REFERENCES host(uid) DEFERRABLE,
    CONSTRAINT interface_pk_namexhost UNIQUE (name,host)
);
\end{minted}


\subsubsection{TEMPLATIZE}

The {\tt TEMPLATIZE} relation needs (see \secref{sec:relation-templatized}) adding a special column to the module. This
column has to comply with following rules:

\begin{itemize}
    \item The name of the column shall start with {\tt template\_} prefix
    \item There shall be only one column in the table whose name starts with {\tt template\_} prefix
    \item The column shall be of the {\tt bigint} type
\end{itemize}

You can template only tables that have at least one column whose value could be inherited from its template.  Columns
that form a {\tt COMPOSITION}, a {\tt TEMPLATIZE} or an {\tt EMBED\_INTO} relation could not inherit their values and
therefore do not count.

\begin{minted}{sql}
SET search_path TO production,deska;

CREATE SEQUENCE hardware_uid START 1;

CREATE TABLE hardware (
    -- this column is required in all plugins
    uid bigint DEFAULT nextval('hardware_uid')
        CONSTRAINT hardware_pk PRIMARY KEY,
    -- this column is required in all plugins
    name identifier
        CONSTRAINT hardware_name_unique UNIQUE NOT NULL,
    vendor bigint 
        CONSTRAINT hardware_fk_vendor REFERENCES vendor(uid) DEFERRABLE,
    template_hardware bigint
);
\end{minted}


\subsubsection{Multi-value Reference}

Multi-value reference relation (see \secref{sec:relation-multi-value-references}) needs adding the foreign key
constraint complying with the following rules:

\begin{itemize}
    \item The target module table has to be present in the database already
    \item The referring column forming the relation has to be of the {\tt identifier\_set} type
    \item The name of the constraint shall start with the {\tt rset\_} prefix
    \item The foreign key shall always reference the {\tt uid} column of the target table
\end{itemize}

\begin{minted}{sql}
SET search_path TO production,deska;

CREATE SEQUENCE host_uid START 1;

CREATE TABLE host (
    -- this column is required in all plugins
    uid bigint DEFAULT nextval('host_uid')
        CONSTRAINT host_pk PRIMARY KEY,
    -- this column is required in all plugins
    name identifier
        CONSTRAINT host_name_unique UNIQUE NOT NULL,
    service identifier_set
        CONSTRAINT rset_host_fk_service REFERENCES service(uid) DEFERRABLE,
);
\end{minted}

\subsubsection{Object Composition}

The object composition (see \secref{sec:relation-contains}) needs adding the foreign key constrain. This constraint has
to comply with following rules:

\begin{itemize}
    \item The target module table has to be present in the database already
    \item The referring column forming the relation has to be of the {\tt bigint} type
    \item The name of the constraint shall start with the {\tt rconta\_} prefix
    \item The foreign key shall always reference the {\tt uid} column of the target table
\end{itemize}

In addition to requirements on module's foreign key, it is also required that:

\begin{itemize}
    \item This module and referenced module do not have any attribute with the same name besides {\tt name} and {\tt uid}
    \item The chain created by {\tt COMPOSITION} foreign keys of defined modules do not contain a cycle
\end{itemize}

\begin{minted}{sql}
SET search_path TO production,deska;

CREATE SEQUENCE host_uid START 1;

-- vendors of hw
CREATE TABLE host (
    -- this column is required in all plugins
    uid bigint DEFAULT nextval('host_uid')
        CONSTRAINT host_pk PRIMARY KEY,
    -- this column is required in all plugins
    name identifier
        CONSTRAINT host_name_unique UNIQUE NOT NULL,
    -- hardwere where it runs
    hardware bigint
        CONSTRAINT rconta_host_fk_hardware REFERENCES hardware(uid) DEFERRABLE,
);
\end{minted}

\section{Indexes}

Setting up indexes is a tricky business which cannot be solved by a few generic rules.  Under certain circumstances,
adding an ``obviously beneficial'' index to a database can in fact lead to a reduction of performance.  Care shall
therefore be taken before creating indexes and administrators shall always double-check that the real result with a
production data forms a positive gain.

The indexes created as a part of the scheme setup are maintained in the resulting database.  The indexes can be used by
the underlying database engine when not querying for any specified version or when the configuration generators are run.

\section{Accessing the Deska Database}

In order to access the Deska database, one has to have access to a properly authorized account on the host where the
Deska database runs.  After the basic account setup has been completed and the SSH key-based authentication works
flawlessly, this is how to add the Joe Random User to the list of people who can manipulate the contents of the Deska
database.  Please start the PostgreSQL console by the following command:

{\tt psql -U postgres -d deska\_database}

\ldots and perform these changes:

\begin{minted}{sql}
-- create user joe
CREATE USER joe;
-- then assign role deska_user
GRANT deska_user TO joe;
-- If you want joe to have access to the "restore" command in the CLI,
-- please also add the deska_admin role.  Kindly keep in mind that such
-- users can also manipulate the database directly, adding or dropping indexes
-- and generally causing a lot of trouble when not behaving responsibly:
GRANT deska_admin TO joe;
\end{minted}

After you have added proper database roles for your users, please activate the database scheme by following instructions
specified in \secref{sec:build-database-deployment}.

\end{document}
