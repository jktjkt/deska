\documentclass[12pt]{article}
\usepackage[utf8]{inputenc}
\usepackage[czech,english]{babel}
\usepackage{a4wide}
\author{Tomáš Hubík \and Lukáš Kerpl \and Jan Kundrát}
\title{Deska: Existing Solutions Analysis}


\begin{document}

\maketitle

\newpage

\tableofcontents

\newpage

\section{Quattor}

Quattor is an integrated system developed especially for use in the W-LCG
computing grid.  Therefore, it should provide the most obvious and persuading
benefits of all the mentioned platforms and systems.  Unfortunately, it has
grown to a rather complex apliance over time, and it is difficult nowadays to
use only the ``system inventory'' part of the project.  Because the Institute
does not want to abandon its existing systems, for example the Cfengine, using
Quattor is not an option in the long run, simply because isolating the required
parts would essentially lead to a fork of the project.  The Institute clearly
lacks the resources and commitment which would be required for maintaining such
a fork.


\section{Lemon}

Lemon is a full-blown monitoring system written and maintained by CERN.  Again,
a platform closely integrated with the grid should, in theory, provide clear
benefits.  However, Lemon has its disadvantages, like a design assumption that
it works on top of Oracle.  Personal experience with a large-scale deployment of
the system also suggests that performance issues are likely to emerge during
heavy use.  In addition to these problems, Lemon is attempting many problems
which are already being handled by the Institute's staffers, which would again
mean a need to manually extract the required parts.


\section{OCS Inventory}
Pozitive things:
\begin{itemize}
\item Licenced under the GNU General Public Licence 2.0
\item Good for hardware and software inventory management.
\item Supports PostgreSQL 7.2 or MySQL 4.1 db.
\item Configuration via Telnet, SSH, VNC, the Web, or Microsoft Terminal Services Client.
\item Written in C++.
\item Built-in backup/restore of OCS Inventory database
\end{itemize}


The bad ones:
\begin{itemize}
\item Does not support virual machines and their mapping to physical.
\item Does not have support for templates or some hiearchy.
\item Exporting information for generation of configuration could be difficult (unsufficient api). %maybe :)
\item Client installation on every computer required.
\item CLI not wery user friendly.
\end{itemize}




\section{Rackmonkey}
Pozitive things:
\begin{itemize}
\item Licenced under the GNU General Public Licence.
\item Good for reporting hardware.
\item Supports Postgres/MySQL/SQLite db.
\end{itemize}


The bad ones:
\begin{itemize}
\item Has only browser interface, no CLI.
\item Bad support for monitoring information about sw.
\item Written in Perl.
\end{itemize}



\section{Smurf}

Smurf, a database layer being used by the CC-IN2P3 computing facility in Lyon,
France, is certainly a promising project.  The author has spent considerable
time coordinating the effort with the Smurf's developers, and they have been a
valuable source of comments and information.  However, the Smurf is currently
undergoing a major rewrite during production use, and is therefore not usable in
its present form.


\end{document}
