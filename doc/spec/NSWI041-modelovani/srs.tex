\documentclass[12pt]{article}
\usepackage[utf8]{inputenc}
\usepackage[english]{babel}
\usepackage{a4wide}
\author{Tomáš Hubík}
\title{Deska}



\begin{document}

{\Huge \textbf{Deska}}

\vspace{0.2in}

{\large Tool for Central Administration of a Grid Site}

\vspace{0.5in}

{\large System Requirements Specification}

\vspace{0.2in}

{\large Prepared for the Institute of Physics of the AS CR}

\vspace{0.2in}

{\large Version 1.0}

\vspace{0.2in}

{\large Date 2009-Nov-02}

\vspace{0.5in}

\subsection*{Revision History}

\begin{table}[!h]
	\begin{tabular}{l l l l}
		\textbf{Date} & \textbf{Version} & \textbf{Author} & \textbf{Description} \\
		2009-Nov-02 & V1.0 & Tomáš Hubík & Document structure \\
	\end{tabular}
	\label{tab:RevisionHistory}
\end{table}


\subsection*{Document Review}

\begin{table}[!h]
	\begin{tabular}{l l l l}
		\textbf{Date} & \textbf{Version} & \textbf{Reviewer Name} & \textbf{Contact Info.} \\
	\end{tabular}
	\label{tab:DocumentReview}
\end{table}


\subsection*{Document Approval}

\begin{table}[!h]
	\begin{tabular}{l l l l}
		\textbf{Date} & \textbf{Version} & \textbf{Reviewer Name} & \textbf{Contact Info.} \\
	\end{tabular}
	\label{tab:DocumentApproval}
\end{table}


\newpage

\tableofcontents

\newpage

\section{Introduction}

\subsection{Purpose}
The purpose of this document is to define the specific requirements for the Tool for Central Administration of a Grid Site 
(henceforth, referred to as ``the System'') and to detail the specifications for the features, capabilities, critical attributes, 
and major characteristics of the proposed system. It is intended to be read by management, and administrators of Institute of 
Physics of the AS CR for the purposes of evaluating the benefits and feasibility of the proposed application as well as to provide 
a basis for the estimation of the time and effort necessary to construct, test, deploy, and maintain it. This document does not 
describe how, when, or where any of these activities will be performed or who will do them.

\subsection{Scope}
The Deska will be responsible for managing and monitoring of a grid site. Especially it automates configuration of several monitoring 
and deployment tools. All functions and features will be displayed and managed through client--server environment. TODO\\
For more details on the scope of this project see the Scope document.

\subsection{System Context}
There are three main ``touch points'' of the Hotel Reservation System: the central DBMS for configurations and history storage, the server, 
that will authorize clients and will generate configurations for single tools and clients, that will conect to the server to make some 
changes in the central configuration. TODO

\subsection{Primary Stakeholders}
The following is a list client stakeholders for different areas within the project. Each area can have many reference stakeholders 
who should be consulted for requirements information gathering. Each area also has one primary stakeholder who, among all reference 
stakeholders, resolves disagreement and has final approval for requirements in that area. TODO

\begin{table}[!h]
	\begin{tabular}{| l | l | l |}
		\hline
		\textbf{Area}		& \textbf{Primary Stakeholder} & \textbf{Reference Stakeholders}\\
		\hline
		Administrators	& Somebody1										 & Somebody1\\
										&															 & Somebody2\\
										&															 & Somebody3\\
		\hline
		Management			& Somebody4										 & Somebody4\\
		\hline
	\end{tabular}
	\label{tab:PrimaryStakeholders}
\end{table}

\subsection{Acronyms and Abbreviations}

TODO

\begin{table}[!h]
	\begin{tabular}{| l | l |}
		\hline
		\textbf{Acronym / Abbreviation}		& \textbf{Expanded Term}\\
		\hline
		DBMS	& Database Management System\\
		\hline
		ECO		&	Engineering Change Order\\
		\hline
		FR		&	Functional requirement\\
		\hline
		NFR		&	Non-functional requirement\\
		\hline
		OS		& Operating System\\
		\hline
		SRS		& System Requirements Specification\\
		\hline
		UC		&	Use Case\\
		\hline
	\end{tabular}
	\label{tab:AcronymsAndAbbreviations}
\end{table}

\subsection{How This Document is Organized}

The following sections provide all known system requirements, including both functional and nonfunctional requirements. 
This document is complete except where noted with reference to an external source. It is helpful but not necessary to read 
the sections in a sequential order. Section 2 describes the project constraints and assumptions. Section 3 describes the 
project risks and how these will be mitigated. Section 4 describes the functional requirements (FRs) of the System. Most 
functional requirements exist to directly support business process; some exist to support the correct operation of the system 
itself. All functional requirements are described in terms of use cases. Section 5 describes the non-functional requirements 
(NFRs) of the System. Section 6 provides a Project Glossary that includes project-related terms as well as software 
developmentrelated terms.

\subsection{Engineering Change Orders}

TODO

[This section describes whether this document will be updated according to agreed Engineering Change Orders either continuously 
during the project, at the end, or not at all will be up to each project to determine. In projects where time is too short to 
update the requirements, the ECOs could simply be appended to the requirements.]

\subsection{References}

\begin{enumerate}
	\item Tool for Central Administration of a Grid Site Scope document
\end{enumerate}


\section{Constraints and Assumptions}
The following sections provide additional detail on the matching sections in the Tool for Central Administration of a Grid Site 
Scope document.

\subsection{Development Process and Team Constraints}
At this time, there are no known constraints.

\subsection{Environmental and Technology Constraints}

\subsubsection{Software Constraints}
At this time, there are no known constraints.

\subsubsection{Hardware Contraints}
At this time, there are no known constraints. TODO

\subsection{Delivery and Deployment Constraints}
The application will be hosted on some machine with UNIX type operating system. The web part of the system will be browser 
independent.\\
FZU wants to have a possibility to influence the essential parts of design of the system, so we have to discuss all important issues
with administrators of FZU.


\section{Risk Mitigation}

\subsection{Technological Risks}
TODO

\subsection{Skills and Resources Risks}
TODO

\subsection{Requirements Risks}
TODO

\subsection{Political Risks}
There are no political risks for this project.


\section{Functional Requirements}
This section defines the actors who use this system while supporting the targeted business processes, as well as the use cases this 
system provides to those actors.

\subsection{Primary Functional Requirements}
In this section, we classify the primary features of the Hotel Reservation System across three categories. This list is an updated 
version of the same list from the Scope document. Some FRs have changed priority and a few new ones have been added.\\
Essential features cannot be done without. High-value features can be done without, although it may be very undesirable to 
do so. Follow-on features are those for which it is not clear they should be included in the first release. In all cases, 
the lists are not exhaustive but include the most important features from a business perspective.

\subsubsection{Essential Features}
\begin{itemize}
	\item Central storage of authoritative information about the whole grid site
	\item Generating configuration files for various tools. For example Nagios, Cfengine
	\item Unified API for creating other modules
\end{itemize}

\subsubsection{High-Value Features}
\begin{itemize}
	\item Data aggregation from more tools to one well-arranged place, one place where to look for problems
\end{itemize}

\subsubsection{Follow-on Features}
\begin{itemize}
	\item Data correlation about load from various sources to determine if there is some dead lock, or hung application
	\item Implementation of modules for other tools, than those, that are currently used on FZU
\end{itemize}

\subsection{Actors}
These are the roles of persons and systems that interact with the System.

\begin{table}[!h]
	\begin{tabular}{| l | l |}
		\hline
		\textbf{Actor name}		& \textbf{Description}\\
		\hline
		SomeName	& Description\\
		\hline
	\end{tabular}
	\label{tab:Actors}
\end{table}

\subsubsection{Actor: SomeName}
TODO

\subsection{Use Cases}

TODO

\begin{table}[!h]
	\begin{tabular}{| l | l | l | l |}
		\hline
		\textbf{Use Case Name} & \textbf{Priority} & \textbf{Number} & \textbf{Description}\\
		\hline
		SomeCase	& E-H & 1 & something\\
		\hline
	\end{tabular}
	\label{tab:UseCases}
\end{table}

\subsection{Applications}

TODO

\begin{table}[!h]
	\begin{tabular}{| l | l |}
		\hline
		\textbf{Application Name} & \textbf{Description}\\ \cline{2-1}
															& \textbf{UseCases}\\
		\hline
		SomeApp	& something\\ \cline{2-1}
						& Supports UCs: E1\\
		\hline
	\end{tabular}
	\label{tab:Applications}
\end{table}

\subsection{Use Case Detailed Requirements}

\subsubsection{SomeApp Requirements}
This section lists all of the detailed requirements for the SomeApp.
TODO

\begin{table}[!h]
	\begin{tabular}{| l | l |}
	 	\hline
		\textbf{Req. Code} & \textbf{Requirement Description}\\
		\hline
		E1-1	& something\\
		\hline
	\end{tabular}
	\label{tab:SomeAppRequirements}
\end{table}


\section{Non-Functional Requirements}
Note: non-functional requirements are distinguished from FRs by starting each requirement number above 100. Therefore, the first 
NFR for Use Case \#E1 would be given the code E1-101.

\subsection{Performance}
This System will not require significant demands on the hardware.

\subsubsection{Current Release}

\begin{table}[!h]
	\begin{tabular}{| l | l |}
		\hline
		\textbf{Req. Code} & \textbf{Requirement Description}\\
		\hline
		E1-101	& something\\
		\hline
	\end{tabular}
	\label{tab:PerformanceRequirements}
\end{table}

\subsubsection{Future Releases}
No additional performance requirements for future releases.


\subsection{Scalability}

\subsubsection{Current Release}

\begin{table}[!h]
	\begin{tabular}{| l | l |}
		\hline
		\textbf{Req. Code} & \textbf{Requirement Description}\\
		\hline
		E1-105	& something\\
		\hline
	\end{tabular}
	\label{tab:ScalabilityRequirements}
\end{table}

\subsubsection{Future Releases}
No additional scalability requirements for future releases.


\subsection{Availability}

\subsubsection{Current Release}

\begin{table}[!h]
	\begin{tabular}{| l | l |}
		\hline
		\textbf{Req. Code} & \textbf{Requirement Description}\\
		\hline
		E1-108	& something\\
		\hline
	\end{tabular}
	\label{tab:AvailabilityRequirements}
\end{table}

\subsubsection{Future Releases}
No additional availability requirements for future releases.


\subsection{Reliability}

\subsubsection{Current Release}

\begin{table}[!h]
	\begin{tabular}{| l | l|}
		\hline
		\textbf{Req. Code} & \textbf{Requirement Description}\\
		\hline
		E1-110	& something\\
		\hline
	\end{tabular}
	\label{tab:ReliabilityRequirements}
\end{table}

\subsubsection{Future Releases}
No additional reliability requirements for future releases.


\subsection{Security}

\subsubsection{Current Release}

\begin{table}[!h]
	\begin{tabular}{| l | l |}
		\hline
		\textbf{Req. Code} & \textbf{Requirement Description}\\
		\hline
		E1-113	& something\\
		\hline
	\end{tabular}
	\label{tab:SecurityRequirements}
\end{table}

\subsubsection{Future Releases}
No additional security requirements for future releases.


\subsection{Manageability}

\subsubsection{Current Release}

\begin{table}[!h]
	\begin{tabular}{| l | l |}
		\hline
		\textbf{Req. Code} & \textbf{Requirement Description}\\
		\hline
		E1-115	& something\\
		\hline
	\end{tabular}
	\label{tab:ManageabilityRequirements}
\end{table}

\subsubsection{Future Releases}
No additional manageability requirements for future releases.


\subsection{Usability}

\subsubsection{Current Release}

\begin{table}[!h]
	\begin{tabular}{| l | l |}
		\hline
		\textbf{Req. Code} & \textbf{Requirement Description}\\
		\hline
		E1-116	& something\\
		\hline
	\end{tabular}
	\label{tab:UsabilityRequirements}
\end{table}

\subsubsection{Future Releases}
No additional manageability requirements for future releases.


\subsection{Maintainability}

[Define the life expectancy of the application, and describe the requirements that will facilitate maintenance throughout its 
lifecycle, including use of specific languages, compilers, tools, libraries, coding conventions, or methodologies that will 
facilitate debugging and enhancement by those engineers who will be responsible for long-term maintenance. Define specific 
requirements for source code headers that might include module name, author's name, date of creation, purpose of module, placeholders 
for SCM, and revision history. There may also be specific requirements for file naming, directory organization, make files, development 
environment, and such that will facilitate maintenance.]\\

[Describe the different levels of application problems that your IT department tracks and the acceptable number of undetected and/or 
unresolved errors at first release. Discuss ``bug inflation'' and how bugs will be assigned a severity.]\\

Typical categories of system problems are listed below:
\begin{itemize}
	\item Severity 1, Fatal - Entire system is affected and cannot be used at all.
	\item Severity 2, Major - Part of the system or critical functionality is affected and an acceptable workaround is not possible.
	\item Severity 3, Medium - A functional problem requires increased effort to avoid via a temporary workaround. The problem cannot be indefinitely deferred.
	\item Severity 4, Minor - Functioning of system is not significantly impaired and users can live with the problem for now.
	\item Severity 5, Enhancement - A change or addition to the requirements is desired to address an unanticipated deficiency.
\end{itemize}


\subsection{Extensibility}

\begin{table}[!h]
	\begin{tabular}{| l | l |}
		\hline
		\textbf{Req. Code} & \textbf{Requirement Description}\\
		\hline
		E1-118	& something\\
		\hline
	\end{tabular}
	\label{tab:ExtensibilityRequirements}
\end{table}


\section{Project Glossary}
[Define all terms referenced in any of the scope or requirements documents. This can include both system-related terms and software 
development terms. For clarity you might choose to separate these two sets of terms into different lists. The Project Glossary can 
also be a separate document to reduce the size of the SRS.]

\end{document}