% vim: spelllang=en spell textwidth=120
\documentclass[deska]{subfiles}
\begin{document}

\chapter{The DeskaDb C++ Library}

\begin{abstract}
This chapter describes the overall architecture of the {\tt DeskaDb}, a C++ library which wraps access to a Deska
database.
\end{abstract}

\section{Usage}

The main feature provided by the DeskaDb library is implemented by the {\tt Deska::Db::Connection} class.  When
instantiated, it provides access to the full set of functions mandated by the Deska DBAPI (see \secref{sec:api-ref}).
Please consult the appropriate reference documentation for semantics of these functions and a general overview of the
Deska DBAPI protocol.

\section{Internal Architecture}

The {\tt Deska::Db::Connection} employs a number of classes under the hood.  Their functions are described in this
section.

At the very lowest level lies the {\tt Deska::Db::JsonExtractionTraits} policy class which defines proper methods of
JSON serialization and deserialization of individual data types, from strings, numbers and Deska identifiers to more
complicated, compound types like IP addresses, timestamps and Deska server-side exceptions.  Each specialization of this
class shall provides two methods, the {\tt toJson()} and {\tt fromJson()}, which are supposed to properly convert back
and forth between the native C++ classes and their associated JSON representation.  These methods shall simply return
the appropriate values, except for those that implement conversion of server-side errors to C++ exceptions, which are
immediately thrown due to the limitations of C++ polymorphism.

The functionality provided by these low-level policy classes is typically used through the {\tt
Deska::Db::JsonExtractor} class, which is in turn employed by the {\tt Deska::Db::JsonHandler} and its specialization
{\tt Deska::Db::JsonHandlerApiWrapper}.  Together, these classes provide a convenient interface where the upper layers
simply declare what kind of objects and attributes, along with their data types, are expected at a particular context,
and the system automatically processes each of them.  The responsibilities are sanely divided; the {\tt JsonHandler}
encapsulates basic attribute access, the {\tt JsonHandlerApiWrapper} adds a few methods which are usable for a streaming
DBAPI-like interface, and the {\tt JsonExtractor} takes care of storing the individual attribute values in a type-safe
manner, using the {\tt JsonField} for the underlying storage.

The core of the DBAPI protocol is implemented through the {\tt JsonApiParser}, a class which performs translation
between the DBAPI function calls and the associated compound JSON structures.  This class is further augmented by
various helper classes like the {\tt JsonWrappedAttributeMap}, {\tt JsonWrappedAttributeMapList}, {\tt
JsonWrappedAttributeMapWithOrigin}, {\tt JsonWrappedAttributeMapWithOriginList}, {\tt JsonWrappedAttribute}, {\tt
JsonWrappedAttributeWithOrigin}, {\tt JsonWrappedObjectModification} and the {\tt
JsonWrappedObjectModificationResultSequence}, which enforce high-level constraints of the DBAPI protocol, like the
correct data type of the Deska object attributes.\footnote{This is different from the JSON-level data types, which simply
enforce the syntax validity of the JSON stream.  These classes, however, work with the dynamically determined database
metastructure, and verifies that the remote server returns valid values for functions which look at the Deska object
data.}

To allow this dynamic checking of high-level constraints, the {\tt CachingJsonApi} class adds metadata caching to the
picture.  This addition is necessary to prevent illegal loops of metadata requests going on between the Deska library
and the remote server.

The {\tt Deska\_p} class adds a streaming IO support on top of the {\tt CachingJsonApi}, and is finally wrapped in the
{\tt Deska::Db::Connection} class which strives to provide a stable ABI on top of C++ objects (which is, one might add,
a futile deal for non-leaf classes with virtual methods).

\subsection{Working with JSON}

The Deska DBAPI is based on a continuous exchange of JSON objects.  The list of attributes expected for an individual
object is always defined by the context in which an object appears.  This leads to the general structure of the JSON
parsing code.

Basically, each function defines a tree of objects which are expected as a return value, and at the same time builds a
JSON structure to be sent to the server.  All of that is accomplished through the {\tt JsonHandler} class, as in this
snippet:

\begin{minted}{c++}
std::vector<ObjectRelation>
JsonApiParser::kindRelations(const Identifier &kindName) const
{
    // Setup a context for the libebt library.  This is used to provide detailed
    // error messages which include a detailed backtrace of the context in which
    // an error has occurred.
    JsonCommandContext c1("kindRelations");

    // Allocate a space for the return value of this function
    std::vector<ObjectRelation> res;

    // The JsonHandlerApiWrapper class is a bit special; it adds the knowledge of
    // the Deska DBAPI protocol where the data we are sending include the
    // ``command'' field, while the expected output contains the ``response''
    // field.  In addition, server-side errors are transparently dealt with and
    // converted to exceptions.
    JsonHandlerApiWrapper h(this, "kindRelations");

    // The data we are sending include the ``kindName'' attribute
    h.argument(j_kindName, kindName);

    // We expect to receive the ``kindRelations'' data. It is an error if this
    // field is missing from the output.
    h.read("kindRelations")
    // ...and the received value is automatically stored to the ``res'' variable.
    // Note that its data type is automatically deduced and enforced.
        .extract(&res);

    // Call to the work() method builds a JSON object from the data we have put
    // together, sends it through the IO socket, waits for a complete response,
    // parses it from JSON, checks all the required fields for presence and their
    // syntax validity and finally extracts the data to the output variable, the
    // ``res''.
    h.work();

    // Everything is done at this point.
    return res;
}
\end{minted}

\subsection{Extending Deska DBAPI}

Instructions for adding new data types to the Deska DBAPI protocol are available in \secref{sec:json-extending}.
Basically, adding new data types involves providing appropriate specialization of the {\tt JsonExtractionTraits} policy
class, instantiating a few templates and adding the unit tests.

\subsubsection{Converting between native types and JSON}

Here is an example of how a typical conversion looks like:

\begin{minted}{c++}
/** Convert a MAC address to and from its JSON representation */
template<> struct JsonConversionTraits<Deska::Db::MacAddress> {
    // This function converts a JSON string into a MAC address.
    // In this instance, we have settled on plain string as the
    // JSON serialization format.  Other classes, especially those
    // with many individual member variables, might be better
    // represented as JSON objects.
    static Deska::Db::MacAddress extract(const json_spirit::Value &v) {
        // Setup a proper context for error reporting
        JsonContext c1("When extracting Deska::Db::MacAddress");
        // Make sure that the value we work with is a string
        checkJsonValueType(v, json_spirit::str_type);
        try {
            // Construct a MAC address from the string representation
            // and throw an exception if it fails
            return Deska::Db::MacAddress(v.get_str());
        } catch (std::domain_error &e) {
            // We were unable to construct a native class.  Let's convert
            // the error into an exception which is JSON-related and throw
            // that.
            throw JsonStructureError(e.what());
        }
    }

    // Convert a native C++ object into JSON
    static json_spirit::Value toJson(const Deska::Db::MacAddress &value) {
        // In this case, we can simply use the existing iostream support
        // of the MacAddress class.
        std::ostringstream ss;
        ss << value;
        return ss.str();
    }
};
\end{minted}

In this example, we were lucky enough to be able to use the textual representation of an object value directly in JSON.
In other cases, this shortcut might not be desirable, and you would have to resort to using JSON objects instead.  The
usual JSON helper classes might come handy in these cases, as shown in the following example which shows how the Deska
code reads records about committed revisions:

\begin{minted}{c++}
/** Extract RevisionMetadata from JSON */
template<>
RevisionMetadata JsonConversionTraits<RevisionMetadata>::extract(const json_spirit::Value &v) {
    // We start with setting up a context for error reporting.  In a language
    // like C++, it's better done by hand due to lack of a runtime introspection.
    JsonContext c1("When converting a JSON Value into a Deska::Db::RevisionMetadata");
    // The class for extracting data...
    JsonHandler h;
    // ...the result and a few auxiliary variables...
    RevisionId revision(0);
    std::string author;
    boost::posix_time::ptime timestamp;
    std::string commitMessage;
    // At this time, we have to set up extracting of the individual fields.
    // No real access takes place at this time; we simply set up a proper
    // structure.
    h.read("revision").extract(&revision);
    h.read("author").extract(&author);
    h.read("timestamp").extract(&timestamp);
    h.read("commitMessage").extract(&commitMessage);
    // Make sure it's an JSON object...
    checkJsonValueType(v, json_spirit::obj_type);
    // ...and extract it now.
    h.parseJsonObject(v.get_obj());
    return RevisionMetadata(revision, author, timestamp, commitMessage);
}
\end{minted}

This piece of a real-world code demonstrates that the Deska-provided JSON helpers are very flexible and well-suited for
converting JSON objects to native C++ classes.

\section{IO Support}

The {\tt DeskaDb} library delegates its IO interaction to a dedicates class, an implementation of the {\tt IOSocket}
interface.  Two distinct implementations are provided, the {\tt ProcessIO}, which launches a child process and
communicates with it through the stdin/stdout, and the {\tt UnixFdIO}, which uses a pair of Unix pipes, usually
inherited from its parent process, to perform the communication.  The latter is required for proper support for
configuration generators, which are described in \secref{sec:config-generators}.

\subsection{Stream Copying}

The {\tt ProcessIO} class uses an extension to the Boost library, the {\tt boost::process} class.  However, we have
encountered quite a few issues when trying to use a completely stream-based interface for JSON parsing.  In short, the
JSON parser already has support for reading a sequence of characters from a {\tt std::istream}, but if used in this
mode, the error conditions reported back in case of a syntax error do not contain a directly usable context.  This is
why we had to add support for copying the data read from the stream so far to a buffer for possible error reporting.

Due to the design of the iostreams library, we had to work on rather low layers and augment the streambuf
implementation.  What we have done at this layer was extending the {\tt boost::process::detail::systembuf} with a pair
of Boost signals for proper event signalling, where we emit events as soon as data are read from the underlying socket.
While strictly not needed for Deska, we have also added a symmetrical support for the same signaling of low-level
writes.

This code is available in the forked {\tt src/3rd-party/process/boost/process/detail/systembuf.hpp} file.

\subsection{Unit Tests and IO}

Writing a proper unit tests for the IO parsing routines is also far from straightforward, mainly to the complex nature
of the IO interaction patterns.  It is trivial to use a string stream as an input of the data, but when writing a test
case for a complex interaction which involves both reading and writing, and making sure that the operations are executed
in order, a simple string-based approach fails.  This is when the {\tt MockStreamBuffer} comes into play.

The {\tt MockStreamBuffer} class, as defined in file {\tt tests/MockStreamBuffer.h}, provides a full implementation of
the usual iostream's streambuf interface.  Before using the associated stream, the programmer shall execute calls to
{\tt expectRead}, {\tt expectWrite} etc. functions to record what kind of access the stream shall support.  If the user
of the stream subsequently deviates from the expected access pattern, an exception will be thrown.\footnote{The exception
indicating an unexpected operation is thrown immediately when the {\tt streambuf} detects an error.  Due to the nature
of iostreams, this might not be the place where the deviation actually comes from.  To minimize this risk of
inconstistency and in order to be as deterministic as possible, the {\tt MockStreamBuffer}'s size shall be set to {\tt
1} at the time of its construction.}  Please also note to {\tt flush} the stream in order to force the actual data to
reach the streambuf:

\begin{minted}{c++}
/** Test a simple RW interaction */
BOOST_FIXTURE_TEST_CASE(simple_rw, MockStreamFixtureNoThrow)
{
    // At first, set up our expectations.  We start with writing a few items.
    // The fact that there are two calls to expectWrite() is not relevant.
    buf.expectWrite("foo bar baz");
    buf.expectWrite("pwn");
    buf.expectRead("abc "); // note the trailing space
    buf.expectReadEof();
    os << "foo " << "bar " << "baz" << std::flush;
    os << "pwn" << std::flush;

    std::string tmp;
    is >> tmp;
    BOOST_CHECK_EQUAL(tmp, "abc");
    BOOST_CHECK(!is.fail());
    BOOST_CHECK(!is.eof());
    // Because of the trailing space and the rules for reading into strings,
    // we are not at the end of the stream just yet.
    // We will be after the next read, though:
    is >> tmp;
    // There was just a space, and hence no real string to be read -> failure.
    BOOST_CHECK(is.fail());
    BOOST_CHECK(is.eof());
    BOOST_CHECK(buf.consumedEverything());
    BOOST_REQUIRE(!(is.bad() || os.bad() || thrown));
}
\end{minted}

The {\tt MockStreamBuffer} class is heavily used in this manner throughout the unit tests to enforce meaningful
operation of the JSON interface.

\end{document}
