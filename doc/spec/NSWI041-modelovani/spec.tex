\documentclass{article}
\usepackage[utf8]{inputenc}
\usepackage[czech,english]{babel}
\usepackage{a4wide}

\begin{document}

\title{Deska: A Tool for Central Administration and Monitoring of a Grid Site}

\author{Jan Kundrát \and Martina Krejčová \and Tomáš Hubík \and Lukáš Kerpl}

\maketitle

\section{Background}

One of the activities taking place at the Institute of Physics of the~AS~CR (FZU)~\cite{fzu} is running a local Tier-2 computing
center connected to an international WLCG computing grid~\cite{wlcg}, supporting user programs from various scientific communities
from the whole world including the D-0 experiment in Fermilab~\cite{d0} and the famous LHC accelerator from CERN~\cite{lhc}.

\section{Motivation}

In February 2010, the total installed computing capacity available to users was about 3000~CPU cores with more than 500~TB of disk
storage, which translates to several hundreds of physical machines.  All the resources are managed by just a few system
administrators (currently slightly more than three full time employees).  Due to the impressive machines-per-administrator ratio,
the staffers are exploring ways about how to make the system administrators' life easier and less error prone.

Over the years, certain monitoring and management systems were deployed at the Institute, including Cfengine~\cite{cfengine},
Nagios~\cite{nagios}, Ganglia~\cite{ganglia}, Munin~\cite{munin} and various in-house monitoring appliances.  While these tools
certainly aim to make the sysadmin's life easier, introduction of each new tool also adds one more place for the administrator to
check if she wants to make sure the infrastructure works as expected, and each new tool has to be configured separately.  In
addition, each site in the WLCG Grid is also being monitored and checked by third parties.\footnote{Monitoring from outside
provides a highly needed perspective about site's availability from throughout the Grid.  Such external tools come in variety of
flavours, from simple tools checking whether certain values published in a global information system are sane to complex solutions
like automated job-submitting robots which monitor whether the site in question is in fact accepting jobs and if these jobs
terminate successfully in a timely manner.  Jobs running on site's worker nodes also have the ability to check for various local
settings like having fresh enough packages and fixed security vulnerabilities.}  Number of systems monitoring a typical site from
outside grows each year, introducing yet another burden on the Institute's staffers.  Therefore, it has been decided that the
Institute shall fund a development of a tool which will:

\begin{itemize}
    \item{automate the configuration process,}
    \item{aggregate the monitoring data at one place}
\end{itemize}

\section{System Requirements}

\subsection{Database}

The aim of the first part of the project is to develop a database storing generic description of all the resources that the Grid
site is using; such a database should be capable of describing hardware infrastructure of the data center\footnote{That is all
physical machines, their hardware models, part numbers, rack locations, network interconnects, history of hardware failures etc},
couplings of operating system instances (aka ``hosts'') to physical machines, logical relations between various
services\footnote{For example, each {\em subcluster} has one or more {\em gateways} called Computing Elements.  Failure of the CE,
as detected by upstream monitoring, might be actually caused by a dying hard drive on one single node in the cluster.  The system
must therefore be capable of storing such complex information.} as well as a history of issues encountered by each component.
This database shall then act as the single authoritative source of information for generating configuration of all components of
the system, including the monitoring appliances.  In addition, tools shall be written to perform consistency checks between the
state recorded in the database and the actual system shape when the automatic configuration generation is not feasible for safety
reasons\footnote{A classic example of such a system are network switches -- system administrators are rather leery when it comes
to automated configuration as a single error can cause large network segments to separate.}.

Due to the Institute's administrators' strong Unix background and on their explicit request, it is expected that the interface to
this database will be provided via a CLI application instead of using a traditional web-based approach.  The main reason for this
design choice is their familiarity with text-based configuration tools and bad experience with web applications in general.  The
application interface shall be similar to a CLI-oriented interface used on enterprise-grade Ethernet switches.  The format shall
enable storing of plaintext dumps of the database in a version control system.

The Database shall provide an API for read-only access to the data contained therein.  Modifications shall be performed only
through the developed user interface and not through the API for security reasons.  The whole system shall be able to scale up to
thousands of nodes on a commodity x86\_64 server hardware.

Furthermore, additional components for generating configuration on basis of actual system state, as retrieved through the
read-only database API, shall be developed for all systems which are currently deployed at FZU's site, namely for Nagios, BIND
name service demon, DHCP server and Ganglia and Munin master servers.  The design of the API should reflect generic needs of any
data center to allow eventual deployment of new components, should the need arise.  The API will also be used as a source of
information for the second part of the project, the Dashboard -- the information available through the API will make it possible
to generate an accurate visualization of the data center and all deployed systems, from the physical view showing racks of servers
to a high-level view depicting logical dependencies between services and machines.


\subsection{Dashboard}

The second part of the system is the Dashboard, a tool whose purpose is to provide an accurate and up-to-date overview of the
state of the system.  The main philosophy behind this component is that there are already plenty of monitoring systems checking
the health and performance metrics of a grid site, yet no central place for the administrators to look for troubles.  Therefore,
it is pretty common that a minor issue which is detected by only a single monitoring system gets unnoticed by the system
administrators for several months.  The goal of the dashboard is, therefore, to provide a central view aggregating data from {\em
all} the monitoring systems, both local and remote.

The system shall be capable of storing the data provided by each monitoring tool, performing a mapping from their naming
convention to an intermediate form, and grouping related sensors together.  Exploring ways in which to query various monitoring
systems for data is part of the design phase of the system.  The data shall be stored in format which makes it possible to view
history of each value up to a configurable age.  An example of such a backing store is the Round Robin Database (RRD)~\cite{rrd}.

The dashboard will aggregate and keep an archive of data coming from all the currently deployed monitoring systems.  Examples of
{\em local} appliances whose results should be aggregated are Nagios, Munin and Ganglia (for monitoring metrics like CPU
utilization, available memory, local disk space, network response times etc.), network infrastructure through SNMP\footnote{The
network monitoring is actually a prime example of the usefulness of the Database and Dashboard integration -- having an
up-to-date, reliable mapping between a hostname and a particular port on some switch is crucial for this task and will enable
accurate and reliable visualization of network utilization}, environment data like air temperature, CRAC units power or UPS
utilization and also values like number of jobs in the queue or number of currently running jobs -- in short, the data sources are
generally speaking deployed and managed by the Institute's own staffers and could thus be slightly modified, both in terms of
configuration and output format, for interoperability purposes.  In contrast, the {\em remote} systems are typically more
complicated partially autonomous systems developed and managed by third parties, and usually provide an executive-oriented
visualization of a particular site's performance.  A typical example of such a tool are the SAM tests\footnote{SAM tests are a
complex framework submitting jobs to sites in a fully automatic way with additional autonomous agents checking that the test jobs
actually hit the target worker node.  The executed jobs themselves then perform variety of tests right on the site's computational
resources, from checking particular software packages' versions to initiating file transfers in order to test the Grid disk
servers.  This approach proved to be extremely valuable and successful at catching many errors.} or various web-based monitors
like GStat~\cite{gstat}.

A web portal shall be developed depicting the health and the present state of a system in an intuitive way.  In addition to that,
it shall be possible for the user of the dashboard to view graphs showing the history of various sensors, both in isolation and
overlaid in order to provide a way of figuring out correlations between different metrics -- for example, system administrators
might be interested in figuring out if the network utilisation and the number of jobs currently running in the job system matches
in order to provide an infrastructure with balanced disk and CPU throughput.  All the data shall be exportable to a format which
makes it easy to explore the data using a third-party statistical tool, but an elementary form of statistical analysis shall be
integrated into the core system.

Furthermore, due to a strong pressure to deploy Nagios as {\em the} monitoring solutions in the Grid, Nagios shall indeed be an
integral component of the design\footnote{For example, it shall be used for gathering data where possible, and detected alerts
shall be pushed back to Nagios for evaluation and passing them to the administrators}.

\section{Problems to Solve}

Monitoring hundreds or thousand of machines is a difficult task.  Problems that the team will have to solve are therefore mostly
related to scalability and efficiency in using limited resources.  The RRDTool toolkit puts high load to the disk subsystem with
its constant scattered writes.  The team will have to investigate current approaches and come up with a solution which allows
collecting data often enough that it is still usable, yet does not overload the monitoring machine with too many IO requests.
Care has to be taken to handle hosts suddenly not responding in a graceful way, so that they do not block the program execution,
and to fault tolerance in general.

Another expected issue is in developing pluggable modules for querying third-party monitoring systems.  It is to be explored if
they provide a reasonable API or if the team will have to resort to suboptimal means like parsing the HTML pages with overview.

The structure of the Database shall be thoroughly planned, too.  The team will have to balance the complexity needed for accurate
representation of the real infrastructure with the required level of usability.  The interface should be intuitive enough so that
the administrator will actually notice the improvements and start to like the system in favor of manual hacks.

\section{Future Improvements}

The system shall be extensible enough to be used outside of the Institute's computing infrastructure.  Areas for further
improvements include providing a plugin for analysis of the batch system controller, tracking data transfers from the disk
servers, improved visualization etc.  A nice feature would be also a ``one-click'' tool for fully automated deployment of new
machines, handling everything from setting up PXE environment for installation to actual system configuration.


\begin{thebibliography}{9}
    \bibitem{fzu}Institute of Physics of the AS CR home page, {\tt http://www.fzu.cz/}
    \bibitem{wlcg}The W-LCG LHC Computing Grid, {\tt http://lcg.web.cern.ch/LCG/}
    \bibitem{d0}The DZero Experiment, {\tt http://www-d0.fnal.gov/}
    \bibitem{lhc}The Large Hadron Collider, {\tt http://lhc.web.cern.ch/lhc/}
    \bibitem{cfengine}Cfengine, {\tt http://www.cfengine.org/}
    \bibitem{nagios}Nagios, {\tt http://www.nagios.org/}
    \bibitem{ganglia}Ganglia, {\tt http://ganglia.sourceforge.net/}
    \bibitem{munin}Munin, {\tt http://munin.projects.linpro.no/}
    \bibitem{rrd}RRDTool, {\tt http://oss.oetiker.ch/rrdtool/}
    \bibitem{gstat}GStat, {\tt http://goc.grid.sinica.edu.tw/gstat/}
\end{thebibliography}

\end{document}
