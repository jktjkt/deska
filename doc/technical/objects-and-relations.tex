% vim: spelllang=en spell textwidth=120
\documentclass[deska]{subfiles}
\begin{document}

\chapter{Objects and Relations}
\label{sec:chap-objects-and-relations}

\begin{abstract}
The Deska database can store data of varying structure.  In this chapter, we provide an overview of what can be stored
and how to use the provided facilities to design a database scheme which closely matches the real-world processes.  The
basic building blocks of the Deska database, the {\em kinds}, {\em objects} and {\em relations}, are described here,
explaining the fundamental Deska principles to the system administrators.
\end{abstract}

\section{Database Features}

The entire Deska system is built around a database storing structured data.  Users deploying this system are free to
adapt the default database scheme to their needs, or indeed come up with completely revised design, provided they obey a
few rules.  A much more in-depth explanation of these rules and principles behind them, mainly targeted at the system
administrators, is provided in~\secref{sec:objects-and-relations}, so readers are encouraged not to give up if they
immediately feel overwhelmed by this quick tour of Deska's features.

\subsection{Kinds, Objects and Attributes}

The database is built around an object-oriented data store.  The object classes (known as the {\tt kinds} in Deska for
various reasons) are defined using standard SQL data definition language features, and enforce the necessary structure
of the data.  Individual {\em objects}, instances of one of these classes and rows in a database table, then store
individual data in their {\em attributes} (table columns), named key-value pairs whose structure and type is defined by
the corresponding kind.

The kinds place no limits on the number of attributes that each object might have, and support a wide range of low-level
data types (see \secref{sec:json-data-types-reference} for a full list).  Almost arbitrary integrity constraints and
triggers can be defined, allowing for further freedom in designing the scheme, as long as they do not interfere with the
rest of the Deska system.

Objects (rows in the database) can be retrieved by their {\em name} (a primary key\footnote{The {\tt name} itself is not
directly utilized as a primary key for performance reasons.  Their functions are, however, equivalent at this level.})
or through an indirect query based on values of object attributes (see \secref{sec:api-filters} for supported
operations).  They cannot be altered through direct SQL data manipulation calls, but each access has
to be performed through a custom RPC-like interface (\secref{sec:api-group-data-modification}).  This interface
implements fully-fledged version control for all objects, preventing accidental modifications and maintaining a detailed
audit trail of any activity.  The historical data can be retrieved (\secref{sec:api-group-history}) at any time.

\subsection{Relations: Providing Structure to the Data}

The {\em relations} describe further associations --- or connections --- between the individual objects.  Modelled
through foreign keys, relations are used to provide a wide range of features, from simple referrals to implementing
tree-shaped structures on top of flat object lists.

The simplest relation is just an ordinary foreign key with no special meaning.  It can be used as a simple pointer for
associating an object with another one, like the {\tt vendor} attribute of a {\tt machine}.  This relation, the {\tt
REFERS\_TO}, is described in detail in~\secref{sec:relation-refers-to}, and does not need any special handling, as any
qualifying foreign key which is not recognized as a specific relation is interpreted as a {\tt REFERS\_TO} one.

More complicated example of a relation is the {\tt CONTAINS} and {\tt CONTAINABLE} pair (described
in~\secref{sec:relation-contains}), which ties together objects of different kinds, combining their set of attributes in
the composite pattern.  This relation enables abstracting common properties, like ``placement in rack'', into a
dedicated object kind and ``including'' this particular kind into multiple other kinds, preventing needless code
duplication and enabling cleaner design of tools which process the data.

The {\tt TEMPLATIZED} relation enables further specialization of muster objects through inheriting their attribute
values as a default --- see~\secref{sec:relation-templatized} for details.

Finally, the flat lists of objects, as provided by the raw Deska database, are transformed into full-fledged trees
through the {\tt EMBED\_INTO} relation (\secref{sec:relation-embed-into}).  This object embedding allows the
administrator to design database scheme in close resemblance with a real world, preventing creation of GUID-like names
in the user-visible layers.

Relations are a rich and complicated topic, and readers are encouraged to explore their features in next section,
which provides a much more detailed explanation of these fundamental features of the Deska database.

Seasoned database architects who have already read the introductory chapters and developed understanding of the Deska
design are kindly requested to be patient while we describe the whole concept in the following sections.

\section{Objects and Relations: The Basic Building Blocks}
\label{sec:objects-and-relations}

Deska does not try to deny its background in relation databases, and therefore it comes at no surprise that the basic
building blocks, the Deska {\em objects}, {\em kinds} and {\em relations}, all have their well-known parallels in the
world of relational databases and object-oriented programming.

Anything which is stored in the Deska database is an {\em object}. An object can be thought of as a row in a DB table ---
all objects of the same type, or {\em kind} as we call that in the Deska jargon,\footnote{Unfortunately we were unable
to call Deska kinds {\em classes}, which would be the best name for them, because a lot of languages use {\tt class} as
a keyword.  We have chosen to use the slightly unintuitive word {\tt kind} instead of obvious alternatives like {\tt
class\_}.} live in one table, and they all can
store the same amount of information in their {\em attributes}, the database columns.  One of the columns is usually a
primary key, which in the Deska corresponds to a special attribute called {\em name}, and serves as a unique identifier
of an object.

Two objects from different kinds can have the same name, and unless the scheme defines a special connection between
these two kinds, the two objects will remain completely isolated from each other.  This flat set of object kinds is
further expanded by the concept of {\tt relations}, a feature which allows modelling complex real-world scenarios on top
of this basic functionality.

\section{Relations}
\label{sec:objects-and-relations-relations}

A {\em relation}~\footnote{These relations do {\em not} refer to relations from relational algebra, the name is again a
result of a limited set of words usable as identifiers.  We have deemed the word ``relation'' to be more accurate than
the alternatives like ``association'' or ``connection'', which all have well-defined meaning in certain programming
frameworks.} is an extra association between two individual object instances which can be used to model wildly differing
connections between real-world objects.  A relation is defined by the left and right {\em kinds} to which it refers and
by the associated {\em type} of the relation.\footnote{This is further augmented by auxiliary information of which
object attribute is used to determine the target object of the relation, but all of that can be safely ignored for now.
Don't worry, we'll get back to that later.}

The relations serve different purposes, all of which are described in this section.  Some of them are suitable for
maintaining referential integrity, others serve as ``context helpers'' for humans to work with, yet others can be used
to build complex inheritance and composition patterns, common to the object-oriented programming.

\subsection{Referential Integrity}
\label{sec:relation-refers-to}

Let's start with perhaps one of the simplest relation, the {\tt REFERS\_TO} one.  The sole purpose of this relation is
to make sure that an object's attribute does not contain just an ordinary text, but that the value of said attribute
refers to an object of a given kind.

There are many use cases for such relation, the most basic one is probably the DB-way of defining
enumerations.\footnote{The domain of the acceptable values of the enumeration is determined from the user-supplied data
at runtime, leading to a dynamic nature of these enums.} Suppose
we are designing a hardware inventory and want to keep track of vendors or manufacturers.  It would be impractical to
modify the database code when a new company selling servers emerges, so it makes sense to treat the list of
manufacturers as data.  This is simply done by creating a new kind, let's call it ``vendor'', and instantiating new
objects for each supplier currently in use.  The {\tt machine} kind which typically holds objects representing each
physical server would then gain a new attribute called {\tt vendor} with type {\tt identifier}, which will make sure
that the possible values have to conform to the syntax constraints defined for identifiers.  For the referential
integrity to work, a new {\em relation} shall be created.  This new relation shall have the type {\tt REFERS\_TO}, and
be from the kind {\tt machine} to kind {\tt vendor}.

When a new delivery from a brand new company arrives, activating that new manufacturer is simply a matter of
creating new object of the {\tt vendor} kind.

\subsection{Relations Explained}

Each relation therefore merely specifies that objects of a particular kind {\em can} somehow refer to objects of some
other kind (or of the same kind, for that matter).  It is important to understand that this relation is {\bf optional},
and can be present or absent on a per-object basis, and that these relations concern the individual objects.  Under the
hood, relations are implemented as database foreign keys, and as foreign keys can be {\tt NULL} unless explicitly
prohibited, an object can easily exist without that particular relation taking any effect.\footnote{Users are free to
add any additional integrity constraints to their database scheme, and can therefore easily force a particular relation
to come into existence through the {\tt NOT NULL} stanza.}

The foreign key is just an integrity constraint on top of a table column which is otherwise similar to other object
attributes.  The column has its own name defined by the scheme architect, and is usually fully accessible to the DBAPI
users.  The established Deska conventions call for either using a name which matches the purpose of the relation (like
{\tt template}, {\tt inherit} etc.), or simply reusing the name of the target kind.  There are certain situations (which
are even present in the sample schemes, as shipped with Deska) which warrant using a completely different name, so
implementors shall refer to their best judgement when inventing naming conventions.

\subsection{Multi-value References}
\label{sec:relation-multi-value-references}

So far, we have shown relations which generally represent an N:1 mapping --- an object of the source kind (a row in the
source table) refers either to nothing, or to exactly one target object.  However, this concept is not sufficient for a
particularly compelling feature, notably for {\em tagging}.

Tagging is probably a well-known concept from the general-purpose computing experience. An object can be tagged by a set
of labels, which generally refer to identifiers of other objects.  In real life, a photo gets tagged with names of
people who are on the picture.  In a data center, each server gets tagged with several roles that it fulfils; for
example, machine {\tt srv012} can act as a web server and an MX host at the same time, and hence shall get two tags, the
{\tt www} and {\tt smtp} ones.

In Deska, this feature is provided by the combination of the {\tt identifier\_set} attribute data type and the {\tt
REFERS\_TO} relation.  In our example, the {\tt role} is a standalone kind, and the {\tt host} kind has an attribute
{\tt services} of type {\tt identifier\_set}.  Where the RDBMS equivalent of tagging would involve dealing with an
auxiliary table just for the mapping, the Deska DBAPI provides an abstraction without any extra kinds.

\subsection{Templates: Default Value Inheritance}
\label{sec:relation-templatized}

There are certain cases where many objects have certain shared properties.  As an example, imagine a situation where you
order a delivery of 50 new, identical servers.  A lot of their properties is going to be the same, or almost the same,
among all of them.  Therefore it might make sense to create one artificial object, let's call it a {\em template} one,
put all the shared data in there, and have fifty new individual objects ``inherit'' the default values from it.  This
scenario is assisted by the {\tt TEMPLATIZED} relation.

Any Deska kind having an attribute whose name starts with {\tt template\_} prefix will gain one additional kind which
serves as the source of the default values.  The {\tt TEMPLATIZED} relation between them is set up automatically and one
can activate it (remember, relations always exist between individual objects) by simply setting the corresponding
{\tt template\_something} attribute.  So in our sample scenario when a vendor delivers fifty new RM-666 boxes, we would
create one instance of the {\tt template\_hardware} kind with the name {\tt rm666\_2011\_03}, assign interesting
properties to that object (like the end of warranty period, support contract number etc.), and then create fifty {\tt
hardware} objects for the individual machines, setting their {\tt template\_hardware} attribute to {\tt
rm666\_2011\_03}.  This will make sure that attributes which have not been explicitly set at the level of each object
are inherited from the associated template.

This default value substitution can be overridden at any time by setting an explicit non-null value at the level of the
leaf object.  Resetting the attribute value back to {\tt null} later effectively restores the inherited default.  It is
not possible to override an inherited attribute value by {\tt null}; doing that requires explicitly cancelling the
relation for that particular object (in our case, the inheritance is stopped by setting the {\tt hardware}'s attribute
{\tt template} to {\tt null}).

Users should be careful to not abuse the template relation where composition would be more appropriate.  As a rule of
thumb, if it makes sense to deal with a subset of object's attributes as a different entity (maybe distinguishing
between the hardware's physical dimensions, its inner compounds like CPU configuration, the model number, and machine's
rack location), it is usually recommended to go with composition or something else instead of templating.

\subsection{Composition}
\label{sec:relation-contains}

In the previous section, we have briefly mentioned that it makes sense to ``split'' properties of one real-world object
into several groups.  On the other hand, doing so imposes a significant burden on the hands of a system administrator
who suddenly does not have to create just a single object for each machine in the rack, but several of them --- one for
the hardware inventory, another for ``rack placement'' (or position), yet another for the logical computer that runs on
top of that, and so on.  To confine this object explosion and to limit the amount of work required to handle such
objects, Deska provides the {\tt CONTAINS} relation.

Using this relation, a logical object is essentially composed of many smaller objects, where each of the ``contained''
parts provides a particular function.  For example, a logical ``computer'' object could be composed of the following
building blocks: {\tt host} (for stuff like hostname, OS configuration, embedded network interfaces etc.), {\tt hardware} (for stuff like
serial number or installed RAM) and {\tt placement} (for the physical space occupied in a rack).  An Ethernet switch would typically
include the {\tt placement} object as well, so it means that the low-level building blocks can be freely reused,
provided no conflicts arise.  It is important to realize that this reusing happens at the {\em kind} level and not at
the object level ---  an individual {\tt placement} object can be linked to a {\tt computer} or to a {\tt switch}, but
never to both.  The {\tt placement} {\em kind}, however, contains many individual objects, some of which refer to {\tt
switch}, and others referring to {\tt computer} instances.

The Deska database contains special hooks which prevent possible ambiguities. For example, when user creates an object
which can be included in another one (in our example, a {\tt placement}), appropriate foreign keys will immediately tie
the freshly created instance to the pre-existing object with the same name, if one exists.  Therefore, if there is
already a switch called {\tt foobar} and one attempts to create a {\tt placement} of the same name, it will immediately
get tied to said switch.  In cases when the database cannot determine the target object to link to without any
ambiguity, ie. when there are already the {\tt host} and {\tt switch} objects with the same name, the database will not
allow creating the {\tt box} object and respond with a constraint error when the user attempts to do so.

It is up to the Deska administrator to define further database constraints to prevent creating the ``low-level''
building blocks objects on their own right, without being ``used'' by the other objects --- an example where this would
be very appropriate is our {\tt placement} kind.  In other situations however, it might be wise to allow the low-level
objects to exist without actually being referenced from elsewhere (it should generally be allowed to put a machine into
inventory even though it remains off and without an assigned hostname, and hence just the {\tt hardware} object shall
exist, without the accompanying {\tt host}).  In all cases, though, the Deska server automatically takes care of setting
up foreign keys properly when the ``contained'' and ``using'' objects come into existence at any time, no matter which
one gets created first.

When an object at either side of the relation gets renamed, the relation is automatically torn apart and the same rules
which are active at the time of object creation are checked again, enforcing that objects with the same name (and only
those) are linked together.  This means that in order to rename a ``logical'' object which is composed of several other
objects, one has to perform a \deskaFuncRef{renameObject} on all of them.  If a single object gets renamed, the rename
can potentially cancel the relation and immediately establish a link to another instance due to them having the same new
name.

Due to technical reasons, the {\tt CONTAINS} relation is complemented by a {\tt CONTAINABLE} relation in the opposite
direction.  As a rule of thumb, if there is a relation {\em A contains B}, there is also a {\em B containable A}.

\subsection{Object Embedding}
\label{sec:relation-embed-into}

Earlier in this chapter, we have mentioned that there is no structure among the kinds, and that as a result, all objects
have to bear a name which is unique among all objects in the given kind.  However, not only the real world in all its
complexities, but also a simple data center is full of examples where this naming convention would get in the way; the
above mentioned example of a network interface name is a prime example.  There are many ways out of that; the most
obvious ones involve GUIDs and other forms of human-unfriendly means.  In Deska, we have chosen another approach which
is assisted by the {\tt EMBED\_INTO} relation, and is hopefully more readable to humans.

An embedded object ``almost'' violates the limitation that an object's name has to be unique in a given kind.  Instead,
the global name is composed of two parts: the name of the {\em parent} object and a logical name, which has to be valid
only in the context of a parent, which are then joined together by a separator {\tt ->}, hereby forming a unique
identifier.  In practice, {\tt srv012}'s network interface {\tt eth0} would be therefore called {\tt srv012->eth0}.
Under the hood, this is implemented by two distinct attributes and the standard rules for constraint checking guarantee
that there is no potential for possible inconsistency.

In practice, this relaxed naming is still not enough to be directly usable by people, and hence the Deska CLI (and any
other GUIs shipped with Deska, for that matter) fully supports object embedding using this relation.  There is no need
to manually type composed identifiers, as the object nesting is presented transparently to users.

The object embedding also has a feature that allows generating locally unique names for the local part of the name.
That features comes handy when a DB scheme uses object embedding e.g. for tracking failures and other events associated
with a piece of hardware.  The Deska way of handling this use case is to create a kind (let's call it {\tt failure}) and
have its objects embedded into the parent object instances (like {\tt hardware}).  However, doing so without an
additional assistance provided by the CLI or the Deska server would impose an annoying burden on users, who would have
to come up with a unique name when logging a ``failure'' of an equipment.  To address this limitation, the Deska DBAPI
call for creating new objects (\deskaFuncRef{createObject}) has an option for automatically assigning a locally-unique
name for the object just created.  This means that the user just has to ``create a failure for machine {\tt srv012}'',
and the Deska DB itself will assign an identifier like {\tt srv012->1} to it.  This feature is directly available from
the CLI, and the user does not have to deal with low-level naming conventions at all.

\subsection{Relations Rationale}

We have seen that the relations allow describing complex object hierarchies without violating the basic Deska design
rule that the database contains flat lists of objects, which themselves consist of a key-value pairs holding the actual
data.  Applications are therefore free to use the Deska DBAPI without having any knowledge about the relations at all,
and they would still have access to all the data.  At the same time, applications which want to go further and provide
additional assistance can use advanced features (and more function calls) to effectively streamline the end users' work
flow.  The worst thing that the former, ``naive'' applications could encounter are cases where a particular action
(like, for example, creating a {\tt switch} and a {\tt host} with the same name) is denied by the integrity checking
constraints.  On the other hand, applications which choose to employ the relations can use them to build rich object
hierarchies and presenting them to the users without creating much confusion.

\subsection{Limits of Relations}

There are certain rules that limit the possible relations.  Basically, any setup which wishes to use relations has to
respect the following two simple conditions:

\begin{itemize}
    \item {\em No duplicates.} Given kinds A and B and an attribute C of kind A, there is at most one relation between A
        and B that is built upon attribute C.
    \item {\em No conflicts.} Any setup which might introduce ambiguity to the possible interpretation of relations is
        prohibited.
\end{itemize}

While the first condition is clear on its own, the second rule deserves further explanation.

First of all, there could ever only be a single {\tt EMBED\_INTO} relation between two given kinds, and that relation
must not be present in both directions.  In addition to that, each kind can only be embedded into at most one other
kind, but never more.\footnote{This limitation might be lifted in future versions of Deska.  Applications are advised to
be ready for such a change.}  Finally, this hierarchy must not ever form a circular graph (with kinds being vertices and
relations being edges).

The {\tt TEMPLATE} relation has similar limitations -- a kind can only be templatized by at most one other kind, and no
two kinds can be templatized by the same template.  A kind which serves as a template for some other kind also provides
default values for itself. There are also special requirements on the exact name of the attribute which is used for the
relation, which are documented in~\secref{sec:db-scheme-req}.

Semantics of the {\tt CONTAINS} relation are also rather strict.  A low-level building block can be contained in many
higher-level objects, provided that the containment graph never contains a diamond-like shape.  This means that if kinds
A and B both contain kind C, either directly or indirectly, through other kinds, it is forbidden to have another kind D
contain A and B at the same time, again directly or indirectly.

The rules for the {\tt REFERS\_TO} are much more relaxed --- essentially, only the {\em no duplicates} rule applies here.
This relation does not convey any semantic which is important for the Deska tools, the meaning is completely up to the
user and can be paraphrased as just ``this object makes a reference of some type to that object''.

All relations are also subject to the general requirements imposed by the SQL implementation; these are described in
dept in~\secref{sec:admin-dbscheme}.

\section{Attributes}

In the previous sections, we have often used the term {\em attribute} without properly codifying what we refer to,
relying on the intuitive definition.  For most purposes, such a definition is enough, but let's reiterate what an
attribute really is and what data it can hold.

As we have already mentioned, a {\em kind} specifies which attributes can be set (and hold data) for an object.  These
attributes are simply a pair of a key and value.  Each attribute hence has a {\em name}, which is defined at the kind
level, and a {\em value}, which is specific to each object instance.  The value is well-typed, meaning that each kind
defines a data type for each attribute, and the data being stored in the attribute values has to conform to the
specified data type.  It is also possible to place additional constraints on the attribute values, either through the
object relations, or via further database triggers.

The list of supported data types covers the most obvious and common needs (like strings and numeric values), and also
includes datacenter-specific types like IPv4/IPv6 or Ethernet MAC addresses.  For a full list of the defined data types,
please consult \secref{sec:json-data-types-reference}.

\section{Wrap-up}

After reading through this chapter, the reader shall be reasonably familiar with the concepts used throughout the Deska
system.  At this time, it might make sense to revisit the sample scenario which we have described
in~\secref{sec:user-workflow}, and visualize how the described actions could be performed on the Deska objects.

\end{document}
