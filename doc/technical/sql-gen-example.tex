% vim: spelllang=en spell textwidth=120
\documentclass[deska]{subfiles}
\begin{document}

\subsection{SQL Generator Example}
\label{sec:sql-gen-example}

Here is an example how the stored procedures are generated.  This example shows generation of set attribute functions.

{\tt Schema} class explores structure of each module table which is in the user defined schema. For each attribute of
this table, an appropriate function for setting this attribute is generated.

Set attribute functions for attributes which refer to {\tt uid} column of some table differ from functions for
attributes which do not refer to any {\tt uid} column (see \secref{sec:deska-db-ver-data-modif}).

\subsubsection{Ordinary column}
In case when the column does not refer to any {\tt uid} column, {\tt Schema} class calls {\tt gen\_set} method of the
{\tt Table} class. The {\tt gen\_set} method uses constant with following string prototype of the set attribute
function:

\begin{minted}{python}
    set_string = '''CREATE FUNCTION 
    %(tbl)s_set_%(colname)s(IN name_ text,IN value text)
    RETURNS integer
    AS
    $$
    DECLARE ver bigint;
        rowuid bigint;
        tmp bigint;
    BEGIN
        --for modifications we need to have opened changeset, this function raises exception in case we don't have
        SELECT get_current_changeset() INTO ver;
        SELECT %(tbl)s_get_uid(name_) INTO rowuid;
        --not found in case there is no object with name name_ in history
        IF NOT FOUND THEN
            RAISE 'No %(tbl)s named %%. Create it first.',name_ USING ERRCODE = '70021';
        END IF;
        -- try if there is already line for current version
        SELECT uid INTO tmp FROM %(tbl)s_history
            WHERE uid = rowuid AND version = ver;
        --object with given name was not modified in this version
        --we need to get its current data to this version
        IF NOT FOUND THEN
            INSERT INTO %(tbl)s_history (%(columns)s,version)
                SELECT %(columns)s,ver FROM %(tbl)s_data_version(id2num(parent(ver))) WHERE uid = rowuid;
        END IF;
        --set new value in %(colname)s column
        UPDATE %(tbl)s_history SET %(colname)s = CAST (value AS %(coltype)s), version = ver
            WHERE uid = rowuid AND version = ver;

        --flag is_generated set to false
        UPDATE changeset SET is_generated = FALSE WHERE id = ver;
        RETURN 1;
    END
    $$
    LANGUAGE plpgsql SECURITY DEFINER;
'''
\end{minted}

The mapping keys are inside the {\tt gen\_set} function replaced with appropriate strings. The mapping key {\tt tbl} is
replaced with name of the table, {\tt colname} with name of the column, {\tt columns} with a list of the table's columns
and {\tt coltype} with the type of the column.

The following example shows a function generated for table {\tt hardware} and its attribute {\tt purchase}:

\begin{minted}{sql}
CREATE FUNCTION
    hardware_set_purchase(IN name_ text,IN value text)
    RETURNS integer
    AS
    $$
    DECLARE ver bigint;
        rowuid bigint;
        tmp bigint;
    BEGIN
        --for modifications we need to have opened changeset, this function raises exception in case we don't have
        SELECT get_current_changeset() INTO ver;
        SELECT hardware_get_uid(name_) INTO rowuid;
        --not found in case there is no object with name name_ in history
        IF NOT FOUND THEN
            RAISE 'No hardware named %. Create it first.',name_ USING ERRCODE = '70021';
        END IF;
        -- try if there is already line for current version
        SELECT uid INTO tmp FROM hardware_history
            WHERE uid = rowuid AND version = ver;
        --object with given name was not modified in this version
        --we need to get its current data to this version
        IF NOT FOUND THEN
            INSERT INTO hardware_history (box,purchase,vendor,uid,template_hw,note,host,name,version)
                SELECT box,purchase,vendor,uid,template_hw,note,host,name,ver FROM hardware_data_version(id2num(parent(ver))) WHERE uid = rowuid;
        END IF;
        --set new value in purchase column
        UPDATE hardware_history SET purchase = CAST (value AS date), version = ver
            WHERE uid = rowuid AND version = ver;

        --flag is_generated set to false
        UPDATE changeset SET is_generated = FALSE WHERE id = ver;
        RETURN 1;
    END
    $$
    LANGUAGE plpgsql SECURITY DEFINER;

\end{minted}

\subsubsection{Column Referring to the uid Column}

In case when the column refers to the {\tt uid} column of another table, the {\tt Schema} class calls the {\tt
gen\_set\_ref\_uid} method of the {\tt Table} class. The {\tt gen\_set\_ref\_uid} method uses constant with following
string prototype of the set attribute function:

\begin{minted}{python}
    set_fk_uid_string = '''CREATE FUNCTION
    %(tbl)s_set_%(colname)s(IN name_ text,IN value text)
    RETURNS integer
    AS
    $$
    DECLARE ver bigint;
        refuid bigint;
        rowuid bigint;
        tmp bigint;
    BEGIN
        SELECT get_current_changeset() INTO ver;
        --value is name of object in reftable
        --we need to know uid of referenced object instead of its name
        IF value IS NULL THEN
            refuid = NULL;
        ELSE
            SELECT %(reftbl)s_get_uid(value) INTO refuid;
        END IF;

        SELECT %(tbl)s_get_uid(name_) INTO rowuid;
        -- try if there is already line for current version
        SELECT uid INTO tmp FROM %(tbl)s_history
            WHERE uid = rowuid AND version = ver;
        --object with given name was not modified in this version
        --we need to get its current data to this version
        IF NOT FOUND THEN
            INSERT INTO %(tbl)s_history (%(columns)s,version)
                SELECT %(columns)s,ver FROM %(tbl)s_data_version(id2num(parent(ver))) WHERE uid = rowuid;
        END IF;
        --set column to refuid - uid of referenced object
        UPDATE %(tbl)s_history SET %(colname)s = refuid
            WHERE uid = rowuid AND version = ver;

        --flag is_generated set to false
        UPDATE changeset SET is_generated = FALSE WHERE id = ver;
        RETURN 1;
    END
    $$
    LANGUAGE plpgsql SECURITY DEFINER;

'''
\end{minted}

In this case, the mapping key {\tt tbl} is replaced with name of the table, {\tt colname} with name of the column, {\tt
columns} with list of the table's columns and {\tt reftbl} with name of the destination table (a table to whose {\tt
uid} column the column in question refers).

This is how a function generated for {\tt hardware}'s {\tt vendor} looks like:

\begin{minted}{sql}
CREATE FUNCTION
    hardware_set_vendor(IN name_ text,IN value text)
    RETURNS integer
    AS
    $$
    DECLARE ver bigint;
        refuid bigint;
        rowuid bigint;
        tmp bigint;
    BEGIN
        SELECT get_current_changeset() INTO ver;
        --value is name of object in reftable
        --we need to know uid of referenced object instead of its name
        IF value IS NULL THEN
            refuid = NULL;
        ELSE
            SELECT vendor_get_uid(value) INTO refuid;
        END IF;

        SELECT hardware_get_uid(name_) INTO rowuid;
        -- try if there is already line for current version
        SELECT uid INTO tmp FROM hardware_history
            WHERE uid = rowuid AND version = ver;
        --object with given name was not modified in this version
        --we need to get its current data to this version
        IF NOT FOUND THEN
            INSERT INTO hardware_history (box,purchase,vendor,uid,template_hw,note,host,name,version)
                SELECT box,purchase,vendor,uid,template_hw,note,host,name,ver FROM hardware_data_version(id2num(parent(ver))) WHERE uid = rowuid;
        END IF;
        --set column to refuid - uid of referenced object
        UPDATE hardware_history SET vendor = refuid
            WHERE uid = rowuid AND version = ver;

        --flag is_generated set to false
        UPDATE changeset SET is_generated = FALSE WHERE id = ver;
        RETURN 1;
    END
    $$
    LANGUAGE plpgsql SECURITY DEFINER;
\end{minted}

\subsubsection{Identifier\_set Column}

In case when the column refers to a set of identifiers (see \secref{sec:relation-multi-value-references}), the {\tt
Schema} class calls {\tt gen\_set\_refuid\_set},  {\tt gen\_refuid\_set\_insert} and {\tt gen\_refuid\_set\_remove}
methods of the {\tt Table} class.

The {\tt gen\_set\_refuid\_set} method uses constant with following string prototype for generating the set attribute
function:

\begin{minted}{python}
    set_refuid_set_string = '''CREATE FUNCTION
    %(tbl)s_set_%(colname)s(IN name_ text,IN value text[])
    RETURNS integer
    AS
    $$
    DECLARE
        ver bigint;
    BEGIN
        ver = get_current_changeset();

        --flag is_generated set to false
        UPDATE changeset SET is_generated = FALSE WHERE id = ver;

        BEGIN
            --row is inserted because of diff and changes between versions
            --this means object was modified
            INSERT INTO %(tbl)s_history (%(columns)s,version)
                SELECT %(columns)s,ver FROM %(tbl)s_data_version(id2num(parent(ver))) WHERE name = name_;
        EXCEPTION
            WHEN unique_violation THEN
            -- do nothing
        END;
        RETURN genproc.inner_%(tbl)s_%(colname)s_set_%(colname)s(name_, value);
    END
    $$
    LANGUAGE plpgsql SECURITY DEFINER;

'''
\end{minted}

Similar to the situation discussed above, the mapping key {\tt tbl} is replaced with name of the table, the {\tt
colname} with name of the column and {\tt columns} with a list of the table's columns.  Functions {\tt
inner\_\%(tbl)s\_\%(colname)s\_set\_\%(colname)s} are generated by the {\tt Multiref} class.   These functions take care
of performing the necessary modification of an inner table that stores identifier set of the {\tt tbl} table's {\tt
colname} column.

This is how a function generated for {\tt host}'s {\tt service} attribute looks like:

\begin{minted}{sql}
CREATE FUNCTION
    host_set_service(IN name_ text,IN value text[])
    RETURNS integer
    AS
    $$
    DECLARE
        ver bigint;
    BEGIN
        ver = get_current_changeset();

        --flag is_generated set to false
        UPDATE changeset SET is_generated = FALSE WHERE id = ver;

        BEGIN
            --row is inserted because of diff and changes between versions
            --this means object was modified
            INSERT INTO host_history (hardware,virtual_hardware,uid,service,name,version)
                SELECT hardware,virtual_hardware,uid,service,name,ver FROM host_data_version(id2num(parent(ver))) WHERE name = name_;
        EXCEPTION
            WHEN unique_violation THEN
            -- do nothing
        END;
        RETURN genproc.inner_host_service_set_service(name_, value);
    END
    $$
    LANGUAGE plpgsql SECURITY DEFINER;
\end{minted}

Functions {\tt gen\_refuid\_set\_insert} and {\tt gen\_refuid\_set\_remove} generate functions {\tt 
\%(tbl)s\_set\_\%(colname)s\_insert} and {\tt \%(tbl)s\_set\_\%(colname)s\_remove} in the same way as 
the {\tt gen\_set\_refuid\_set}, they just call functions {\tt inner\_\%(tbl)s\_set\_\%(colname)s\_insert} and {\tt
inner\_\%(tbl)s\_set\_\%(colname)s\_remove} (which are generated by {\tt Multiref} class).  The functions {\tt
inner\_\%(tbl)s\_set\_\%(colname)s\_insert} and {\tt inner\_\%(tbl)s\_set\_\%(colname)s\_remove} insert or remove item
into or from the identifier set.

\end{document} 
