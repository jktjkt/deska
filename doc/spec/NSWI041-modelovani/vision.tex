\documentclass[12pt]{article}
\usepackage[utf8]{inputenc}
\usepackage[english]{babel}
\usepackage{a4wide}
\author{Tomáš Hubík}
\title{Deska}



\begin{document}

{\Huge \textbf{Deska}}

\vspace{0.2in}

{\large Tool for Central Administration of a Grid Site}

\vspace{0.5in}

{\large Scope Document}

\vspace{0.2in}

{\large Prepared for Institute of Physics of the AS CR}

\vspace{0.2in}

{\large Version 1.1}

\vspace{0.2in}

{\large Date 2009-Oct-27}

\vspace{0.5in}

\subsection*{Revision History}

\begin{table}[!h]
	\begin{tabular}{l l l l}
		\textbf{Date} & \textbf{Version} & \textbf{Author} & \textbf{Description} \\
		2009-Oct-23 & V1.0 & Tomáš Hubík & Master plan \\
		2009-Oct-27 & V1.1 & Tomáš Hubík & Features specified \\
	\end{tabular}
	\label{tab:RevisionHistory}
\end{table}


\subsection*{Document Review}

\begin{table}[!h]
	\begin{tabular}{l l l l}
		\textbf{Date} & \textbf{Version} & \textbf{Reviewer Name} & \textbf{Contact Info.} \\
	\end{tabular}
	\label{tab:DocumentReview}
\end{table}


\subsection*{Document Approval}

\begin{table}[!h]
	\begin{tabular}{l l l l}
		\textbf{Date} & \textbf{Version} & \textbf{Reviewer Name} & \textbf{Contact Info.} \\
	\end{tabular}
	\label{tab:DocumentApproval}
\end{table}


\newpage

\tableofcontents

\newpage

\section{Introduction}

\subsection{Purpose}
This document defines the project scope for the Tool for Central Administration of a Grid Site. It establishes the business need for 
the Deska, and it outlines the high level requirements needed to satisfy the specified need. This document is not an exhaustive 
requirements description, but will instead provide overall direction for a separate -- more detailed -- set of system requirements. 
This document's primary purpose is to establish priority among the most important issues, considerations, features, and overall goals.

\subsection{Scope}
The Deska will be responsible for managing and monitoring of a grid site. Especially it automates configuration of several monitoring 
and deployment tools. All functions and features will be displayed and managed through a web interface.


\section{Business Opportunity}

\subsection{Background}
One of the activities taking at the Institute of Physics of the AS CR (FZU) is running a local Tier-2 computing center connected to 
an international computing grid, running user programs from various scientific communities from the whole world. Currently 
installed capacity of the clusters is about 1500 CPUs with more than 250TB of disk storage. The center itself consists of several 
hundreds of physical machines. While certain policies and management tools were developed and deployed over the years, the center 
still lacks a centralized dashboard listing the state of various resources.\\
The administrators were pretty successful at deploying various industrial-standard management and monitoring tools. Nagios is used 
for fabric monitoring, most of the machine's configuration is being handled by Cfengine. SNMP agents are used for monitoring 
critical parts of the network infrastructure and Munin is used for generating graphs depicting performance characteristics of the 
computing nodes. Various in-house applications were developed to assist with keeping track of HW issues and interesting events.\\
Unfortunately, all these tools are currently being managed in an isolated manner. Despite some efforts, there is still no central 
place to define roles of machines which would in turn be used by all the other tools.

 
\subsection{Positioning}
The system will not replace any of the existing monitoring tools. It will replace only their web interfaces and display all the 
information by itself in some well-arranged way. It will substitute the unpractical way of configuration of currently used tools, 
that must be made manually by editing of several files now.

\subsection{Impact of Missed Opportunity}
The fundamental benefit of this system is that it will save time of administrators, that can focus on other things. It will also 
decrease the possibility of making a mistake while adding new machine and resulting time consuming finding of this mistake.


\section{Proposed Solution}

\subsection{Primary Functional Requirements}
In this section, we classify the primary features of the Tool for Central Administration of a Grid Site across three categories. 
Essential features cannot be done without. High-value features can be done without, although it may be very undesirable to do so. 
Follow-on Features are those for which it is not clear they should be included in the first release. In all cases, the lists are not 
exhaustive but include the most important features from a business perspective.

\subsubsection{Essential Features}
\begin{itemize}
	\item Central storage of authoritative information about the whole grid site
	\item Generating configuration files for various tools. For example Nagios, Cfengine
	\item Unified API for creating other modules
\end{itemize}

\subsubsection{High-Value Features}
\begin{itemize}
	\item Data aggregation from more tools to one well-arranged place, one place where to look for problems
\end{itemize}

\subsubsection{Follow-on Features}
\begin{itemize}
	\item Data correlation about load from various sources to determine if there is some dead lock, or hung application
	\item Implementation of modules for other tools, than those, that are currently used on FZU
\end{itemize}

\subsection{Primary Non-Functional Requirements (NFRs)}
In this section, we identify the primary non-functional requirements for the Tool for Central Administration of a Grid Site. This 
list is not exhaustive but is intended to capture the most important requirements from a business perspective. A full list of NFRs 
will be included in the SRS document.

\subsubsection{Performance, Throughput, and Scalability}
The system is expected to handle up to thousands of managed machines, tens of administrators with about ten changes performed each hour.\\
The system will provide a high level of scalability, so we will be able to run it over various monitoring and deployment tools. 
Configuring of single applications will be made by modules. These modules will use our system's API, so creating a new module for 
another application will be an easy task.

\subsubsection{Reliability and Availability}
Because of the fact, that our system will not replace any of the functional parts of current system, it is not necessary to be 
available 7 by 24 by 365.\\
It is necessary to be well protected against failure while generating configuration files, because this failure can lead to failure 
of the whole grid site. This will be made by some versioning of these files, so there will be some way back if this failure occurs.

\subsubsection{Security}
Several features of the system require authorization. The system will be divided into security levels. Each level will be allowed 
to make changes only of some attributes. Every module will have some list of attributes, which it will need, so if somebody 
does not have permission for changing all these required attributes, he will not be allowed to activate this module.

\subsubsection{Usability}
The application will be used by administrators who speak english, so the system will initially be translated into English.\\
It is also expected, that administrators have some experience with grid site maintenance.


\section{Risks}
Making some mistake in changing of attributes can have very serious consequence to the whole grid site. Therefore administrator can
check all generated configuration files and can make the final decision whether apply these changes, or not.


\section{Constraints}

\subsection{Development Process and Team Constraints}
At this time, there are no known constraints.

\subsection{Environmental and Technology Constraints}
At this time, there are no known constraints.

\subsection{Delivery and Deployment Constraints}
The application will be hosted on some machine with UNIX type operating system. The web part of the system will be browser 
independent.\\
FZU wants to have a possibility to influence the essential parts of design of the system, so we have to discuss all important issues
with administrators of FZU.

\end{document}