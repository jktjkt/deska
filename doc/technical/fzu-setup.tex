% vim: spelllang=en spell textwidth=120
\documentclass[deska]{subfiles}
\begin{document}

\chapter{Deska Setup at FZÚ}
\label{sec:fzu-setup}

\begin{abstract}
This chapter describes the process and deliverables which were implemented as a part of integrating the Deska system
with the pre-existing infrastructure of the Institute of Physics.  In particular, we discuss the features of the
database scheme, the process of the data conversion and the deployed output configuration generators.
\end{abstract}

\section{Database Scheme}
\label{sec:fzu-scheme}
\todo{Lukas: fzu DB scheme}

\section{Data conversion}
\label{sec:fzu-farmdb}

The Institute of Physics has had a web-based application for inventory management for a few years.  This
ASP.NET~\cite{asp.net} based application has served well, but has focused mostly on keeping track of rack placement of
the hardware, leaving much of more complicated tasks like assigning roles to machines unsolved.  The database also mixed
different sorts of items in one table (i.e. it made no distinctions between a typical server and an Ethernet switch or a
Fibre Channel disk array).  Converting this information to a highly structured data required by the Deska scheme has
proven challenging.  The process was not helped by the fact that the existing database, the FarmDB, used dimensions in
``rack fractions'' where a twin-sized SGI machine has width of 50\%, while the Deska scheme uses a hierarchy of nested
boxes and internal offsetting.

The conversion started with processing an SQL dump, constructing the objects and maintaining an in-memory representation
of the database.  The dictionary of objects was then walked and a dump suitable for processing by the Deska CLI
(\secref{sec:cli-usage}) got produced.

The full sources of the conversion script are available in the Deska source tree (\path{doc/farmdb/convert.py}), along
with the source data (\path{doc/farmdb/dbo.*.Table.sql}) and a helper script (\path{doc/farmdb/deunicodeify.sh}).

\section{Configuration Generators}
\label{sec:fzu-cfggen}
\todo{Honza: describe the history and issues of our cfggens}

\end{document}
