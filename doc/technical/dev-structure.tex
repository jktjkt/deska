% vim: spelllang=en spell textwidth=120
\documentclass[deska]{subfiles}
\begin{document}

\chapter{Developer's Introduction}
\label{sec:dev-intro}

\begin{abstract}
The Deska application suite is a complex system.  This part aims at navigating the maintenance programmer through the
basic components and explaining the interactions between them.
\end{abstract}

\section{System Structure}
The Deska system as a whole is built as an interconnected set of components.  Each component implements a particular
interface for the other parts to use.  At the basic level, and with respect to the traditional client-server model, we
can divide these components to a two groups, the first one being a {\em client} and the second one being a {\em server}.
The server holds all of the data stored in the system, while the clients ask the server to perform operations on their
behalf.  The clients talk to the server over a custom, JSON-based~\cite{json} protocol called the DBAPI, which is
described in much more detail in~\secref{sec:dbapi-protocol}.

The first client which deserves proper introduction is the CLI application, the Deska console
(see~\secref{sec:cli-app}).  This application is implemented in C++ with heavy use of the Boost~\cite{boost} libraries
and uses the provided {\tt libDeskaDb} C++ library (see \secref{sec:cpp-deskadb}) to handle communication with the Deska
server.

The Deska server is built as a thin wrapper around a PostgreSQL~\cite{postgresql} database.  The first layer of the
server is implemented in Python (see \secref{sec:server-py}) and for the most time simply translates the JSON calls
received from the clients to the PostgreSQL's stored procedures.  It also handles a few more complex tasks, like the
interaction with the configuration generators (\secref{sec:config-generators}) which are described later in this
chapter.

The other part of the Deska server is implemented as a rich set of stored procedures implemented in a mixture of
PL/PgSQL and PgPython~\cite{pg-python} languages.  Part of this system is hand-written (see \secref{sec:deska-db}),
especially the set of the core business logic.  The code which actually performs modifications of the user data (see
\secref{sec:admin-dbscheme} for a recapitulation of the role of the user-supplied scheme) is, on the other hand,
machine-generated based on the exact contents and layout of the database scheme.  The code generator is described in
\secref{sec:sql-generator}.

The Deska system does not exist in an isolated environment, so the data which are stored in the database has to be made
available to the third-party applications.  While in theory the applications could be patched to use the Deska database
as the source of information for their operation, it is way more scalable to produce a static set of configuration files
at the time new data get stored to the Deska database.  This task is done by the configuration generators (see
\secref{sec:config-generators}).  These generators are executed by the Python Deska server before each database commit,
and talk the regular DBAPI protocol (\secref{sec:dbapi-protocol}) to communicate with the rest of the Deska.  As most of
these generators are implemented in Python, we have provided a Python library (see \secref{sec:python-bindings}) on top
of the {\tt libDeskaDb} C++ library.  This Python code not only encapsulates type-safe access to
various Deska objects, but also build a native, high-level representation of the class hierarchy (see
\secref{sec:objects-and-relations-relations}) as first-class Python objects.

The configuration generators are designed to be easily pluggable to an existing data center infrastructure, and play
well with the already deployed tools.  Hooks are available to facilitate a seamless integration with the
already-existing VCS\footnote{Version Control System} system which are very likely present at each site and use for
their configuration management.  The whole idea is that the Deska system is merely another user providing input to the
fabric management systems.  Deska ships with everything required to interact with a Git~\cite{git} repository; the set
of scripts is documented in \secref{sec:cfggen-git}.

Finally, the configuration generators used at the Institute of Physics are introduced in \secref{sec:fzu-cfggen}; some
of the challenges which we had to solve were not obvious at the first time.

\section{Auxiliary Tools}

Certain tools have been developed to integrate the Deska with the concrete infrastructure of the Institute of Physics.
The script which is used for producing the data in the production database is briefly discussed in
\secref{sec:fzu-farmdb}.

We have also came across several bugs, other issues or simple drawbacks in various tools with which we have worked.  We
have always tried to do our best in submitting our patches upstream and getting them merged.  A list of these issues is
provided in \secref{sec:patches}.

\section{Automated Tests}

Due to the complexity of the Deska system, we have tried to do our best to ensure a smooth operation and prevent
critical regressions.  The Deska system therefore ships with a rich suite of test cases which verify functionality of
the low-level building blocks, as well as the interaction patterns between the components and the overall system
functionality.  These tests are covered by \secref{sec:automated-tests}.

\end{document}
