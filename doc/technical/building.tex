% vim: spelllang=en spell textwidth=120
\documentclass[deska]{subfiles}
\begin{document}

\chapter{Installing Deska}
\label{sec:building}

\begin{abstract}
This chapter guides the reader through the installation process of the Deska application suite.
\end{abstract}

\section{Packages}

FIXME: write something when we actually have any packages

\section{Manual Building}

The Deska system can be built on any reasonably modern Linux system.  The following platforms are equally tested and
supported:

\begin{itemize}
    \item Red Hat Enterprise Linux \cite{rhel} 5 (release 5.6 or newer, including derivatives like Scientific
        Linux~\cite{scientific-linux})
    \item Red Hat Enterprise Linux 6 (release 6.1 or newer, including derivatives like Scientific Linux)
    \item Gentoo Linux~\cite{gentoo}
\end{itemize}

\subsection{Building}

All platforms require a C++ compiler with decent support for templates.  Please be advised that compilation of the CLI
involves instantiation of rather complicated template structures which might require unusual amounts of RAM.  We have
observed a real memory consumption of up to 1.5~GB of RAM for each compile job.

The build process requires the following packages as an absolute minimum on a RHEL5 machine:

\begin{itemize}
    \item CMake (2.6 or later)~\cite{cmake}
    \item Boost Date Time (1.41 or newer)~\cite{boost}
    \item Boost Program Options (1.41 or newer)
    \item Boost Python (1.41 or newer)
    \item Boost System (1.41 or newer)
    \item Boost Unit Test Framework (1.41 or newer)
    \item GNU Readline (5.1 or newer; in case of RHEL5, we also require the {\tt termcap} library version 2.0.8 or
        newer)~\cite{gnu-readline}
    \item The {\tt libebt} library (optional, shipped in the sources)~\cite{libebt}
    \item The {\tt json\_spirit} library (optional, shipped in the sources)~\cite{json-spirit}
    \item The non-standard Boost Process library (patched version which is required for the tests is shipped in the
        sources)~\cite{boost-process}
\end{itemize}

The set of libraries listed above is enough to build the CLI of the Deska system.  However, in order to be able to
deploy the Deska server, the following packages are required to be present on the target server system, in addition to
the former list:

\begin{itemize}
    \item PostgreSQL (9.0 or newer)~\cite{postgresql}
    \item Python (2.6, or Python 2.4 with the {\tt simplejson}~\cite{simplejson} module version 2.1.0)~\cite{python}
    \item PsycoPg2 (2.0.14 or newer)~\cite{psycopg}
    \item The {\tt pg-python} procedural language (1.0.0 or later; not to be confused with the PlPython
        package)~\cite{pg-python}
    \item GitPython (the 0.1 branch; tested with 0.1.7, later branches have changed the API in an incompatible
        way)~\cite{git-python}
\end{itemize}

Although not strictly compulsory for regular builds, the following packages are required for the unit tests.  Running
unit tests after each manual build is highly suggested:

\begin{itemize}
    \item PgTAP unit test framework for Postgres (0.25 or later)~\cite{pgtap}
\end{itemize}

Fully automated test runs might require the following packages to be present and configured:

\begin{itemize}
    \item CMake (2.8 or later provides proper Git integration for nightly builds~\cite{deska-dashboard})
    \item Git (1.6.6 or later)
\end{itemize}

Finally, in order to build the documentation from source, the following packages will have to be present:

\begin{itemize}
    \item The {\tt pdflatex} \LaTeX~PDF compiler
    \item The {\tt minted} \LaTeX~package~\cite{latex-minted}
    \item The {\tt pygments} Python package~\cite{pygments}
    \item The {\tt subfiles} \LaTeX~package (optional, shipped with Deska)~\cite{latex-subfiles}
\end{itemize}

\subsubsection{Getting the Sources}

Apart from the stable releases available from the project homepage~\cite{deska-project}, the sources are also available
from an anonymous Git repository~\cite{deska-git}:

\begin{minted}{console}
$ git clone git://repo.or.cz/deska.git
Cloning into deska...
remote: Counting objects: 23348, done.
remote: Compressing objects: 100% (9151/9151), done.
remote: Total 23348 (delta 14099), reused 23208 (delta 14001)
Receiving objects: 100% (23348/23348), 5.12 MiB | 183 KiB/s, done.
Resolving deltas: 100% (14099/14099), done.
\end{minted}

After the sources are obtained and the source tree is ready, use the following command to perform the build.  The build
process is controlled by standard CMake options (like the {\tt -DCMAKE\_INSTALL\_PREFIX} for target path) and a few
custom ones.  Be sure to pass on the {\tt -DRUN\_SQL\_TESTS=1} statement to include complex test cases which involve
interaction with a remote PostgreSQL database.  If you would like to build the documentation as well, use the {\tt
-DBUILD\_DOCS=1} option.

\begin{minted}{console}
$ cd deska/
$ mkdir _build
$ cd _build/
$ cmake -DCMAKE_INSTALL_PREFIX=/opt -DRUN_SQL_TESTS=1 -DBUILD_DOCS=1 ..
-- The C compiler identification is GNU
-- The CXX compiler identification is GNU
-- Check for working C compiler: /usr/bin/gcc
-- Check for working C compiler: /usr/bin/gcc -- works
-- Detecting C compiler ABI info
-- Detecting C compiler ABI info - done
-- Check for working CXX compiler: /usr/bin/c++
-- Check for working CXX compiler: /usr/bin/c++ -- works
-- Detecting CXX compiler ABI info
-- Detecting CXX compiler ABI info - done
-- Boost version: 1.46.1
-- Found the following Boost libraries:
--   system
--   date_time
--   unit_test_framework
--   python
--   program_options
-- Using libebt from /home/jkt/deska/src/3rd-party/libebt-1.3.0
-- Using json_spirit from /home/jkt/deska/src/3rd-party/json_spirit_4.04
-- Using boost::process from /home/jkt/deska/src/3rd-party/process
-- Found GNU readline: /usr/lib64/libreadline.so
-- Found PythonLibs: /usr/lib64/libpython2.7.so
-- Will run SQL database tests upon each `make test` run.
-- Will re-initialize the database before each test
-- Copying directory /home/jkt/deska/_build/tests/sql
-- Writing out test configuration
-- Configuring done
-- Generating done
-- Build files have been written to: /home/jkt/deska/_build
$ time make -j4
# ...
\end{minted}

After the sources have been built, it is highly recommended to run the automated test suite.  No extra configuration is
required, apart from an available PostgreSQL server binary.  The test case will configure and start its own PostgreSQL
server instance, using the {\tt /dev/shm} {\tt shmfs} in-memory filesystem as the backing store, and will use just the
Unix socket for communication.  This test assumes that the PostgreSQL administration tools, like the {\tt initdb} and
the {\tt postgres} binary, are available to the current user.

\begin{minted}{console}
$ time ../run-standalone-tests.sh
# This step might take several minutes to complete.
100% tests passed, 0 tests failed out of 48
\end{minted}

When the tests have finished correctly, the client tools can be installed by running {\tt make install}.  The server,
however, requires more complex procedure to set up.

\section{Setup and Configuration}

FIXME: describe the database setup here

\end{document}
