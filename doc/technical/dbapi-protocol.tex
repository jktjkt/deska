% vim: spelllang=en spell textwidth=120
\documentclass{article}
\usepackage[utf8]{inputenc}
\usepackage[czech,english]{babel}
\usepackage{a4wide}
\usepackage{listings}

\begin{document}

\title{Deska: The Database Access Protocol}

\author{Jan Kundrát}

\maketitle

\begin{abstract}
This chapter discusses design of the protocol that serves as an exclusive communication method for accessing the Deska
database server.  We discuss the overall design, as well as provide a detailed API reference along with example usage.
This document is intended to serve as a complete source of information for both DBAPI usage and possible independent
reimplementation.
\end{abstract}

\section{Background}

First of all, it might not be obvious why bother with designing a custom protocol for accessing a database.  The Deska
database, after all, is built on PostgreSQL which already provides a library which can be used to access database
servers.  However, using a RDBMS-specific transport library would make the Deska application suite needlessly tied to
one particular database server implementation.  Exposing a low-level SQL-like interface, even when wrapped by stored
procedures, would make it very hard to prevent users from accessing the database without first going through a
well-controlled set of entry points.  In addition to that, we argue that it is possible to provide an API which is
easier and safer to use, compared to the raw SQL usage.

\section{Protocol Overview}

The Deska Database API is modelled after a traditional RPC interface where user invokes a procedure on the remote host
and waits for its result.  This in turn determines the mode of operation, which happens in {\em command} -- {\em
response} cycles.  That is, after a client issues a particular command, it will block and wait for the server's
response.  No form of command pipelining is available at this point, but certain provisions are in place which will
allow future extensions towards non-blocking operation.

After defining the general mode of operation for an interface, there is a need to establish a method for serialization
of data sent over the wire.  There are many widely-used ways for serializing commands, from binary-only protocols to the
XML-RPC family.  For the Deska DB API, we have chosen to use JSON, a fairly recent arrival to the already broad family
of serialization standards.  The main reason behind this choice is the fact that the protocol itself is reasonably
terse, so that one can pretty easily look at the raw form of traffic and tell what is going on without using extra tools
or frameworks.  There is also quite a number of ready-made libraries for JSON parsing, so the JSON itself is as
ubiquitous as, say, the XML.

Even though the JSON standard itself is well-established, we still had to establish a common way of converting our RPC
requests to a particular JSON form. To that end, we have chosen to use JSON {\em objects}\footnote{The reader should get
herself familiar with the notion of JSON before continuing with the rest of this document, as we will not go into much
detail explaining how low-level JSON works.}, which are merely a sequence of key-value pairs, with key being a string
identifier and value an arbitrary JSON object.  

Due to certain technical limitations, a simple stream of JSON objects is rather impractical to parse.  In fact, none of
the common libraries for JSON parsing that we have tried play well with JSON streaming.  Some of these libraries would
happily attempt to blindly ``read everything'' from a stream, convert that to a string and only then attempt the actual
parsing.  Needless to say, such an approach will not work at all when the connection expects to exchange several rounds
of JSON command-response pairs, each of them occurring in response to the previous one.  Even the more advanced JSON
libraries have issues with parsing a sequence of JSON objects following each other, without any delimiters, as most of
these tools try to ``peek'' at the next character of the stream input after parsing a full JSON object, while the ``next
character'' is typically not available at that point.  Therefore, in order to facilitate the usage of these libraries,
we have chosen to mandate that each complete JSON object at the top-level scope has to be followed by a newline (LF),
the {\tt ``\textbackslash n''} character.  Given that the newline character will never occur in a serialized JSON data,
this usage is completely safe and allows the protocol implementations to use per-line parsing for any IO interaction.

\section{JSON Serialization}

This section defines how an abstract notion of a command or response is converted to a JSON object.

\subsection{Commands and Responses}

Each command is uniquely identified by its name, and each command has a matching response with the same name. Therefore,
the hypothetical command {\tt foo}, which does not accept any commands, will be transmitted in the following form (we
will omit the trailing line feed for brevity in the following examples):

\begin{lstlisting}
    {``command'': ``foo''}
\end{lstlisting}

A response is always constructed by simply changing the {\tt command} key to a {\tt response} literal, therefore a
response for the above command will look like this:

\begin{lstlisting}
    {``response'': ``foo''}
\end{lstlisting}

\subsubsection{Passing Arguments}

Most of the commands will very likely accept some arguments.  These commands are transmitted in a very similar way to
the one we discussed above, with the arguments simply appended to the top-level JSON object.  The parameters are also
repeated in the relevant response:

\begin{lstlisting}
    {``command'': ``foo'', ``arg1'': ``arg1 value'',
        ``name of arg2'': ``some value''}
    {``response'': ``foo'', ``arg1'': ``arg1 value'',
        ``name of arg2'': ``some value''}
\end{lstlisting}

The JSON standard mandates that white space outside of string literals is ignored, and therefore implementations are
free to add it to make the JSON data more readable, or not include them at all.  In addition, to improve the
interoperability, the Deska DBAPI mandates that the order of appearance of individual key-value pairs inside a JSON
object is {\em not} significant, and therefore one is explicitly allowed to use a sequence of {\tt arg2}, {\tt command},
{\tt arg1} for the example above.  Please note that the order of key-value pairs in the response object can be different
than in the associated command.

\subsubsection{Returning Data}

So far we have only established procedures which do not return any usable data at all.  In practice, though, most of the
commands have to return a usable value.  This is achieved by extending the top-level JSON object of the response with a
single key named after the command, which contains the JSON serialization of the return value.  No other items are
added, changed or returned.

For example, if the above command was extended with a return value of {\tt string} type, a full command-response pair
would look like this one:

\begin{lstlisting}
    {``command'': ``foo'', ``arg1'': ``arg1 value'',
        ``name of arg2'': ``some value''}
    {``response'': ``foo'', ``arg1'': ``arg1 value'',
        ``name of arg2'': ``some value'', ``foo'': ``The returned value''}
\end{lstlisting}

Again, the order of appearance of the individual key-value pairs in a particular JSON object is not relevant.

\subsection{Serializing Data}

So far, we have mentioned that the data are transfered ``serialized as JSON values'', but did not specify how this
serialization is defined and what the mapping between native data types in a programming language (like an {\tt int}, or
a {\tt string}), the database server and the JSON transport is.

\subsubsection{Primitive types}

There are four basic data types in JSON, a {\tt string}, an {\tt integer}, a {\tt float} and a {\tt boolean}.  These
values are directly used for transport of the corresponding C++ data types.  However, this mapping is not fully
bidirectional for practical reasons -- one cannot look at the serialized JSON value and deduce what the original data
type was; a context is required for that operation.  This allows us to ``overload'' these JSON data types and serialize
multiple source types using a single JSON type.  A typical example of this is any particular C++'s {\tt enum}, whose
symbolic name is transmitted as a string literal with the matching name.

Suppose the following C++ definition:

\begin{lstlisting}[language=c++]
    enum Foo {
        BAR,
        BAZ
    };
\end{lstlisting}

A value of {\tt BAR} is transferred as a JSON string {\tt ``BAR''}.  There is no technical way of distinguishing between
a value of the {\tt enum Foo(BAR)} and the string literal {\tt ``BAR''}; the difference is always provided by the
context.

FIXME: date, timestamp,\ldots

\subsubsection{Containers}

The Deska DBAPI supports passing two different kinds of containers, a {\em sequence} and a {\em map}.  The sequence is
usually represented by the C++'s {\tt std::vector}, while the map maps well to the {\tt std::map}.

A sequence is represented as the JSON's list with the individual list items being the respective serialization of the
underlying data type. Suppose we want to send a {\tt std::vector<int>} of (1, 2, 3) as the {\tt a1} argument of command
{\tt foo}:

\begin{lstlisting}
    {``command'': ``foo'', ``a1'': [1, 2, 3]}
\end{lstlisting}

A map is only allowed to be indexed by string literals. A map similar to the sequence from the previous example would
look like this:

\begin{lstlisting}
    {``command'': ``foo'', ``a1'': [``a'': 1, ``b'': 2, ``c'': 3]}
\end{lstlisting}

The example above is equivalent to a {\tt std::map<std::string, int>} which contains $"a" \rightarrow 1$, $"b"
\rightarrow 2$ and $"c" \rightarrow 3$.  Please note that the order of appearance of these items in a map is undefined,
as the {\tt std::map} container will not respect the order of insertion.

\subsubsection{Custom Objects}

Custom objects, ie. instances of C++'s classes, are transmitted as JSON objects. Due to various popular JSON
impleemntations not respecting the order of appearance of individual values, the arguments to a class' constructor are
passed prefixed by name.  Therefore, given class {\tt Foo} with the following C++ constructor:

\begin{lstlisting}[language=c++]
    Foo::Foo(int i, const std::string &s);
\end{lstlisting}

We can modify our example command {\tt foo} to accept an instance of class {\tt Foo} like this:

\begin{lstlisting}
    {``command'': ``foo'', ``a1'': {``i'': 10, ``s'': ``something''}}
\end{lstlisting}

Again, the user of the Deska DBAPI cannot infer the class name from this value without knowing an extra bit of
information about the syntax of the command {\tt foo} -- without that knowledge, we are typically not aware of the
existence of class {\tt Foo} at all.

\subsubsection{Optional Values}

FIXME: talk about {\tt null} and missing arguments

\subsection{Exceptions}

There are certain situations where the execution of a particular command terminates in an unexpected manner.  One reason
could be that the user of the API has made an error and for example tried to retrieve an object which does not exist, or
that the execution hit an implementation-specific failure which has to be communicated to the caller, like an error with
SQL processing.  For this reason and in order to prevent cluttering the API with artificial {\tt NULL} values and
special placeholders, the Deska DB API supports {\em exceptions}.

FIXME: define and describe how they're transmitted.

\section{Deska DBAPI Reference}

\subsection{The DBAPI at a Glance}

FIXME: high-level overview

\subsection{Reference}

FIXME: list of commands with examples

\end{document}
