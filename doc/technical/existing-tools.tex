% vim: spelllang=en spell textwidth=120
\documentclass[deska]{subfiles}

\begin{document}

\appendix
\chapter{Existing Tools}
\label{sec:evaluating-existing-tools}

Many proprietary tools were excluded from the evaluation due to their high price or inability to effectively work in a
vendor-neutral manner.\footnote{It is common for the manufacturers of server-class hardware to have their own solution
to mass management of the individual nodes.  However, having full support for working with devices from other equipment
manufacturers is not usual.}  We have nonetheless evaluated a number of open-source tools, both from the general IT
industry, as well as those which were developed at other WLCG sites.

\section{Industry-standard Tools}

\subsection{OCS Inventory NG}

The OCS~Inventory~NG~\cite{ocs-inventory} is perhaps the most well-known information system used in the context of
hardware management.  The biggest selling point of the OCS Inventory is its ability to {\em enumerate} the whole network
and automatically discover all of the connected devices, and with the help of locally-installed agents monitor various
aspects of each host's operation.  On the other hand, it does not appear to be designed for the Institute's use case,
which is {\em providing} an authoritative answer based on policy decisions instead of an transient snapshot of a
datacenter state.

\subsection{Rackmonkey}

The Rackmonkey~\cite{rackmonkey} is an opensource project whose main goal is to make hardware management easy.
Unfortunately, at the time we were evaluating the existing tools, the Rackmonkey had no support for machines with a
different form factor than ``pizza boxes'', like blades or the twin servers, and no features which could be used for
managing other kinds of equipment.

\subsection{IBM xCat}

IBM xCat~\cite{xcat}, the ``Extreme Cluster Administration Tool'',\footnote{The current meaning of the abbreviation is
``Extreme {\em Cloud} Administration Tool, apparently to better match current buzzwords in the industry.} is a promising
project with a philosophy which is very close to the traditional, old-school Unix system administrator.  At the same
time, it is optimized for central {\em deployment} and {\em management} of the individual nodes, which are tasks that
are already well handled at the Institute.


\section{WLCG-specific Alternatives}

\subsection{Quattor}

Quattor~\cite{quattor} is an opensource project with roots in the WLCG grid community.  The biggest advantage of this
tool is its diverse community which nowadays include developers from other industries, including several financial
institutions.  However, the project has rather wide scope and --- similarly to the xCat --- delivers a complete,
integrated solution for multiple tasks, including fabric management.

\subsection{Smurf}

The Smurf project~\cite{smurf} is another tool which is produced by the WLCG community.  Developed in the CC-IN2P3 in
Lyon, France, it sports a modular architecture and appears to be well thought-through.  Unfortunately, at the time we
were evaluating the existing rojects, the Smurf project has been undergoing a major overhaul and there were essentially
no documentation.

\section{Augmenting What Works Well: Nagios, Ganglia, Cfengine, Puppet,\ldots}

There are many existing opensource tools which assist in the day to day operation of a massive computing infrastructure;
some of them are intended as monitoring appliances \cite{munin} \cite{ganglia} \cite{nagios} \cite{mrtg}, others aim to
streamline the propagation of changes from a central place to individual nodes \cite{cfengine} \cite{puppet}.  Deska
does not plan to replace these tools, but to {\em leverage} their power and use them, instead of reinventing the wheel.


\end{document}
