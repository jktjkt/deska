% vim: spelllang=en spell textwidth=120
\documentclass{article}
\usepackage[utf8]{inputenc}
\usepackage[czech,english]{babel}
\usepackage{a4wide}
\usepackage{listings}
\usepackage{longtable}
\usepackage{etoolbox}
\usepackage{lmodern}
\usepackage[T1]{fontenc}
\usepackage{textcomp}

\begin{document}

\title{Deska: Objects and Relations}

\author{Jan Kundrát}

\maketitle

\newcommand{\deskaFuncRef}[1]{{\tt {#1}}}

\begin{abstract}
The Deska database can store data of varying structure.  In this chapter, we provide an overview of what can be stored,
and how to use the provided facilities to design a database scheme which closely matches the real-world processes.
\end{abstract}

\section{Objects and Relations: the basic building blocks}

Deska does not actively try to deny its background in relation databases, and therefore it comes at no surprise that the
basic building blocks, the Deska {\em objects}, {\em kinds} and {\em relations} all have their well-known parallels in
the relation database world.

Anything which is stored in the Deska database is an {\em object}. An object can be thought of as a row in a relational
database table -- all objects of the same type, or {\em kind} as we call that in the Deska jargon, live in the same
table, and they all can store the same amount of information in their {\em attributes} -- the database
columns.  One of the columns is usually a primary key, which in Deska corresponds to a special attribute called {\em
name}.  This name is used to uniquely identify an {\em object} in a {\em kind}, similarly to how a primary key is a
unique reference to a single row in a database table.

The primary key on its own is rather meaningless, though -- one has to know what table it belongs to in order to do
anything useful with this information.  The situation is the same in Deska, where one has to always pass a tuple of
identifiers when trying to identify an object -- the name of the {\em kind} and the identifier of the object's {\em
name}.

Similarly to how a typical RDBMS hosts many tables, Deska usually works with many {\em kinds} at one.  All tables are
born equal, and hence even the Deska kinds all exists on the same level of a virtual hierarchy.  Even though the
real-world is full of examples where something ``belongs'' to something else, like an {\tt eth0} interface could belong
to machine called {\tt www}, there's no such thing at this level of the Deska interface -- when you want to work with a
particular object, perhaps the {\tt www}'s {\tt eth0}, you always have to use a globally unique name as a reference.
Of course, all of this is hidden (and much else provided by) by the upper layers thanks to the concept of {\em
relations}.

\section{Relations}

A {\em relation} is something which exists between two {\em kinds} and can be used to model wildly differing scenarios.
A relation consists of the left and right {\em kinds} to which it refers, and the associated {\em type} of the
relation\footnote{This is further augmented by auxiliary information of which object attribute is used by the relation,
but all of that can be safely ignored for now.  Don't worry, we'll get back to that later.}.

The relations serve different purposes, all of which are described in this section.  Some of them are suitable for
maintaining referential integrity, others serve as ``context helpers'' for humans to work with, yet others can be used
to build complex inheritance and composition patterns, common to the object-oriented programming.

\subsection{Referential Integrity}

Let's start with perhaps one of the simplest relation, the {\tt REFERS\_TO} one.  The sole purpose of this relation is
to make sure that an object's attribute does not contain just an ordinary text, but that the value of said attribute
shall be checked that it refers to an instance of some other kind.

There are many use cases for such relation, the most basic one is probably the DB-way of enumerations.  Suppose we have
a hardware inventory and want to keep track of vendors or manufacturers.  It would be impractical to modify the database
code when a new company selling servers emerges, so it makes sense to treat the ``manufacturer'' as data.  This is
simply done by creating a new kind, let's call it ``vendor'', and instantiating new objects for each supplier currently
in use.  The {\tt machine} kind which typically holds objects representing each physical server would then gain a new
attribute called {\tt vendor} with type {\tt identifier}, which will make sure that the possible values have to conform
to the syntax constraints defined for identifiers.  For the referential integrity to work, a new {\em relation} shall be
created.  This new relation shall have the type {\tt REFERS\_TO}, and be from the kind {\tt machine} to kind {\tt
vendor}.

When a new delivery from a brand new company arrives, activating that new manufacturer is simply a matter of
creating new object of the {\tt vendor} kind.

\subsection{Relations Explained}

Each relation therefore gives a potential that objects of a particular kind cold somehow refer to objects of some other
kind (or of the same kind, for that matter).  It is important to understand that this relation is {\bf optional}, and
can be present or absent on a per-object basis.  Under the hood, it is implemented as a database foreign key, and as the
foreign keys can be {\tt NULL} unless explicitly prohibited, an object can easily exist without that particular
relation.

The foreign key is a column similar to other object properties.  It has its own name, which might or might not be the
same as the name of the target kind.  The established Deska conventions call for either using a name which matches the
purpose of the relation (like {\tt template}, {\tt inherit} etc.), or simply reusing the name of the other kind.
There are certain situations (which are indeed present in the sample schemes, as shipped with Deska) which warrant using
a completely different name, so implementors shall refer to their best judgement when inventing naming conventions.

\subsection{Multi-value references}

So far, we have shown relations which generally represent a N:1 mapping -- an object of the source kind (a row in the
source table) refers to at most one of the target objects.  However, this concept is not sufficient for a particularly
compelling feature, notably for {\em tagging}.

Tagging is probably a well-known concept from the general-purpose computing -- an object can be tagged by a set of
labels, which generally refer to identifiers of other objects.  In real life, a photo can be tagged with names of people
on the picture.  In a data center, a server can be tagged with several roles that it has to fulfil; for example, machine
{\tt srv012} can act as a web server and an MX host at the same time.

In Deska, this feature is provided by the combination of the {\tt identifier\_set} attribute data type and the {\tt
REFERS\_TO} relation.  The purpose of this is to hide away the fact that the relational databases tend to model the M:N
relation, as used in tagging, with an auxiliary table.  The Deska DBAPI provides means of accessing the contained
information without exporting the auxiliary table at all.

\subsection{Templates and default value inheritance}

FIXME

\subsection{Composition}

FIXME

\subsection{Object Embedding}

Earlier in this chapter, we have mentioned that there is no structure in the kinds, and that as a result, all objects
have to bear a name which is unique in the given kind.  However, not only the real world in all its complexities, but
also a simple data center is full of examples where this naming convention would get in the way; the above mentioned
example of a network interface name is a prime example.  There are many ways out of that; the most obvious ones involve
GUIDs and other forms of human-unfriendly means.  In Deska, we have chosen another approach which is assisted by the
{\tt EMBED\_INTO} relation, and is hopefully more readable to humans.

An embedded object ``almost'' violates the limitation that an object's name has to be unique in a given kind.  Instead,
the global name is composed of two parts: the name of the {\em parent} object and a logical name, which has to be valid
only in the context of a parent, which are then joined together by a separator {\tt ->}.  In practice, {\tt srv012}'s
network interface {\tt eth0} would be therefore called {\tt srv012->eth0}.  Under the hood, this is implemented by
two of attributes, and the standard rules for constraint checking guarantee that there is no potential for a possible
inconsistency.

In practice, this relaxed naming is still not enough to be directly usable by people, and hence the Deska CLI (and any
other GUIs shipped with Deska, for that matter) fully support the object embedding with this relation.  There is no need
to manually type composed identifiers, as the object nesting ``just happens''.

The object embedding also has a feature that allows generating locally unique names for the local part of the name.
That features comes handy when a DB scheme uses object embedding e.g. for tracking failures and other events associated
with a piece of hardware.  The Deska way of handling this use case is to create a kind (let's call it {\tt failure}) and
have it embedded into the parent object (like {\tt hardware}).  However, doing so without an additional assistance
provided by the CLI or the Deska server would impose an annoying burden on users, who would have to come up with a
unique name when logging a ``failure'' of an equipment.  To address this limitation, the Deska DBAPI call for creating
new objects (\deskaFuncRef{createObject}) has an option for automatically assigning a locally-unique name for the object
just created.  This means that the user just has to ``create a failure for machine {\tt srv012}'', and the Deska DB
itself will assign an identifier like {\tt srv012->1} to it.  This feature is directly available from the CLI, and the
user does not have to deal with low-level naming conventions at all.

\section{Attributes}

In the previous sections, we have often used the term {\em attribute} without properly codifying what we refer to,
relying on the intuitive definition.  For most purposes, such a definition is enough, but let's reiterate what an
attribute really is and what data it can hold.

FIXME: do it!

\end{document}
