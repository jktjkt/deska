% vim: spelllang=en spell textwidth=120
\documentclass[deska]{subfiles}
\begin{document}

\chapter{User's Guide}
\label{sec:user-intro}

\begin{abstract}
This section provides a quick overview of the tasks which a typical user performs when using the Deska system.
\end{abstract}

The Deska system can be essentially divided into two big parts: the CLI library (see \secref{sec:cli-usage}), which is
used by the Unix system administrators to access the database, perform modifications, query the data etc., and the
Deska server which implements object storage (see \secref{sec:admin-dbscheme}) and integrates the whole business logic.
In order to use Deska, both components have to be available and well-configured.

\section{The Deska Work Flow}
\label{sec:user-workflow}

The goal of the Deska system is to put all the important information about a data center into a single place, and make
sure that the changes which are performed in the contained data are deployed to various places throughout the data
center.  A set of components works in concert towards this goal. We will illustrate how a basic change, like adding a
new machine to the data center, is performed using Deska, and how is the change reflected in production systems.

Some of the details might not become immediately clear when reading this section.  All of the required concepts of the
Deska operation are explained in much more detail in the rest of this documentation, especially in the
\secref{sec:chap-objects-and-relations}.

\subsection{Step Zero: Planning}

The Deska database can be used even before the new hardware is ordered.  If properly maintained, the Deska contains data
about all physical equipment, including their dimensions, placement in racks, power consumption and network
interconnects, which together place constraints about whether a data center could actually {\em accommodate} the
upcoming delivery.  Users can verify that the target racks actually have some free space, the Ethernet/FC switches have
available (and licensed!) ports, and that the total allowed power consumption of a rack, or a room, is not exceeded.
Experience at various sites have shown that catching these kinds of errors early can save substantial expenses at a
surprisingly little costs of time and effort.

\subsection{Step One: Physical Installation}

When the managers had seen strong evidence that the new delivery is going to fit the existing system and the order has
been delivered, the machines will have to be installed at the target racks by the suppliers or third-party contractors.
At this time, the Deska {\em users} shall insert records about the new delivery into the Deska database.

It might happen that the new hardware is of a type which has not been delivered before.  In these cases, it would be
necessary to create proper representation of the new hardware model.~\footnote{Common to any inventory management
system, the initial time investment is proportional to the required usefulness of the system; being able to distinguish
between sub-families of a particular hardware brand directly translates to increased initial overhead when processing
the delivery.  Finding a proper balance between managing too much detail and wasting resources on one side and not
describing the world sufficiently is, unfortunately, a human problem which is not solved by any computer system.}  This
is how one might add information about the newly delivered model.  The commands being used are described in much more
detail later throughout these docs, so reads shall not give up when the exact meaning of this example is not immediately
clear.

We will start with launching the {\tt deska-cli}.  At first, we determine whether the database already contains
something about the ``HP ProLiant DL380 G6'', the model which we have ordered.  We will make use of the CLI's tab
completion, where pressing the {\tt Tab} key at any time offers a context-sensitive list of possible completions:

\begin{minted}{console}
> modelhardware HP-
HP-BL20p                           HP-ProLiant-DL320-G6
HP-BL35p                           HP-ProLiant-DL360-G4p
HP-BL460c                          HP-ProLiant-DL380-G5
HP-BL465c                          HP-ProLiant-DL380-G6
\end{minted}

Our intended object appears to be in place.  Let's verify that it indeed holds a correct data:

\begin{minted}{console}
> show modelhardware HP-ProLiant-DL380-G6
cpu_cores 4
cpu_ht true
cpu_sockets 1
cpu_type "Intel Xeon 2.2 GHz"
hdd_drive_capacity 300
no hdd_drive_count
hdd_note "SAS 15k RPM (8x)"
no hepspec
modelbox 2u
note "SAS 15k RPM (8x)"
power_max 286
no power_supply_count
ram 6
vendor HP
no weight
\end{minted}

Everything looks correct so far, and the auxiliary data are in place.  It is therefore time to actually put in the
information about each individual machine.  Here is how this could be done by the Deska CLI (again, the details will be
explained later on):\footnote{The examples have been slightly altered from the real FZU scheme for better readability.
For example, a real scheme uses simply {\tt y} instead of {\tt rack\_pos\_y} for brevity, etc.  These differences are
mere cosmetic changes, though.} \footnote{The {\tt start} command is necessary to open a {\em changeset}, and thereby
switch the database to a mode in which it accepts requests for modifications.  This workflow is described later in this
documentation.}

\begin{minted}{text}
> start
Changeset tmp4 started.
tmp4> hardware srv123
Object(s) hardware srv123 do(es) not exist. Create? y
tmp4:hardware srv123> inside L01
Object(s) rack-placement srv123 do(es) not exist. Create and connect to hardware srv123? y
tmp4:hardware srv123> rack_pos_y 12
\end{minted}

\subsection{Step Two: Verifying the Results}

Now the database contains the data about the new arrivals.  Even though that we have not configured all of the required
properties, we can already check the {\em impact} of the change which we have just performed.  We will start with
reviewing the modifications performed in the Deska database:

\begin{minted}{diff}
tmp4:hardware srv123> diff
+ box srv123
+     hardware srv123
+     inside L01
+     y 12
+ end
+ hardware srv123
+     box srv123
+     modelhardware HP-ProLiant-DL380-G6
+ end
\end{minted}

The changes look reasonable, they affect just the systems which have just arrived, and there are no extra changes nor
typos.  We shall therefore check about how are these modifications going to affect the {\em output configuration} of
various services:

\begin{minted}{diff}
tmp4:hardware srv123> configdiff
diff --git a/RackView.qml b/RackView.qml
index c34796f..bccd828 100644
--- a/RackView.qml
+++ b/RackView.qml
@@ -83,6 +83,12 @@ height: 700
       rackY: 36
     }
     DeskaDividedBox {
+      id: box_new_delivery
+      name: "srv123"
+      rackY: 36
+      consumesBaysY: 2
+    }
+    DeskaDividedBox {
       id: box_hyperwin
       name: "hyperwin"
       rackX: 0
\end{minted}

The changes look like a direct result of the newly added data, and therefore we are free to promote this state of the
database into the production environment:

\begin{minted}{text}
tmp4:hardware srv123> commit
Commit message: Adding a box for the new delivery
Changeset tmp4 commited to r4.
>
\end{minted}

\subsection{Step Three: Affecting Production}

What we, regular users, have done so far is simply performing a small change in the database, and letting Deska
``propagate it further''.  This last part is something which deserves further description.

First of all, the Deska system itself does not have any knowledge of how to convert the data in the Database into the
usable output.  This task is handled by the {\em configuration generators}, which are described in much more detail in
\secref{sec:config-generators}.  A sample set of these generators, as used at FZÚ, is shipped with Deska (see
\secref{sec:fzu-cfggen}), but chances are that these scripts will have to be modified to suit each site's operation and
local specialities.

When the configuration generators are configured and executed, their output has to be somehow put into the use.  This is
achieved in two steps.  In the first step, the resulting data are {\em published} somewhere.  It is assumed that each
site already has their own configuration management system, and that it is managed by a version control system of some
kind.  The Deska ships with a set of scripts which achieve full integration with the Git~\cite{git} version control
system, which means that each time a Deska user saves the data into the Deska database, the output of the configuration
generators is automatically pushed into a Git repository.

The final thing, which is {\em not} handled by the Deska system, is actual delivery of the updated configuration to the
target systems, like the DNS servers or network switches.  In this case, the Deska system is trying hard to {\em
integrate} with the existing tools like Puppet~\cite{puppet} or Cfengine~\cite{cfengine}, and not reinvent the wheel.
These systems are generally mature enough and work well, and replacing them would therefore be clearly out of scope for
the Deska.

\section{Users and Their Roles}

Throughout this document, we will assume that two sorts of people are going to use Deska on the daily basis: the Deska
{\em administrator}, a person who is, generally speaking, ``in charge'' of the whole system, configures the database
scheme, performs backups and grants access to regular users, and the Deska {\em user}, someone whose only means of
access to the Deska database is through a dedicated application.\footnote{To be technically correct, the only means of
access to the Deska database for a regular user is over the DBAPI protocol (see \secref{sec:dbapi-protocol} for
details).  The user {\em could} theoretically hand-craft the JSON commands by hand, and the Deska server would have no
way to distinguish them from an access through an approved application.  This situation is similar to a case of a web
application which implements user authentication/authorization on top of the HTTP protocol, and does not care about a
particular browser being used.}

\subsection{Deska Administrator}

The Deska administrator is responsible for the initial setup and the day-to-day operation of the Deska database.  She
will have to install the application on the server (\secref{sec:building}), setup the server to connect to a PostgreSQL
database(see \secref{sec:build-server-setup}), deploy an appropriate database scheme (\secref{sec:admin-dbscheme}),
manage user accounts and periodically create backups (\secref{sec:cli-db-backup}) and store them in a safe location.

A good administrator will have some knowledge of the SQL language, as spoken by the PostgreSQL~9~\cite{postgresql}.  If
the default database scheme, as shipped with Deska, is not sufficient for a specific environment, the administrator will
also be in charge of the design of this scheme, which is a task which might require some experience with designing
database applications.

The administrator is also responsible to set up the output configuration generators, scripts which actually turn the
data in the Deska database into usable configuration for the DNS servers, Ethernet switches, power outlets etc.  A set
of sample scripts, as used in real-world scenario at the Institute of Physics, are discussed in \secref{sec:fzu-cfggen}.
They will absolutely require adaptation to the target environment, but could serve as a valuable basis for further
development.

\subsection{Regular User: a Unix System Administrator}

In contrast to the administrator, the regular user which is typically each and every Unix system administrator on site,
does not have to worry about the Deska server and its setup at all.  It is assumed that she will be given the
appropriate connection details for configuring the command-line application (see \secref{sec:cli-connection-setup}), and
also that she will be briefed by the administrator about how the database scheme works.

Nevertheless, the regular users shall get themselves very familiar with the general Deska concepts, especially as
described in \secref{sec:chap-objects-and-relations}.  Thorough understanding of {\em objects}, {\em kinds} and {\em
relations} is crucial for effective working with the Deska system.

These users shall also be familiar with the configuration of various services which are deployed at their site, and
especially shall be able to assess the impact of the changes in the generated configuration files.  Similarly to other
systems which use a central place to specify an {\em authoritative} answer about the data, performing careless changes
in the Deska database could have devastating consequences to the operation of the user's site.

That said, no change is ever going to affect the production environment until the user explicitly confirms the result.
To decrease the chances of disastrous errors, the Deska system will always present the user with a summary of changes in
the database, along with their impact to the output configuration, before the data are saved and promoted to production.

\end{document}
