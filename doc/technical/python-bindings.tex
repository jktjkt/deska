% vim: spelllang=en spell textwidth=120
\documentclass[deska]{subfiles}
\begin{document}

\chapter{Python Bindings}

\begin{abstract}
In this chapter, we deal with the Python bindings for most of the functions from the  {\tt DeskaDb} library.
\end{abstract}

The Python bindings for the {\tt DeskaDb} library are consisted of two layers.  The lower-level one is maintained
manually with help of the Boost.Python package, and contains essentially a one-to-one mapping of the interesting bits of
the C++ implementation to the Python code.  At this layer, we provide the low-level building blocks, from simple
bindings of the class methods to explicit conversion functions which deal with the differences between the Python and
C++ type system.  In contrast to the lower-level functions, the second half of the Python code provides a high-level
access to the Deska objects form the database.

\section{Low-level Bindings}

The lower-level Python bindings are implemented in C++ with help of the Boost.Python library.  The functionality we
implement is centered around the {\tt Deska::Db::Connection} class and its methods which provide a read-only access to
the data in the Deska database.  Functions which work on various C++ data types, including the variants such as the {\tt
Deska::Db::Value}, are provided at this place.  The source code is in the {\tt src/deska/LowLevelPyDeska} directory.

The code is made accessible to Python via the {\tt libLowLevelPyDeska.so} shared module, which can be directly imported
from Python.  Working with these functions directly is not recommended for the end users of the Deska system, as the
types mimic the C++ classes rather closely, not following much of the Python conventions, as indicated in the following
example.  The code shown here is a real excerpt from the Deska code which transforms the result of the {\tt objectData}
functions into native Python objects:

\begin{minted}{python}
import libLowLevelPyDeska as _l

def _map_to_class_with_values(d):
    """Convert a map<string, Value> into a class with native Python data"""
    res = _KindInstanceInResult()
    for x in d:
        val = _l.DeskaDbValue_2_Py(x.data())
        if isinstance(val, _l.std_set_std_string):
            val = [item for item in val]
        setattr(res, x.key(), val)
    return res
\end{minted}

As we have already mentioned, this form is rather hostile to the potential user, and its usage might be rather error
prone.  This is why it is further wrapped by the upper layers of the Python interface.

\section{High-level Python Code}

In contrast to the lower-level stuff described in the previous section, which mimics the C++ interface rather closely,
the high-level Python API is generated completely automatically at runtime and is intended to closely match the Deska
database scheme.  To reach this goal, this layer uses the metadata discovery features (see
\secref{sec:api-group-dbscheme}) to create a hierarchy of Python classes, each for a distinct kind.  These classes then
provide an interface for iterating over all objects of a kind, or for performing server-side filtering.  The returned
data is always presented in a form which is native to Python; this means that the C++ timestamps are converted to the
{\tt datetime.datetime} values etc.

\subsection{Server-side Filtering}

In the following example, we iterate over all hosts in the Deska database whose name is not ``foo''.  Please note that
the Python bindings have no hardcoded knowledge of a ``host'', and everything is deduced from the database scheme on the
fly, without any manual intervention.  The high-level Python bindings are capable of converting any valid Deska database
scheme to native Python objects without any user effort at all:

\begin{minted}{python}
import deska
# Discover the DB scheme & create classes.
# The whole scheme discovery is hidden behind this call.
deska.init()

# Iterate over all hosts in the DB whose name is not equal to the given string.
# Please note that the filtering is done completely on the server side, as the
# search condition is transparently converted to a Deska ``filter''.
for host in deska.host[deska.host.name != "foo"]:
    # Let's just print each host's name
    print "%s\n" % host
\end{minted}

\subsection{Iterating over Everything and Data Access}

The ability of performing server-side searches has a significant advantage when the amount of data to process is huge.
However, in certain situations, it might make more sense to delegate the filtering to native Python code, or just skip
the filtering altogether.  The following example demonstrates how to iterate over all objects of a given kind.  In
addition, we also print object attributes along with their name to show how to access these values:

\begin{minted}{python}
# Same as above
import deska
deska.init()

# The magic .all() method lets you iterate over all objects
# of a given kind.  Object attributes are available as the regular
# class members.  Remember that Python by default iterates over
# keys of a container, so you have to explicitly use iteritems().
for (name, host) in deska.host._all().iteritems():
    print "%s: %s\n" % (name, host.note_host)
\end{minted}

Iteration over all items is allowed by using the method {\tt \_all()} instead of an filter expression.  The iteration
itself follows the traditional Python rules, ie. by default it iterates just over the host names.  Use the {\tt
iteritems()} function to iterate over a list of (name, data) pairs.

\subsection{Native Data Types}

The high-level Python code also implicitly converts the data stored in object attributes into native Python values.
This means that the timestamps are converted into the {\tt datetime.datetime} objects,  Unfortunately, Python lacks a
native data type for an IP address (both IPv4 and IPv6) and Ethernet MAC addresses.  Therefore, these are passed to the
user as Python strings.  This behavior might change in future versions of Deska or when native Python classes for
Internet addresses become common.

\end{document}
